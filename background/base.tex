%%%%%%%%%%%%%%%%%%%%%%%%%%%%%%%%%%%%%%%%%%%%%%%%%%%%%%%%%%%%%%%%%%%%%%%%%%%%%%%%
\chapter{A Review: The Recovery of 3D Facial Shape From Images and Videos}\label{ch:background}
%%%%%%%%%%%%%%%%%%%%%%%%%%%%%%%%%%%%%%%%%%%%%%%%%%%%%%%%%%%%%%%%%%%%%%%%%%%%%%%%
\minitoc{}
%%%%%%%%%%%%%%%%%%%%%%%%%%%%%%%%%%%%%%%%%%%%%%%%%%%%%%%%%%%%%%%%%%%%%%%%%%%%%%%%
The recovery of 3D facial shape from images and videos has a rich history
that is closely interwoven with the more general area of generic 3D shape
recovery. Due to the many applications of 3D facial data and the
relative ease of facial data collection, many generic methods for 3D shape
recovery have given examples of reconstructions for faces. Therefore, it
becomes difficult to separate methods that focus on 3D facial shape recovery
and more generic methods that provide facial shape recovery results
amongst many other examples. This is particularly prevalent in the Structure-
from-Motion (SfM) literature whereby few papers focus solely on face recovery
yet many papers show examples of facial reconstructions. In order to provide
a thorough treatment on both the history and the current state-of-the-art for
3D facial shape recovery, it is necessary to isolate key examples
of these more generic methodologies and present their application to faces.
Furthermore, there are examples of 3D facial shape recovery across all methods
of general shape recovery given in \cref{tbl:3d_recovery_methods} and thus
the relevant literature is vast. In this section, we will attempt to provide
an exhaustive review of methods for 3D facial shape recovery and will provide
literature for both sparse and dense shape recovery, despite the focus
of this thesis being on dense recovery. Given the scope of the review, the
literature will be separated into the categories provided by
\cref{tbl:3d_recovery_methods}. Each subsection will briefly describe the
key mechanics involved for the reconstruction method but will not focus on
an exhaustive review of all of the relevant literature for a given method.
Instead, only those methods that focus on facial shape recovery or provide
representative reconstructions of faces will be considered.

The rest of this section will proceed as follows. \cref{sec:bg_db_capture} will
discuss the currently available databases of 3D facial shape in the form
of meshes and reflectance information. It will also briefly touch on recent
innovations for capturing these databases. \cref{sec:bg_sfs} provides an overview
of Shape-from-Shading (SfS) methods for facial surface normal recovery from
single images. \cref{sec:bg_ps} discusses both calibrated and
Uncalibrated Photometric Stereo methods. When discussing Uncalibrated methods,
more recent large-scale methods from Internet images are also included.
\cref{sec:bg_3dmm} presents the relevant literature for Morphable Models,
beginning with work of \citet{volker1999morphable}.
\cref{sec:bg_model_based} gives an
overview of methods that attempt to directly infer shape information from images
by using models. \cref{sec:bg_sfm} discusses Structure-from-Motion methods,
including relevant methods that provide representative reconstructions
on faces despite not leveraging facial models.
%%%%%%%%%%%%%%%%%%%%%%%%%%%%%%%%%%%%%%%%%%%%%%%%%%%%%%%%%%%%%%%%%%%%%%%%%%%%%%%%
%%%%%%%%%%%%%%%%%%%%%%%%%%%%%%%%%%%%%%%%%%%%%%%%%%%%%%%%%%%%%%%%%%%%%%%%%%%%%%%%
\section{Databases And Capture Methods}\label{ch:bg_db_capture}
%%%%%%%%%%%%%%%%%%%%%%%%%%%%%%%%%%%%%%%%%%%%%%%%%%%%%%%%%%%%%%%%%%%%%%%%%%%%%%%%
In contrast to generic shape recovery, facial shape recovery implies a strong
assumption that the input image contains a face. This prior knowledge, that the
image contains a face, enables the use of training data to improve the
quality of the reconstruction. However, collection of this training data, which
may take the form of either direct 3D information (meshes, range images) or
functions of the facial surface (normals, parametric reflectance data) must
use accurate shape recovery methods in order to be useful. Therefore, the
collection of accurate facial shape for the purposes of providing approximate
ground truth data is a research topic in its own right. In this section, we
begin by discussing the existing 3D facial databases and the manner of their
collection. Following this, we discuss the details of the 3D recovery algorithms
and focus on works intended specifically for facial surface recovery.
%%%%%%%%%%%%%%%%%%%%%%%%%%%%%%%%%%%%%%%%
% CANDIDE 3d wireframe model
% MPI     7 views, 200 laser scanned heads, 5 full 3D, Cyberware, static, neutral
% ND-D    953 scans, Minolta Vivid 900 3D scanner, 656 subjects, neutral, static
% ND-2006 13,450 images containing 6 different types of expressions, 888 distinct people
%         Static, Minolta Vivid 910 range scanner
% 3D-TEC 212 individuals (106 twins),  Minolta VIVID 910 3D scanner, 1 neutral, 1 smiling, static
% CASIA 3D 4624 scans of 123 persons, Minolta Vivid 910, various poses, emotions, static
% B3D(AC)^2 scanned using \cite{weise2007fast}, 14 subjects, 1109 sequences, dynamic 25 FPS, reading
% BIWI Head Pose  15K images of 20 people, various head poses, missing frames, kinect
% UOY Face Database 97 people, multi-pose, neutral, smile and frown
% GavabDB 61 individuals, 427 images, multi pose, static, 3 expressions
% FRAV3D 106 subjects,  MINOLTA VIVID-700, multi-pose, static
% Basel 200 scans, neutral, ABW-3D structured light scanner
% BJUT-3D, cyberware
% EURECOM Kinect, 52 people * 18 image, multi poses, expressions, static
% XM2VTS stereo, 25 subjects, neutral,
% Texas 3DFRD 1149 pairs of facial color and range images of 105 adult human subjects,
%             stereo imaging system manufactured by 3Q Technologies, neutral, frontal
% ZJU-3DFED  40 subjects, total 360, expressions, InSpeck 3D MEGA Capturor DF
% 3D_RMA structured light, 120 people, 6 shots, multi-pose
% Beckmann cyberware, 400-500, single shot
% UMB-DB Minolta Vivid 900 laser scanner
\begin{table}
\centering
\resizebox{\textwidth}{!}{%
\begin{tabular}{@{}cccccccc@{}}
\toprule
\multicolumn{3}{c}{DB}                                                              & \# Subjects & \# Outputs  & Expressions  & Poses        & Type   \\ \midrule
CANDIDE v1                  &\cite{Rydfalk:1987tg}          & 1987                  & 1           & 1           & $\checkmark$ & ---          & MO     \\ \midrule
MPI                         &\cite{Troje:1996ep}            & 1996                  & 200         & 200         &              &              & M      \\ \midrule
CANDIDE v2                  &\cite{Ahlberg:1998uk}          & \multirow{2}{*}{1998} & 1           & 1           & $\checkmark$ & ---          & MO     \\
3D\_RMA                     &\cite{beumier2001face}         &                       & 120         & 720         &              & $\checkmark$ & D      \\ \midrule
USF                         &\cite{RefWorks:96}             & \multirow{3}{*}{1999} & 200         & 200         &              & ---          & MO     \\
XM2VTSdb                    &\cite{messer1999xm2vtsdb}      &                       & 295         & 2360        &              &              & M      \\
YaleB                       &\cite{RefWorks:314}            &                       & 10          & 5760        & $\checkmark$ &              & PS     \\ \midrule
CASIA 3D                    &\cite{casia3d}                 & \multirow{2}{*}{2003} & 123         & 1845        & $\checkmark$ &              & D      \\
FSU                         &\cite{hesher2003novel}         &                       & 37          & 222         & $\checkmark$ &              & D      \\ \midrule
GavabDB                     &\cite{moreno2004gavabdb}       & 2004                  & 61          & 720         & $\checkmark$ & $\checkmark$ & D      \\ \midrule
FRGC v2                     &\cite{phillips2005overview}    & 2005                  & 466         & 4007        & $\checkmark$ &              & D      \\ \midrule
BU3D-FE                     &\cite{Yin:2006cc}              & \multirow{4}{*}{2006} & 100         & 2500        & $\checkmark$ &              & M      \\
ND-2006                     &\cite{faltemier2007using}      &                       & 888         & 13450       & $\checkmark$ &              & D      \\
FRAV3D                      &\cite{conde2006multimodal}     &                       & 106         & 1696        & $\checkmark$ &              & M      \\
ZJU-3DFED                   &\cite{wang2006exploring}       &                       & 40          & 360         & $\checkmark$ &              & M      \\ \midrule
Beckmann                    &\cite{hu2007building}          & 2007                  & $\sim$400   &~?           & $\checkmark$ &              & M      \\ \midrule
Bosphorus                   &\cite{Savran:2008gg}           & \multirow{3}{*}{2008} & 105         & 4652        & $\checkmark$ &              & M      \\
UoY                         &\cite{heseltine2008three}      &                       & 350         & 5250        & $\checkmark$ & $\checkmark$ & M      \\
BU4D-FE                     &\cite{yin2008high}             &                       & 101         & Many        & $\checkmark$ & $\checkmark$ & 4D M   \\ \midrule
Basel                       &\cite{paysan20093d}            & \multirow{2}{*}{2009} & 200         & 200         &              & ---          & MO     \\
BJUT-3D                     &\cite{baocai2009bjut}          &                       & 500         & 500         &              &              & M      \\ \midrule
B3D (AC)\textsuperscript{2} &\cite{fanelli20103}            & \multirow{2}{*}{2010} & 14          & Many        & $\checkmark$ &              & 4D D/M \\
Texas 3DFRD                 &\cite{gupta2010anthropometric} &                       & 118         & 1149        &              &              & M      \\ \midrule
3D-TEC                      &\cite{vijayan2011twins}        & \multirow{4}{*}{2011} & 212         & 212         & $\checkmark$ &              & M      \\
BIWI Head Pose              &\cite{fanelli2013random}       &                       & 20          & $\sim$15000 &              & $\checkmark$ & D      \\
UMB-DB                      &\cite{colombo2011umb}          &                       & 143         & 1473        & $\checkmark$ &              & M      \\
Florence 3D                 &\cite{bagdanov2011florence}    &                       & 53          & 212         &              & $\checkmark$ & M      \\ \midrule
ICT-3DRFE                   &\cite{stratou2012exploring}    & \multirow{2}{*}{2012} & 23          & 345         & $\checkmark$ &              & PS     \\
Superfaces                  &\cite{berretti2012superfaces}  &                       & 20          &~?           &              & $\checkmark$ & M/D    \\ \midrule                               
Photoface                   &\cite{RefWorks:293}            & 2013                  & 261         & 1839        & $\checkmark$ &              & PS/M   \\ \midrule
FaceWarehouse               &\cite{Cao:2014gy}              & \multirow{3}{*}{2014} & 150         & 3000        & $\checkmark$ & ---          & MO/D   \\
BP4D-Spontaneous            &\cite{Zhang:2014id}            &                       & 41          & Many        & $\checkmark$ & $\checkmark$ & 4D M   \\
EURECOM                     &\cite{min2014kinectfacedb}     &                       & 52          & 936         & $\checkmark$ & $\checkmark$ & D      \\ \midrule
SURREY                      &\cite{Huber:F5Dca9zy}          & 2016                  & 169         & 169         &              & ---          & MO     \\ \bottomrule
\end{tabular}%
}
\caption{A timeline of existing 3D facial databases including depth data, mesh
         data, models and Photometric Stereo images. PS denotes Photometric
         Stereo images, MO is a statistical model, M is a 3D mesh, 
         D is depth/range data and 4D implies a 3D video.}
\label{tbl:timeline_db}
\end{table}
%%%%%%%%%%%%%%%%%%%%%%%%%%%%%%%%%%%%%%%%
%%%%%%%%%%%%%%%%%%%%%%%%%%%%%%%%%%%%%%%%%%%%%%%%%%%%%%%%%%%%%%%%%%%%%%%%%%%%%%%%
\subsection{Databases}\label{subsec:bg_databases}
%%%%%%%%%%%%%%%%%%%%%%%%%%%%%%%%%%%%%%%%%%%%%%%%%%%%%%%%%%%%%%%%%%%%%%%%%%%%%%%%
The collection of 3D facial databases has a rich history
that spans over many decades. \cref{tbl:timeline_db} provides a
timeline outlining these 3D facial databases and the composition of the 
data that they provide.
It is particularly interesting to note that, despite the fact that one of the
first 3D facial models was released in 1987 (CANDIDE~\cite{Rydfalk:1987tg}), the
majority of existing databases were produced in the last decade. The quality
of these facial models has also increased substantially, which is demonstrated
in \cref{fig:db_examples}. 

The CANDIDE~\cite{Rydfalk:1987tg,Ahlberg:1998uk} models were parametrizable
wireframe models that contained a low number of vertices (75--95) and were
designed to be controlled according to pre-defined Action Units. The CANDIDE
model was not captured from real data and thus had limited uses for modelling
identity. Early attempts at recording facial databases focused on the use
of laser scanning technology~\cite{cyberware,minolta} or structured light. For
example, the early work of \citet{beumier2001face} used an internally developed 
structured light system to provide low quality neutral face scans. Structured
light was also used for the Bosphorus~\cite{Savran:2008gg} 
and UoY~\cite{heseltine2008three} databases. One of the most commonly used
parametric 3D models is that of the Basel 3D Morphable Model 
(3DMM)~\cite{paysan20093d} which also utilized a custom structured light
scanner called the ABW-3D scanner. The introduction of the consumer priced
Kinect by Microsoft~\cite{zhang2012microsoft} ensured that structured light
technology was extremely cheap to acquire and thus the BIWI Head
Pose~\cite{fanelli2013random}, EURECOM~\cite{min2014kinectfacedb},
Superfaces~\cite{berretti2012superfaces} and the
FaceWarehouse~\cite{Cao:2014gy} databases used the Kinect for capturing data.

Researchers at the MPI used the \citet{cyberware} laser scanner to capture
neutral faces of high quality which were used in the seminal work of the 3DMM
by \citet{RefWorks:96}. The Cyberware scanner was also employed in the more 
recent works of BJUT-3D~\cite{baocai2009bjut} and Beckmann~\cite{hu2007building}
as it provides high quality shape and texture. Another commonly used
laser scanning system is the Minolta Vivid 900/910~\cite{minolta} which was
employed by Notre Dame University when collecting data for the Facial
Recognition Grand Challenge (FRGC)~\cite{phillips2005overview} and the following
ND-2006~\cite{faltemier2007using}. It was also employed by Notre Dame for the
collection of a twins database (3D-TEC)~\cite{vijayan2011twins}. 
The Konika Minolta was also for the databases of CASIA 3D~\cite{casia3d},
FSU~\cite{hesher2003novel} and the UMB-DB~\cite{colombo2011umb} which is unique
in containing many examples of occlusions. The more advanced
Minolta VI-700 digitizer~\cite{minolta} was used to collect both
GavabDB~\cite{moreno2004gavabdb} and FRAV3D~\cite{conde2006multimodal}.
ZJU-3DFED~\cite{wang2006exploring} used a white-light digitizer to capture
the 40 subjects of their work. 

Stereo systems provide a non-invasive method of capturing that is possible to
set up at relatively low cost by an expert. One of the earliest databases to
provide 3D face estimates via stereo was XM2VTSdb~\cite{messer1999xm2vtsdb}. 
Commercial systems such as the 3dMD scanning system~\cite{3dmd} have been used
by the BU3D-FE~\cite{Yin:2006cc}, Florence 3D~\cite{bagdanov2011florence},
Superfaces~\cite{berretti2012superfaces} and the very recent 3DMM provided by
Surrey University~\cite{Huber:F5Dca9zy}. A similar stereo capture system was was
for the Texas 3DFRD~\cite{gupta2010anthropometric} database. The most common
commercial systems for capturing high-quality 4D (3D video) data employ passive
stereo methods. For example, the commercial
system of \citet{di4d} is capable of capturing high resolution images and
converting them into 3D meshes at 60 frames per second (fps). 
The Di4D~\cite{di4d} scanning system
has been used for the only publicly available 4D databases of BU4D-
FE~\cite{yin2008high} and BP4D-Spontaneous~\cite{Zhang:2014id}. BP4D-Spontaneous
is unique in that it contains spontaneously incited emotions rather than the
posed expressions that are more common in other existing databases. A custom
system combining stereo and active light was used by the B3D
(AC)\textsuperscript{2}~\cite{weise2007fast,fanelli2013random} database to
provide 17 FPS scanning of faces.

There have also been efforts to collect databases using illumination constraints
whereby 3D shape is generally recovered using Photometric Stereo (PS) 
algorithms. One of the earliest examples of such a database is the 
YaleB~\cite{RefWorks:314} database which contains 10 subjects captured in 
9 poses each under 64 illumination conditions. A large scale PS database was
provided by the Photoface database~\cite{RefWorks:293} which provided a low-cost
and practical methodology for capturing illuminated images in a public setting.
The ICT-3DRFE~\cite{stratou2012exploring} database provides multiple expressions
of 23 subjects captured by a 156 light ``light-stage''. Millimetre-accurate
facial shape is recovered through computational stereo and high frequency 
mesoscopic detail is recovered via the spherical lighting conditions proposed
by \citet{ma2007rapid}. The ICT-3DRFE database thus provides a dense 
3D facial mesh, as well as a diffuse and specular texture map and diffuse and
specular normals.
%%%%%%%%%%%%%%%%%%%%%%%%%%%%%%%%%%%%%%%%%%%%%%%%%%%%%%%%%%%%%%%%%%%%%%%%%%%%%%%%
\subsection{Capture Methods}\label{subsec:bg_capture}
%%%%%%%%%%%%%%%%%%%%%%%%%%%%%%%%%%%%%%%%%%%%%%%%%%%%%%%%%%%%%%%%%%%%%%%%%%%%%%%%
%%%%%%%%%%%%%%%%%%%%%%%%%%%%%%%%%%%%%%%%
\begin{figure}[h]
	\centering
	\begin{subfigure}[b]{0.3\textwidth}
		\centering
		\includegraphics[height=1.9in]{background/images/basel}
		\caption{Basel~\cite{paysan20093d}}\label{fig:db_examples_basel}
	\end{subfigure}
	\begin{subfigure}[b]{0.3\textwidth}
		\centering
		\includegraphics[height=2in]{background/images/bu3d}
		\caption{BU3D-FE~\cite{Yin:2006cc}}\label{fig:db_examples_bu3d}
	\end{subfigure}
	\begin{subfigure}[b]{0.3\textwidth}
		\centering
		\includegraphics[height=2in]{background/images/bp4d}
		\caption{BP4D-S~\cite{Zhang:2014id}}\label{fig:db_examples_bp4d}
	\end{subfigure} \\
	\begin{subfigure}[b]{0.3\textwidth}
		\centering
		\includegraphics[height=2in]{background/images/frgc}
		\caption{FRGC v2~\cite{phillips2005overview}}\label{fig:db_examples_frgc}
	\end{subfigure}
	\begin{subfigure}[b]{0.3\textwidth}
		\centering
		\includegraphics[height=2in]{background/images/biwi}
		\caption{BIWI~\cite{fanelli2013random}}\label{fig:db_examples_biwi}
	\end{subfigure}
	\begin{subfigure}[b]{0.3\textwidth}
		\centering
		\includegraphics[height=1.9in]{background/images/ict}
		\caption{ICT-3DRFE~\cite{stratou2012exploring}}\label{fig:db_examples_ict}
	\end{subfigure}
	\caption{Examples of the quality of the facial meshes provided in various
	         databases. The meshes were captured by the following systems:
	         \lsubref{fig:db_examples_basel} is from the Basel 3DMM, 
	         which was manually cleaned and captured using a structured light
	         scanner, \lsubref{fig:db_examples_bu3d} by the 
	         3dMD~\cite{3dmd} active stereo system, 
	         \lsubref{fig:db_examples_bp4d} by the 
	         Di4D~\cite{di4d} 4D passive stereo system, 
	         \lsubref{fig:db_examples_frgc} by the 
	         Minolta~\cite{minolta} laser scanning system, 
	         \lsubref{fig:db_examples_biwi} by the 
	         Microsoft Kinect~\cite{zhang2012microsoft} and 
	         \lsubref{fig:db_examples_ict} by a passive stereo 
	         light stage~\cite{debevec2000acquiring}.}
\label{fig:db_examples}
\end{figure}
%%%%%%%%%%%%%%%%%%%%%%%%%%%%%%%%%%%%%%%%
The majority of the databases discussed previously were collected using
commercial apparatus and the focus of the database was the gathering of the
subjects rather than the capturing methodology itself. However, the capturing of
high quality facial surface information is an active area of research and is of
particular interest to the computer graphics community. High quality facial
surface information including mesoscopic details such as fine wrinkles and pores
are necessary for photo-realistic rendering of faces. Photo-realistic rendering
is widely applicable and has seen applications in
films~\cite{borshukov2005universal}, video games~\cite{vonderPahlen:2014kg},
virtual makeup systems~\cite{scherbaum2011computer} and
animatronics~\cite{jung2011believable}. The recovery of these mesoscopic details
requires careful modelling of the interaction between human skin and light. To
this end, many works focus solely on these reflectance characteristics and seek
to provide highly realistic parametric reflectance functions for skin. In this
review, we are only interested in methods that recover realistic 3D shape and
thus do not consider works that focus solely on reflectance function modelling.
For more information about these methods the interested reader should consult
the recent survey on facial appearance capture by \citet{Klehm:2015jb}.

Given the intended use cases for the captured data, facial capturing has focused
on scenarios where both the illumination and camera parameters are tightly
controlled. For this reason, much of the literature focuses on two primary
areas: structured light and stereo. Passive stereo is a photogrammetric method
that requires 2 or more calibrated cameras who's relative positions are used to
infer the 3D position of an objects surface. A key requirement of passive stereo
is a set of correspondences that are shared across multiple camera views. These
correspondences are defined as a set of 2D coordinates that mark salient areas
of the object, visible across multiple cameras. These 2D points are assumed to
represent the 2D projection of a real 3D point onto each camera plane. It is
these correspondences that are used to form geometric constraints for recovery
of 3D positions. However, the computation of these correspondences is itself 
ill-defined as variation in illumination and orientation may cause
the appearance of a true correspondence to vary heavily. Active stereo is
an extension of passive stereo that attempts to simplify the correspondence 
problem by projecting a pattern onto the object at capture time. The pattern
thus provides less ambiguous texture cues for computing correspondences.
In contrast, structured light only requires a single camera and unlike
active stereo this camera must be calibrated with respect to the position of the 
projector supplying the light pattern. The light patterns projected 
by the projector encode the correspondences and 3D information is recovered 
via triangulation.

\textbf{Passive (computational) stereo} is an extremely common technique for
surface recovery and the geometric relationships between two calibrated cameras
are relatively well
understood~\cite{barnard1982computational,seitz2006comparison}. When passive
stereo is applied to reasonably smooth, convex objects in uncluttered scenes, as
is generally the case when capturing faces, it is largely considered a
technology. Commercial systems such as 3dMD~\cite{3dmd} and Di4D~\cite{di4d} are
employed in many areas of entertainment and have been used to capture many of
the high quality databases in \cref{tbl:timeline_db}. Both of these companies
also provided high frame rate systems that are capable of capturing 3D
information at 60+ fps, albeit temporally consistent meshes still require
further post-processing. For this reason, few works in passive stereo focus
solely on the recovery of macroscopic shape using only stereo. Whilst older
works such as that of \citet{Lengagne:1996ej} required 3D models to perform
inference, such as segmentation, on depth maps, modern depth maps are of much
higher quality. Even relatively recent reviews on the subject of passive stereo
applied to faecs~\cite{Leclercq:2005ee} have become obsolete given the price and
quality of modern digital camera systems. For example, the work of
\citet{Beeler:2010dg} demonstrates that standard passive stereo methods can
recover 3D surfaces that contain many high frequency features such as wrinkles.
The highest quality results presented by \citet{Beeler:2010dg} comes from a 7
camera studio setup where per-pixel correspondences and disparity map refinement
is computed in a coarse-to-fine manner. Further constraints are imposed that are
face specific including a smoothness constraint when computing correspondences
and a second-order anisoptropic surface consistency term that reduces smoothing
across depth discontinuities in the disparity maps. Finally, mesoscopic details
are transferred into the mesh via a photometrically consistent surface
refinement procedure. Although the recovered details are not metrically correct,
they do significantly improve the qualitative accuracy of the recovered surface.
Under the assumption that these mesoscopic variations in intensity are linked to
variations in the geometry of the skin, a high-pass filter is first performed in
order to filter out any details captured by stereo. Then, the result of the
stereo is refined along the surface normal direction and is weighted to ignore
high frequency areas caused by larger features such as hairs.

% Need to heavily edit/rewrite this because I got the classification wrong
Structured light methods were some of the
earliest methods for 3D acquisition~\cite{will1971grid}. These methods became
particularly effective for real-time acquisition~\cite{rusinkiewicz2002real} and
more recently structured light was used in the infrared light spectrum for
the original Microsoft Kinect~\cite{zhang2012microsoft}. Although the quality
of output, as depicted for the BIWI head pose database in 
\cref{fig:db_examples_biwi}, is known to be low in the case of the Microsoft
Kinect, structured light is capable of high recovering high quality macroscopic
shape. For example, the Spacetime stereo method of \citet{Zhang:2005ww} was
effectively applied to faces in~\cite{zhang2004spacetime}. Here, 2 projectors
are used in conjunction with 2 stereo pairs (4 cameras) to perform stereo
on a sequence of images. The use of both structured light and multiple frames
allows for the recovery of high quality facial shape. \citet{zhang2004spacetime}
also propose a global optimisation that helps reduce the ridging artefacts
that are common in structured light geometries. Real time capture of 
high-quality facial shape using structured light has also been 
proposed~\cite{zhang2006high,weise2007fast}. \citet{zhang2006high} proposed
a set of custom hardware the enabled 40 FPS capture and reconstruction. The
reconstruction is computed rapidly via a novel three-step phase shifting
algorithm. \citet{weise2007fast}.
%%%%%%%%%%%%%%%%%%%%%%%%%%%%%%%%%%%%%%%%%%%%%%%%%%%%%%%%%%%%%%%%%%%%%%%%%%%%%%%%

%%%%%%%%%%%%%%%%%%%%%%%%%%%%%%%%%%%%%%%%%%%%%%%%%%%%%%%%%%%%%%%%%%%%%%%%%%%%%%%%
\section{Shape-from-Shading}\label{ch:bg_sfs}
%%%%%%%%%%%%%%%%%%%%%%%%%%%%%%%%%%%%%%%%%%%%%%%%%%%%%%%%%%%%%%%%%%%%%%%%%%%%%%%%
%%%%%%%%%%%%%%%%%%%%%%%%%%%%%%%%%%%%%%%%
\begin{figure}[t]
	\centering
	\begin{subfigure}[b]{0.24\textwidth}
		\centering
		\includegraphics[height=2in]{background/images/frontal}
		\caption*{Frontal}
	\end{subfigure}
	\begin{subfigure}[b]{0.24\textwidth}
		\centering
		\includegraphics[height=2in]{background/images/invert}
		\caption*{Inverted}
	\end{subfigure}
	\begin{subfigure}[b]{0.24\textwidth}
		\centering
		\includegraphics[height=2in]{background/images/frontal_rotate}
		\caption*{Frontal}
	\end{subfigure}
	\begin{subfigure}[b]{0.24\textwidth}
		\centering
		\includegraphics[height=2in]{background/images/invert_rotate}
		\caption*{Inverted}
	\end{subfigure}
	\caption{An example of a bas-relief ambiguity for a mesh illuminated
	         frontally under orthographic projection, with a lambertian shader.
	         ``Inverted'' implies that the mesh is actually facing away from the
	         camera and thus the interior is visible, as demonstrated by the
	         rotated ``Inverted'' image. Both of the non-frontal images are
	         rotated versions of the frontal images, approximately $25^\circ$
	         around the Yaw axis.}
\label{fig:bg_sfs_bas_relief}
\end{figure}
%%%%%%%%%%%%%%%%%%%%%%%%%%%%%%%%%%%%%%%%
Shape-from-shading (SfS)~\cite{horn1970shape} is the process of attempting to
recover surface information from an object in an image using \textit{inverse
rendering}, or \textit{image formation}, methods. The primary assumption is that
shading, or the intensity of a pixel in the image, is generated as a function of
the surface geometry and its interaction with light reflected/absorbed by the
surface and captured by an imaging device. Naturally, the reality of this
process in the physical world is a complex interaction between light and both
microscopic and macroscopic elements of the surface structure. This is further
complicated by the noise present in the recording procedure of the camera
sensing hardware. Furthermore, it is well known that shading alone is
insufficient to disambiguate shape. For example, the well known bas-relief
ambiguity~\cite{belhumeur1999bas} is demonstrated for a facial mesh in
\cref{fig:bg_sfs_bas_relief}. The bas-relief ambiguity states that for an object
imaged under orthographic projection that exhibits Lambertian reflectance, there
exists a family of transformations (generalized bas-relief transformations) for
which the images produced will be identical. In fact, more generally there
exists an infinite number of ways to describe any image given only shading
information through different arrangements of surfaces, lightings and
albedos~\cite{adelson1996perception}. However, despite the ill-posedness of the
SfS problem, shading does in fact provide a very strong imaging prior and many
higher frequency detail such as wrinkles can only be recovered using shading
cues. Before discussing the facial surface recovery literature, we briefly
describe the image formation problem including the common assumptions
made.
%%%%%%%%%%%%%%%%%%%%%%%%%%%%%%%%%%%%%%%%%%%%%%%%%%%%%%%%%%%%%%%%%%%%%%%%%%%%%%%%
\subsection{Image Formation}
%%%%%%%%%%%%%%%%%%%%%%%%%%%%%%%%%%%%%%%%%%%%%%%%%%%%%%%%%%%%%%%%%%%%%%%%%%%%%%%%
%%%%%%%%%%%%%%%%%%%%%%%%%%%%%%%%%%%%%%%%
\begin{figure}
	\centering
	\begin{tabular}{cc}
		\multicolumn{2}{c}{
			\begin{subfigure}[b]{\textwidth}
				\centering
				\caption*{Radiometric Terms}
				\begin{tabular}{@{}lll@{}}
					\toprule
					Term              & Symbol                                      & Unit                 \\ \midrule
					Solid Angle       & $d\omega$                                    & ${sr}^{-1}$          \\
					Radiant Flux      & $\Phi$                                       & $W$                  \\
					Radiant Intensity & $J = d \Phi / d \omega$                      & $W {sr}^{-1}$        \\
					Irradiance        & $E = d \Phi / d A$                           & $W m^{-2}$           \\
					Radiance          & $L = d^2 \Phi / (dA \cos{\theta_r} d\omega)$ & $W m^{-2} {sr}^{-1}$ \\ \bottomrule
				\end{tabular}
			\end{subfigure}
		} \\[2cm]
		\begin{subfigure}[b]{0.48\textwidth}
			\centering
			\includegraphics[width=\textwidth]{background/images/irradiance}
			\caption*{Irradiance}
		\end{subfigure} &
		\begin{subfigure}[b]{0.48\textwidth}
			\centering
			\includegraphics[width=\textwidth]{background/images/radiance}
			\caption*{Radiance}
		\end{subfigure}
	\end{tabular}
	\caption{Illustration of common radiometric terms, focusing on the surface
	         irradiance and radiance. ${sr}^{-1}$ denotes steradians, the
	         Standard International unit of solid angular measure and
	         $W$ denotes watts.}
\label{fig:bg_sfs_rad_irrad}
\end{figure}
%%%%%%%%%%%%%%%%%%%%%%%%%%%%%%%%%%%%%%%%
When discussing image formation methods a number of assumptions are commonly
made in order to ensure tractability of of the rendering physics. Firstly,
unless explicitly mentioned, we assume an orthographic camera projection. This
is a reasonable assumption for most facial images as faces tend to the be
the focus of an image and thus photographs are commonly taken close enough
to the face that perspective effects are minimal. We also only consider
reflection functions that can be expressed as a Bidirectional
Reflectance-Distribution Function (BRDF). A BRDF is a convenient construct
that allows the expression of how bright a surface will appear from a given
view point direction when illuminated from another direction. More formally, it
is the ratio of the reflected radiance in the viewing direction to the
irradiance, in the direction toward the light source.
See \cref{fig:bg_sfs_rad_irrad} for an illustration of the radiance and
irradiance as well as a table of useful radiometric terms. BRDFs are functions
of local illumination and do not model global illumination effects such
as shadows or inter-reflections.
\textit{Radiant flux} is the power emitted from a light source, measured in
watts $(W)$.
The \textit{solid angle} subtended by a surface patch is defined as the
surface area of a unit sphere covered by the surface's projection onto the
sphere and is measured in steradians $(sr)$. In \cref{fig:bg_sfs_rad_irrad},
$r$ refers to the distance from the sphere's origin to the patch.
\textit{Radiant intensity} is the radiant flux per unit solid angle and is
measured in watts per steradian $(W {sr}^{-1})$.
The \textit{irradiance} is the amount of energy received by a given surface
patch, measured in watts per square meter $(W m^{-2})$.
The \textit{radiance} is the amount of energy emitted per unit foreshortened
surface area per unit solid angle, measured in watts per square meter
per steradian $(W m^{-2} {sr}^{-1})$. In \cref{fig:bg_sfs_rad_irrad},
the unit foreshortened area is given by $dA \cos{\theta_r}$.
It is important to note that radiance, unlike irradiance, is a directional
quantity. This implies that the viewing angle affects the amount of perceived
light from a given image area and for some reflectance functions that manifests
as specular style highlights. Finally, we rely on the fact that the image
irradiance captured by the camera sensor is directly proportional to the
scene radiance~\cite{horn1979calculating}. To simplify matters, when referring
to image irradiance we assume that the linear relationship between the scene
radiance and image irradiance is the identity
\ie~scene radiance = image irradiance. An illustration of the relationship
between scene radiance and image irradiance is given in
\cref{fig:bg_sfs_scene_to_intensity}.

Given the previous definitions, we can now formally define the general equation
for a BRDF
%%%%%%%%%%%%%%%%%%%
\begin{align}\label{eg:bg_sfs_general_brdf}
	f(\theta_i,\phi_i;\theta_r,\phi_r) &= \frac{L(\theta_r,\phi_r)}{E(\theta_i,\phi_i)} \\
	L(\theta_r,\phi_r)                 &= E(\theta_i,\phi_i) f(\theta_i,\phi_i;\theta_r,\phi_r) \\
	L(\theta_r,\phi_r)                 &= \int_{2\pi} I_{\operatorname{src}}(\theta_i,\phi_i) f(\theta_i,\phi_i;\theta_r,\phi_r) \; \cos{\theta_i} \; d\omega_i
\end{align}
%%%%%%%%%%%%%%%%%%%
where $I_{\operatorname{src}}(\theta_i,\phi_i)$ is the intensity of output
from a given light source. In fact, assuming isotropy, the BRDF is
only a function of 3 parameters, $f(\theta_i,\theta_r, \Delta \phi)$ where
$\Delta \phi = \phi_i - \phi_r$. An isotropic BRDF is invariant to rotations
around the direction of the normal. We also assume Helmholtz reciprocity, which
states that if the light source and viewing direction are swapped the BRDF
remains unchanged
\ie~$f(\theta_i,\phi_i;\theta_r,\phi_r) = f(\theta_r,\phi_r;\theta_i,\phi_i)$.
An illustration of the general BRDF is given in \cref{fig:bg_sfs_brdf_example}.
%%%%%%%%%%%%%%%%%%%%%%%%%%%%%%%%%%%%%%%%
\begin{figure}[t]
	\centering
	\begin{subfigure}[b]{0.45\textwidth}
		\centering
		\includegraphics[width=\textwidth]{background/images/general_brdf}
		\caption*{General BRDF}
	\end{subfigure}
	\begin{subfigure}[b]{0.45\textwidth}
		\centering
		\includegraphics[width=\textwidth]{background/images/lambertian_brdf}
		\caption*{Lambertian BRDF}
	\end{subfigure}
	\caption{The left image shows the general radiance equation for a generic
	         BRDF that is parametrized by the irradiance $(\theta_i, \phi_i)$
	         and radiance $(\theta_r, \theta_r)$ reflecting from the surface.
	         The right image shows the Lambertian BRDF assuming unit light
	         intensity. Note that the Lambertian BRDF is independent of
	         the viewing direction.}
\label{fig:bg_sfs_brdf_example}
\end{figure}
%%%%%%%%%%%%%%%%%%%%%%%%%%%%%%%%%%%%%%%%

The most commonly assumed BRDF is that of the Lambertian BRDF.\@ A Lambertian
BRDF models only the diffuse component of radiance and is physically accurate
for materials whose reflectance are dominated by scattering effects. For
example, a commonly cited highly Lambertian object is chalk. In more
detail, the Lambertian BRDF modifies \cref{eg:bg_sfs_general_brdf} such that
$f(\theta_i,\theta_r, \Delta \phi) = \rho_{\smallsub{d}} / \pi$ and therefore
the image irradiance is defined as
%%%%%%%%%%%%%%%%%%%
\begin{align}\label{eg:bg_sfs_lambertian_brdf}
	L_{\operatorname{lambert}} &= I_{\operatorname{src}} \frac{\rho_{\smallsub{d}}}{\pi} \cos{\theta_i} \\
	L_{\operatorname{lambert}} &= I_{\operatorname{src}} \frac{\rho_{\smallsub{d}}}{\pi} \bb{n}^T \bb{s}
\end{align}
%%%%%%%%%%%%%%%%%%%
where $\bb{n}$ is the unit normal of the surface patch, $\bb{s}$ is the unit
light vector and $\rho_{\smallsub{d}}$ is the diffuse albedo scaled by $1/\pi$
to ensure $\rho_{\smallsub{d}} \in [0, 1]$. The Lambertian BRDF is illustrated
in \cref{fig:bg_sfs_brdf_example}. Note that for the Lambertian BRDF, the image
irradiance does not depend on the viewing direction and thus
$L_{\operatorname{lambert}}$ is not a function of $(\theta_r,\phi_r)$.
Physically, this implies that Lambertian surfaces do not exhibit specular
effects and appear evenly lit from all viewing angles, scaled by the cosine
of the angle between the light and normal vectors. Commonly,
\cref{eg:bg_sfs_lambertian_brdf} is simplified further as it is assumed that the
radiant intensity is normalised and thus the most commonly cited form of the
Lambertian BRDF is
%%%%%%%%%%%%%%%%%%%
\begin{align}\label{eg:bg_sfs_lambertian_simple}
	L_{\operatorname{lambert}} = \rho_{\smallsub{d}} \bb{n}^T \bb{s}
\end{align}
%%%%%%%%%%%%%%%%%%%
Finally, \cref{eg:bg_sfs_lambertian_simple} may generate outputs that are
not physically realisable. For example, a patch lit from the vector opposite
its normal would produce negative intensity. Therefore, the correct physical
form of the Lambertian BRDF is
$L_{\operatorname{lambert}} = \rho_{\smallsub{d}} \max(\bb{n}^T \bb{s}, 0)$.
%%%%%%%%%%%%%%%%%%%%%%%%%%%%%%%%%%%%%%%%
\begin{figure}[t]
	\centering
	\includegraphics[width=\textwidth]{background/images/scene_radiance_to_pixel}
	\caption{Illustration of the formation of a pixel intensity in image space
	         from scene radiance. Scene radiance is linearly proportional
	         to image irradiance~\cite{horn1979calculating} and image irradiance
	         is non-linearly mapped to a pixel intensity via the camera
	         response function, $f$.}
\label{fig:bg_sfs_scene_to_intensity}
\end{figure}
%%%%%%%%%%%%%%%%%%%%%%%%%%%%%%%%%%%%%%%%
For completeness, is it worthwhile noting that the image irradiance, formed
through the interaction of the scene radiance and the optic lens, is not
the final measured pixel intensity value commonly utilized on Computer Vision.
The final measured pixel intensity is a function of the image irradiance
and a non-linear function commonly referred to as the
\textit{camera response function}. The camera response function models a number
of effects such as detector sensitivity, vignetting, lens falloff and the
camera electronics. In fact, many manufacturers intentionally model
the camera response function to simulate the responses of other media
such as film~\cite{grossberg2003space}. The calibration function is consistent
over the entire image area and is invertible. Therefore, the camera response
function modifies the image irradiance as follows: $I = f(L)$ where $I$ denotes
a singe pixel in the image, $f$ denotes the camera response function and $L$
denotes the image irradiance as was being discussed above.
\cref{fig:bg_sfs_scene_to_intensity} gives an illustration of this process. Note
that the transformation from scene radiance to image irradiance is
linear~\cite{horn1979calculating} and was previously ignored when discussing
radiometric terms. Camera response calibration requires acquiring controlled
images of a MacBeth board or other colour chart, or the use of pre-calculated
response model~\cite{grossberg2003space}. Therefore, unless explicitly
mentioned, we make the strong assumption that the camera response function
is the identity function and thus pixels in any given image directly
represent the scene radiance.

In the following section we review relevant facial SfS algorithms. Unless
explicitly mentioned, the methods below assume that faces exhibit ideal
Lambertian reflectance given by \cref{eg:bg_sfs_lambertian_simple},
which has been shown to be a reasonable approximation
of facial reflectance~\cite{Sirovich:1987te,georghiades2001fromfew,%
Basri:2003ie,turk1991eigenfaces,Hallinan:1994dz,ramamoorthi2002analytic,%
ramamoorthi2001relationship,shashua1997photometric,moses1993face}. For a more
thorough treatment of the generic SfS literature we suggest the surveys
of \citet{zhang1999shape} and \citet{durou2008numerical}.
%%%%%%%%%%%%%%%%%%%%%%%%%%%%%%%%%%%%%%%%%%%%%%%%%%%%%%%%%%%%%%%%%%%%%%%%%%%%%%%%
\subsection{SfS Facial Surface Recovery}
%%%%%%%%%%%%%%%%%%%%%%%%%%%%%%%%%%%%%%%%%%%%%%%%%%%%%%%%%%%%%%%%%%%%%%%%%%%%%%%%
The current state-of-the-art in SfS is the generic method of
\citet{barron2015shape}, called
Shape, Illumination and Reflectance from Shading (SIRFS). The SIRFS method is
described as a generalisation of the modern ``intrinsic image'' algorithm in
which shading is parametrized in terms shape and illumination.
\citet{barron2015shape} seek to recover the most likely explanation, in a
statistical sense, by imposing a set of very general priors around smoothness
and colour composition on the input image. Although this is a state-of-the-art
method, it does not perform well for shape recover of faces.
\citet{li2014intrinsic} extend SIRFS by enforcing facial specific priors. They
use the physically based skin BRDF proposed by \citet{weyrich2006analysis} and
place priors on the skin reflectance parameters using the data
from~\cite{weyrich2006analysis}. They also impose a geometric prior on the face
shape by providing an initial estimate of the 3D face shape
using~\cite{Yang:2011gj}. \citet{li2014intrinsic} show superior results to the
original SIRFS, particular for surface normal recovery.
\citet{atick1996statistical} propose an analysis-by-synthesis method
for minimizing the reconstruction error between a statistical model of facial 
surface based on depth images rendered using a Lambertian BRDF and the input 
image. Although this is termed as SfS, we classify it as analysis-by-synthesis
and discuss it in further detail in \cref{ch:bg_3dmm}.
\citet{dovgard2004statistical} combine the symmetric SfS method of 
\citet{yilmaz2002estimation} with the statistical model of 
\citet{atick1996statistical} in order to resolve an ambiguity present in
the symmetric formulation of~\cite{yilmaz2002estimation}.
%TODO: Smiths IJCV about facial shading (smith2010estimating)
\citet{biswas2009robust} concentrate on robust albedo estimation, though they
demonstrate the accuracy of their albedo estimation by performing SfS
using the method of \citet{ping1994shape}. In particular, they utilize a mean
shape as the initial shape estimate and thus produce a robust estimate
of the albedo using a Linear Minimum Mean Square Error Estimator (LMMSE). They
proceed to normalise the image using the albedo estimate and then transform
the normalised image to appear lit by the initial illumination direction
estimate. Finally, SfS is performed on the transformed image, which they
showed was much more accurate than performing SfS on the initial input image.
% TODO: Template deformation methods like Kemelmacher
% TODO: Model based methods like Smith/Minsik
%%%%%%%%%%%%%%%%%%%%%%%%%%%%%%%%%%%%%%%%%%%%%%%%%%%%%%%%%%%%%%%%%%%%%%%%%%%%%%%%

%%%%%%%%%%%%%%%%%%%%%%%%%%%%%%%%%%%%%%%%%%%%%%%%%%%%%%%%%%%%%%%%%%%%%%%%%%%%%%%%
\section{Calibrated and Uncalibrated Photometric Stereo}\label{ch:bg_ps}
%%%%%%%%%%%%%%%%%%%%%%%%%%%%%%%%%%%%%%%%%%%%%%%%%%%%%%%%%%%%%%%%%%%%%%%%%%%%%%%%

%%%%%%%%%%%%%%%%%%%%%%%%%%%%%%%%%%%%%%%%%%%%%%%%%%%%%%%%%%%%%%%%%%%%%%%%%%%%%%%%

%%%%%%%%%%%%%%%%%%%%%%%%%%%%%%%%%%%%%%%%%%%%%%%%%%%%%%%%%%%%%%%%%%%%%%%%%%%%%%%%
\section{3D Morphable Models}\label{ch:bg_3dmm}
%%%%%%%%%%%%%%%%%%%%%%%%%%%%%%%%%%%%%%%%%%%%%%%%%%%%%%%%%%%%%%%%%%%%%%%%%%%%%%%%
\cite{atick1996statistical} proposed one of the earliest analysis-by-synthesis
methods for facial shape recovery. They compute PCA on a set of 
Cyberware~\cite{cyberware} scanned heads they are parametrized using
cylindrical coordinates. They then pose shape recovery as the problem
of recovering the PCA coefficients for a given input image by minimizing
the least squares error between the basis rendered orthographically
using a Lambertian BRDF (with assumed known light and uniform albedo) 
and the input image. This work was inspired by the Eigenfaces work of 
\citet{Sirovich:1987te} but utilized 3D data rather than images. To solve
the least squares a linearisation is performed via a Taylor series expansion
and conjugate gradient descent is applied.
%%%%%%%%%%%%%%%%%%%%%%%%%%%%%%%%%%%%%%%%%%%%%%%%%%%%%%%%%%%%%%%%%%%%%%%%%%%%%%%%

%%%%%%%%%%%%%%%%%%%%%%%%%%%%%%%%%%%%%%%%%%%%%%%%%%%%%%%%%%%%%%%%%%%%%%%%%%%%%%%%
\section{Shape Transfer and Model-based Methods}\label{ch:bg_model_based}
%%%%%%%%%%%%%%%%%%%%%%%%%%%%%%%%%%%%%%%%%%%%%%%%%%%%%%%%%%%%%%%%%%%%%%%%%%%%%%%%
%TODO: Check Castelan's work
\citet{minsik2013robust} propose a direct mapping from 3D facial shape to an
image of a face under general unknown lighting. The depth and texture pairs from
FRGC~\cite{phillips2005overview} are used to generate a discrete set of
renderings of faces with cast shadows. N-Mode
SVD~\cite{vasilescu2003multilinear} followed by orthogonalisation is applied to
both the rendered textures and the depths in order to perform dimensionality
reduction. This provides a set of transformed images and a subspace is recovered
using a generalised eigenvalue method to further reduce the dimensionality.
Finally, Canonical Correlations Analysis (CCA) is applied between the
dimensionality reduced depth images and a hyperspherical representation of the
image subspace. At test time, the image is transformed using the subspace and
the depth is recovered using the CCA mapping. This process is very fast for
recovery of an input image and handles arbitrary lighting conditions, but
only applies to frontal faces and does not guarantee photometric consistency
of the recovered shape.
\citet{minsik2014realtime} extend the work of \citet{minsik2011fast} with a
novel optimisation procedure that is extremely efficient. Similar
to~\cite{minsik2011fast}, a tensor formulation is proposed. However, the
bilinear model of~\cite{minsik2011fast} is relaxed by enforcing that the
illumination and identity modes can be described as a set of rank-one matrices.
Furthermore, in order to incorporate cast shadows, they render the input meshes
under a variety of illumination conditions including cast shadows and perform
the tensor decomposition on the rendered data rather than the spherical
harmonics, similar to \citet{minsik2013robust}. At test time, the input image
is projected against the bilinear illumination basis and the best rank-one
structure of the lighting and identity coefficients are recovered. The images
are aligned using affine alignment of the eye centres and depth is recovered
from the learnt tensor depth model and the recovered rank-one identity
coefficients.
%%%%%%%%%%%%%%%%%%%%%%%%%%%%%%%%%%%%%%%%%%%%%%%%%%%%%%%%%%%%%%%%%%%%%%%%%%%%%%%%

%%%%%%%%%%%%%%%%%%%%%%%%%%%%%%%%%%%%%%%%%%%%%%%%%%%%%%%%%%%%%%%%%%%%%%%%%%%%%%%%
\section{Structure-from-Motion}\label{sec:bg_sfm}
%%%%%%%%%%%%%%%%%%%%%%%%%%%%%%%%%%%%%%%%%%%%%%%%%%%%%%%%%%%%%%%%%%%%%%%%%%%%%%%%
%TODO: Check references 75-86 of
%      A SURVEY ON 3D MODELING OF HUMAN FACES FOR FACE RECOGNITION
Structure-from-Motion (SfM)~\cite{ullman1979interpretation,tomasi1992shape,hartley2003multiple}
is a method for recovering the 3D structure of a set of 2D points that
correspond to salient areas of an image. These 2D coordinates are expected to be
in correspondence across multiple images and thus are modelled as representing
the projection of a 3D structure onto an image plane observed from multiple
camera positions. The positions of these cameras and the 3D structure is assumed
to be unknown. Rigid SfM has been successfully employed to recover large outdoor
scenes such as the Colosseum in Rome~\cite{agarwal2009building}. Although large
scenes which may contain very sparse correspondences typically optimise rigid
SfM via ``bundle-adjustment''~\cite{triggs1999bundle} which seeks to
minimise the re-projection error between the predicted camera poses and
the known 2D points. Since facial sequences will contain many fewer
known 2D locations viewed over likely fewer camera positions, we focus on
the matrix factorisation formation of SfM~\cite{tomasi1992shape}. Given
a set of $p$ 2D points for $n$ images, where the 2D points are in
correspondence across the images, we construct
the measurement matrix $\bb{W} = [\bb{w}_1, \ldots, \bb{w}_n] \in \R^{2n \times p}$
where each column represents the 2D positions of a single point from each input frame,
$\bb{w}_i = {[x_{1,1}, y_{1,1}, \ldots, x_{i,n}, y_{i,n}]}^T \in \R^{2n \times 1}$
are the $n$ locations of the $i$th 2D point. The 2D points
in each frame are assumed to be formed via orthographic projection,
$\bb{w}_{i,j} = \bb{R}_j \bb{s}_i$ where $\bb{R}_j \in \R^{2 \times 3}$ are the
first two rows of a the $j$th frames' camera rotation matrix and
$\bb{s}_i = [x, y, z] \in \R^{3 \times 1}$ are
the true 3D coordinates of the $i$th 2D point. The entire 3D shape matrix
is then given by $\bb{S} = [\bb{s}_1, \ldots, \bb{s}_p] \in \R^{3 \times p}$.
Given that the camera is orthographic, the measurement matrix is assumed to be
zero-mean across the columns and thus any translation in the camera plane is
accounted for. The matrix factorisation formulation of rigid SfM is thus simply
defined
%%%%%%%%%%%%%%%%%%%
\begin{equation}\label{eg:bg_sfm_rigid_objective}
	\bb{W} = \bb{M} \bb{S}
\end{equation}
%%%%%%%%%%%%%%%%%%%
where $\bb{M} = {[{\bb{R}_1}^T, \ldots, {\bb{R}_n}^T]}^T$. The solution to
\cref{eg:bg_sfm_rigid_objective} is found via a 
Singular Value Decomposition (SVD) under the constraint that 
$\rank{\bb{W}} \leq 3$ assuming noise free measurements. 
The SVD introduces a $3 \times 3$ gauge ambiguity which
can be disambiguated by imposing metric constraints on the fact that the
rotation matrices should be orthogonal~\cite{tomasi1992shape}. This solves the
gauge ambiguity up to a global rotation which can be resolved by assuming that
the world transformation matrix is equivalent to the rotation of the first
camera.

Unfortunately, images of faces rarely contain the same static
face from many observation points. Although possible, constraining the
subject to be perfectly still is a very strong restriction.
If this level of control is possible then it is highly likely that multi-
view stereo would be preferable and the camera positions could also be
controlled. What is more likely is that either multiple faces of differing
individuals (unconstrained image collection) or a video of a single individual
in motion (sequences of images) are captured. In this case, the movement
of the true 3D surface is no longer rigid and is described as non-rigid. A
video of an expressive face would certainly classify as deformable and thus
non-rigid SfM would be required. Non-rigid SfM differs from rigid SfM in
that the shape matrix is not assumed to be fix and thus the 3D shape can
vary over every frame. Obviously this is highly under-constrained and so the
most general assumption made is that the shape matrix is in fact formed
from a linear shape basis with $k$ basis shapes. As first proposed by 
\citet{bregler2000recovering}, this is defined by
%%%%%%%%%%%%%%%%%%%
\begin{equation}\label{eg:bg_sfm_non_rigid_objective}
	\bb{W} = \bb{D} (\bb{C} \otimes \bb{I}_3) \bb{B}
\end{equation}
%%%%%%%%%%%%%%%%%%%
where $\bb{D} \in \R^{2n \times 3}$ is the block diagonal camera matrix
formed by $\sum_{j=1}^n e_j {e_j}^T \otimes \bb{R}_j$ using the canonical
basis $\mathbb{F}^n$,
$\bb{C} = [{\bb{c}_1}^T, \ldots, {\bb{c}_k}^T] \in \R^{n \times k}$ where
$\bb{c}_q = {[c_{1, q}, \ldots, c_{n, q}]}^T \in \R^{n \times 1}$ are the 
coefficients for the $q$th shape basis for every frame and $\bb{I}_3$ is a
$3 \times 3$ identity matrix. The shape matrix from
\cref{eg:bg_sfm_rigid_objective} is replaced by the linear basis, 
$\bb{S} = (\bb{C} \otimes \bb{I}_3) \bb{B}$, where
$\bb{B} = {[{\bb{\hat{B}}_1}^T, \ldots, {\bb{\hat{B}}_k}^T]}^T \in \R^{3k \times p}$
and each matrix
$\bb{\hat{B}}_q = [\bb{\hat{b}}_{1,q}, \ldots, \bb{\hat{b}}_{p,q}] \in \R^{3 \times p}$
where $\bb{\hat{b}}_{i,q} = [x_{i,q}, y_{i,q}, z_{i,q}] \in \R^{3 \times 1}$ and thus
$\bb{\hat{B}}_q$ is the $q$th shape basis. Note that the solution to
\cref{eg:bg_sfm_non_rigid_objective} is found via an SVD by noting that
$\bb{M} = \bb{D} (\bb{C} \otimes \bb{I}_3)$ and thus the solution is identical
to \cref{eg:bg_sfm_rigid_objective}, $\bb{W} = \bb{M} \bb{B}$, but with the 
constraint that the $\rank{\bb{W}} \leq 3k$. Again, this yields the solution up 
to a $3k \times 3k$ ambiguity and the matrices $\bb{C}$ and $\bb{D}$ are
recovered by forming a Euclidean upgrade 
step~\cite{akhter2009defense,xiao2006closed,brand2005direct} or alternatively
$\bb{M}$ and $\bb{S}$ may be recovered directly~\cite{dai2014simple}.
Many additional priors have been proposed to improve the recovery of the 3D
shape including trajectory based methods~\cite{akhter2011trajectory} and
smoothness priors on the motion between the frames~\cite{bartoli2008coarse}.
%%%%%%%%%%%%%%%%%%%%%%%%%%%%%%%%%%%%%%%%%%%%%%%%%%%%%%%%%%%%%%%%%%%%%%%%%%%%%%%%

%%%%%%%%%%%%%%%%%%%%%%%%%%%%%%%%%%%%%%%%%%%%%%%%%%%%%%%%%%%%%%%%%%%%%%%%%%%%%%%%
\stopcontents[chapters]
%%%%%%%%%%%%%%%%%%%%%%%%%%%%%%%%%%%%%%%%%%%%%%%%%%%%%%%%%%%%%%%%%%%%%%%%%%%%%%%%