%%%%%%%%%%%%%%%%%%%%%%%%%%%%%%%%%%%%%%%%%%%%%%%%%%%%%%%%%%%%%%%%%%%%%%%%%%%%%%%%
\section{Model-based and Alignment Methods}\label{sec:bg_model_based}
%%%%%%%%%%%%%%%%%%%%%%%%%%%%%%%%%%%%%%%%%%%%%%%%%%%%%%%%%%%%%%%%%%%%%%%%%%%%%%%%
In this section we considered methods that seek to recover 3D by the use
of explicit models of faces. Given some model of the 3D structure of a human
face, it is possible to attempt to find some feature of the input image
that is best matched to some aspect of the model. We separate the literature
into two distinct areas: those that consider reconstruction as a problem of
alignment and those that seek to recover structure from the model assuming
alignment is solved. In contrast to other seemingly similar techniques such as
3D Morphable Models~\cite{volker1999morphable} that use 3D statistical models
of faces the methods considered in this section do not necessarily enforce any
specific structure on the formation of the image.
%%%%%%%%%%%%%%%%%%%%%%%%%%%%%%%%%%%%%%%%%%%%%%%%%%%%%%%%%%%%%%%%%%%%%%%%%%%%%%%%
\subsection{Model-based Methods}\label{subsec:bg_model_based_model}
%%%%%%%%%%%%%%%%%%%%%%%%%%%%%%%%%%%%%%%%%%%%%%%%%%%%%%%%%%%%%%%%%%%%%%%%%%%%%%%%
%TODO: Check Castelan's work
\citet{minsik2013robust} propose a direct mapping from 3D facial shape to an
image of a face under general unknown lighting. The depth and texture pairs from
FRGC~\cite{phillips2005overview} are used to generate a discrete set of
renderings of faces with cast shadows. N-Mode
SVD~\cite{vasilescu2003multilinear} followed by orthogonalisation is applied to
both the rendered textures and the depths in order to perform dimensionality
reduction. This provides a set of transformed images and a subspace is recovered
using a generalised eigenvalue method to further reduce the dimensionality.
Finally, Canonical Correlations Analysis (CCA) is applied between the
dimensionality reduced depth images and a hyperspherical representation of the
image subspace. At test time, the image is transformed using the subspace and
the depth is recovered using the CCA mapping. This process is very fast for
recovery of an input image and handles arbitrary lighting conditions, but
only applies to frontal faces and does not guarantee photometric consistency
of the recovered shape.
\citet{minsik2014realtime} extend the work of \citet{minsik2011fast} with a
novel optimisation procedure that is extremely efficient. Similar
to~\cite{minsik2011fast}, a tensor formulation is proposed. However, the
bilinear model of~\cite{minsik2011fast} is relaxed by enforcing that the
illumination and identity modes can be described as a set of rank-one matrices.
Furthermore, in order to incorporate cast shadows, they render the input meshes
under a variety of illumination conditions including cast shadows and perform
the tensor decomposition on the rendered data rather than the spherical
harmonics, similar to \citet{minsik2013robust}. At test time, the input image
is projected against the bilinear illumination basis and the best rank-one
structure of the lighting and identity coefficients are recovered. The images
are aligned using affine alignment of the eye centres and depth is recovered
from the learnt tensor depth model and the recovered rank-one identity
coefficients.
%%%%%%%%%%%%%%%%%%%%%%%%%%%%%%%%%%%%%%%%%%%%%%%%%%%%%%%%%%%%%%%%%%%%%%%%%%%%%%%%
\subsection{Alignment Methods}\label{subsec:bg_model_based_alignment}
%%%%%%%%%%%%%%%%%%%%%%%%%%%%%%%%%%%%%%%%%%%%%%%%%%%%%%%%%%%%%%%%%%%%%%%%%%%%%%%%

%%%%%%%%%%%%%%%%%%%%%%%%%%%%%%%%%%%%%%%%%%%%%%%%%%%%%%%%%%%%%%%%%%%%%%%%%%%%%%%%
