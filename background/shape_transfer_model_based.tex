%%%%%%%%%%%%%%%%%%%%%%%%%%%%%%%%%%%%%%%%%%%%%%%%%%%%%%%%%%%%%%%%%%%%%%%%%%%%%%%%
\section{Shape Transfer and Model-based Methods}\label{ch:bg_model_based}
%%%%%%%%%%%%%%%%%%%%%%%%%%%%%%%%%%%%%%%%%%%%%%%%%%%%%%%%%%%%%%%%%%%%%%%%%%%%%%%%
\citet{minsik2013robust} propose a direct mapping from 3D facial shape to an
image of a face under general unknown lighting. The depth and texture pairs from
FRGC~\cite{phillips2005overview} are used to generate a discrete set of
renderings of faces with cast shadows. N-Mode
SVD~\cite{vasilescu2003multilinear} followed by orthogonalisation is applied to
both the rendered textures and the depths in order to perform dimensionality
reduction. This provides a set of transformed images and a subspace is recovered
using a generalised eigenvalue method to further reduce the dimensionality.
Finally, Canonical Correlations Analysis (CCA) is applied between the
dimensionality reduced depth images and a hyperspherical representation of the
image subspace. At test time, the image is transformed using the subspace and
the depth is recovered using the CCA mapping. This process is very fast for
recovery of an input image and handles arbitrary lighting conditions, but
only applies to frontal faces and does not guarantee photometric consistency
of the recovered shape.
%%%%%%%%%%%%%%%%%%%%%%%%%%%%%%%%%%%%%%%%%%%%%%%%%%%%%%%%%%%%%%%%%%%%%%%%%%%%%%%%
