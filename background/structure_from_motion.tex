%%%%%%%%%%%%%%%%%%%%%%%%%%%%%%%%%%%%%%%%%%%%%%%%%%%%%%%%%%%%%%%%%%%%%%%%%%%%%%%%
\section{Structure-from-Motion}\label{sec:bg_sfm}
%%%%%%%%%%%%%%%%%%%%%%%%%%%%%%%%%%%%%%%%%%%%%%%%%%%%%%%%%%%%%%%%%%%%%%%%%%%%%%%%
%TODO: Check references 75-86 of
%      A SURVEY ON 3D MODELING OF HUMAN FACES FOR FACE RECOGNITION
Structure-from-Motion (SfM)~\cite{ullman1979interpretation,tomasi1992shape,hartley2003multiple}
is a method for recovering the 3D structure of a set of 2D points that
correspond to salient areas of an image. These 2D coordinates are expected to be
in correspondence across multiple images and thus are modelled as representing
the projection of a 3D structure onto an image plane observed from multiple
camera positions. The positions of these cameras and the 3D structure is assumed
to be unknown. Rigid SfM has been successfully employed to recover large outdoor
scenes such as the Colosseum in Rome~\cite{agarwal2009building}. Although large
scenes which may contain very sparse correspondences typically optimise rigid
SfM via ``bundle-adjustment''~\cite{triggs1999bundle} which seeks to
minimise the re-projection error between the predicted camera poses and
the known 2D points. Since facial sequences will contain many fewer
known 2D locations viewed over likely fewer camera positions, we focus on
the matrix factorisation formation of SfM~\cite{tomasi1992shape}. Given
a set of $p$ 2D points for $n$ images, where the 2D points are in
correspondence across the images, we construct
the measurement matrix $\bb{W} = [\bb{w}_1, \ldots, \bb{w}_n] \in \R^{2p \times n}$
where each column represents the measurements from each input frame,
$\bb{w} = {[x_1, y_1, \ldots, x_p, y_p]}^T \in \R^{2p \times 1}$. The 2D points
in each frame are assumed to be formed via orthographic projection,
$\bb{w} = \bb{R} \bb{S}$ where $\bb{R} \in \R^{2 \times 3}$ are the first two
rows of a camera rotation matrix and
$\bb{S} = [\bb{s}_1, \ldots, \bb{s}_n] \in \R^{3 \times n}$ where
$\bb{s} = {[x_1, y_1, z_1]}^T \in \R^{3 \times 1}$ are
the true 3D coordinates of a given 2D point. Given the parametrisation of motion
as an orthographic projection, the measurement matrix is assumed to zero-mean
across the columns and thus any translation in the camera plane is accounted
for. The matrix factorisation formulation of rigid SfM is thus simply defined
%%%%%%%%%%%%%%%%%%%
\begin{equation}\label{eg:bg_sfm_rigid_objective}
	\bb{W} = \bb{D} \bb{S}
\end{equation}
%%%%%%%%%%%%%%%%%%%
where $\bb{D} \in \R^{2n \times 3}$ is the block diagonal camera matrix
formed by $\sum_{i=1}^n e_i {e_i}^T \otimes \bb{R}_i$ using the canonical
basis $\mathbb{F}^n$. However, images of faces rarely contain the same static
face from many observation points. Although this is possible, constraining the
subject to be perfectly still is a very strong restriction on the 3D recovery
conditions. If this level of control is possible it is highly likely that multi-
view stereo would be preferable and the camera positions could also be
controlled. What is more likely is that either multiple faces of differing
individuals (unconstrained image collection) or a video of a single individual
in motion (sequences of images) would be captured.
%%%%%%%%%%%%%%%%%%%%%%%%%%%%%%%%%%%%%%%%%%%%%%%%%%%%%%%%%%%%%%%%%%%%%%%%%%%%%%%%
