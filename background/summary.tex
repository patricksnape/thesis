%%%%%%%%%%%%%%%%%%%%%%%%%%%%%%%%%%%%%%%%%%%%%%%%%%%%%%%%%%%%%%%%%%%%%%%%%%%%%%%%
\section{Summary}\label{sec:bg_summary}
%%%%%%%%%%%%%%%%%%%%%%%%%%%%%%%%%%%%%%%%%%%%%%%%%%%%%%%%%%%%%%%%%%%%%%%%%%%%%%%%
In this background section we have outlined both the history and state-of-the-art
methods for recovering dense facial surfaces from images and videos. Although
this is a widely studied area with many methods providing impressive results,
it is still yet to be considered a solved problem. In particular, one of the
strongest assumptions made in most high quality shape recovery methods is
that of a detailed understanding of the construction of the scene. Whether
this is in the form of the illumination conditions, surface reflectance
properties or in prior knoweldge of correspondances between multiple images,
the majority of state-of-the-art methods are unlikely to be applicable to
``in-the-wild'' images. At the moment, the most commonly capture data
is that of colour camera photographs under uncontrolled conditons with
little knowledge of the scene construction. Therefore, the introduction
of priors such as parametric models of the scenes surfaces or illumination
are generally necessary to perform surface recovery. However, recent
advances in sensor technology may soon provide multi-modal data inputs for
consumers that will enable more wide-scale scene geometry understanding
without the need for such restrictive assumptions. Examples such as the
introduction of the Microsoft Kinect~\cite{zhang2012microsoft} have shown
that consumer grade electronics for multi-modal imaging can greatly enhance
the ability of modern systems to understand commonly observed scenes such as
dwelling interiors. Despite this, the state-of-the-art methods are lacking in
their ability to be able to be applied to commonly observed data such as videos
of humans interacting directly with a camera.

Therefore, we believe that the most challenging scenario,
recovering shape from ``in-the-wild'' facial data, is still
a very complex issue within Computer Vision. For this reason, we work to extend
this area in the following chapters in order to shed further light on the
applicability of both shading and alignment based methods for recovering
dense 3D facial shape.
