%%%%%%%%%%%%%%%%%%%%%%%%%%%%%%%%%%%%%%%%%%%%%%%%%%%%%%%%%%%%%%%%%%%%%%%%%%%%%%%%
\section{Calibrated and Uncalibrated Photometric Stereo}\label{sec:bg_ps}
%%%%%%%%%%%%%%%%%%%%%%%%%%%%%%%%%%%%%%%%%%%%%%%%%%%%%%%%%%%%%%%%%%%%%%%%%%%%%%%%
The original Photometric Stereo (PS)~\cite{woodham1980photometric} problem
is a relaxation of Shape-from-Shading (SfS) in which multiple images under
known, varying illumination are required. The most common form of
PS~\cite{woodham1980photometric} assumes a Lambertian BRDF and time-multiplexed
illumination. Time-multiplexed illumination implies that the input images
are lit under novel illumination patterns and the images are acquired
sequentially with a time delay between each capture. Given the natural
ill-posedness of the Lambertian SfS problem (as shown by
\labelcref{eq:bg_sfs_least_squares_lambert}), PS solves the problem by requiring
3 or more images under known illumination. Given 3+ images,
\cref{eq:bg_sfs_least_squares_lambert} becomes stable and an albedo per pixel
may also be recovered. More precisely, a surface normal and diffuse albedo
may be recovered per pixel by solving a linear least squares problem:
%%%%%%%%%%%%%%%%%%%
\begin{equation}\label{eg:bg_ps_lambertian_ls}
	 \min_{\bb{\tilde{N}}} \lVert \bb{X} - \bb{\tilde{N}} \bb{S} \rVert_F
\end{equation}
%%%%%%%%%%%%%%%%%%%
following the notation from \cref{eq:bg_sfs_least_squares_lambert},
assuming $n$ input images of the same width and height ($d$ total pixels)
%%%%%%%%%%%%%%%%%%%
\begin{align*}
	\bb{X} &= [\bb{I}_1, \ldots, \bb{I}_n] \in \R^{d \times n}, \\
	\bb{S} &= {[\bb{s}_1, \ldots, \bb{s}_n]}^T \in \R^{3 \times n}, \\
	\bb{\tilde{N}} &= {[\rho_{\smallsub{d}}(x_1, y_1)\bb{n}(x_1, y_1), \ldots, \rho_{\smallsub{d}}(x_w, y_h)\bb{n}(x_w, y_h)]}^T \in \R^{d \times 3}
\end{align*}
%%%%%%%%%%%%%%%%%%%
where $\bb{\tilde{N}}$ is the normal matrix scaled by the diffuse
albedo. \cref{eg:bg_ps_lambertian_ls} is solved via a
pseudoinverse as $\bb{\tilde{N}} = \bb{X} \bb{S}^{\dagger}$ and the norm of each
row of $\bb{\tilde{N}}$ recovers the diffuse albedo,
${\rho_{\smallsub{d}}}^i = \norm{\bb{\tilde{n}}^i}$
where $i$ is an index into the image space and thus $i \in [1, d]$
is the $i$th pixel and
$\bb{\tilde{n}}^i = {\left(\bb{\tilde{N}}_{i\ast}\right)}^T \in \R^{3 \times 1}$.
Similarly, the unit normals are recovered by
$\bb{\tilde{n}}^i = \frac{\bb{\tilde{n}}^i}{\norm{\bb{\tilde{n}}^i}}$.
%%%%%%%%%%%%%%%%%%%%%%%%%%%%%%%%%%%%%%%%
\begin{figure}[t]
	\centering
	\includegraphics[height=2in]{background/images/photoface0}
	\includegraphics[height=2in]{background/images/photoface1}
	\includegraphics[height=2in]{background/images/photoface2}
	\includegraphics[height=2in]{background/images/photoface3}
	\caption{An example of Photometric Stereo imaging ``in-the-wild'' from
	         the Photoface database~\cite{zafeiriou2013photoface}.
	         Notice the cast-shadows (yellow dashed) and specular highlights
	         (cyan solid) present in the images.}
\label{fig:bg_ps_photoface}
\end{figure}
%%%%%%%%%%%%%%%%%%%%%%%%%%%%%%%%%%%%%%%%

Although PS resolves the ambiguity present in the SfS problem with respect
to the number of inputs, traditional Lambertian PS is still sensitive to
deviations from Lambertian effects in the input images. For faces, this
is particularly evident in the form of cast shadows. For a given pixel that
corresponds to a point on the facial surface, this pixel must have varying
intensity in each of the input images. Practically, this implies that the face
must be lit from a set of reasonably obtuse angles which are prone to causing
cast shadows. An example of both specular highlights and cast shadows that
naturally occur in facial PS imaging is given in \cref{fig:bg_ps_photoface}.
Another issue is that time-multiplexed illumination causes the input
images to lose alignment with one another. If the subject moves at all
during capturing then the input images are not aligned and this breaks
one of the primary assumptions of PS:\@ that each pixel has multiple intensities
of the same surface point with which to compute the normal.
Note that although the traditional form of Photometric Stereo is only unique
for 3 or more images without cast shadows, it is possible to obtain
solutions for as little as as
images~\cite{mecca2013uniqueness,mecca2016unifying}. However, performing PS
on less than 3 images generally requires more complex numerical solutions to the
PS problem or the use of other useful cues such as photomtric
ratios~\citet{smith2016height,queau2016unbiased,mecca2016unifying}. Photometric
ratios have the distinct advantage that they lead to linear equations in
depth and thus avoid the integration of surface normals which may be non-integrable
and thus need to be constrained to lie in an integrable space e.g. using methods
such as that of \citet{frankot1988method}. Generally, set of of partial differential
equations (PDEs) are constructed from the photometric ratios which can then
be solved using methods such as semi-Langrangian schemes to recover surface
depth~\cite{mecca2013uniqueness,mecca2016unifying}. However, the primary
challenge in photometric ratio methods is how to robustly choose pixel
pairs that conform strongly to the reflectance model assumptions. Although this
is typically assume to be a Lambertian reflectance model, more complex models
have also been considered for photometric ratio
methods~\cite{chandraker2013differential,mecca2015realistic}.
\citet{smith2016height} improve upon the robustness of the photometric ratio
methods by building a model based solution. They propose to use the results
of a more traditional PS method to serve as a ``model'' of the underlying surface
and then later estimates of the surface shape that deviate strongly from
this prior can be rejected as noise. These robustly chosen photometric
ratio samples are then using to solve a sparse linear least squares problem.
\cite{queau2016unbiased} also propose to use photometric ratios for solving
PS but instead extend the greyscale assumption of~\cite{smith2016height}
to perform PS on coloured images. It is also possible to solve directly
for depth using image formation assumptions by recovering the maximum
likelihood estimate of the depth~\cite{harrison2012maximum}.

An alternative
way to perform Photometric Stereo is
\textit{Multi-spectral Photometric Stereo} (MS-PS). Here MS-PS refers to
using lighting of varying wavelengths in order to simultaneously capture
multiple inputs per pixel. This is sometimes referred to as
Colour Photometric Stereo~\cite{petrov1987light,woodham1994gradient,%
kontsevich1994reconstruction,hernandez2007non},
but there is no hard requirement that the wavelengths are perfectly
interpretable as `colours'~\cite{fyffe2011single} and therefore we refer to this
as multi-spectral. This is also in contrast to other uses of the term multi-
spectral which may refer to the separation of differing reflectance behaviours
within an input image~\cite{nayar1997separation,mallick2005beyond,zickler2008color}.
In contrast to traditional time-multiplexed PS (TM-PS), multi-spectral PS
recovers multiple inputs by assuming that a colour camera CCD has sensors that
are sensitive to at least 3, traditionally red, green and blue (RGB), distinct
wavelengths. Thus, rather than multiple time multiplexed illuminations from
identical lights, MS-PS assumes multiple spectrally unique illuminations from
multiple directions which contribute differently to each sensor output
of the camera. More precisely, assuming $s$ different sensors within the camera
(typically $s = 3$, RGB), MS-PS is defined by:
%%%%%%%%%%%%%%%%%%%
\begin{equation}\label{eg:bg_ps_lambertian_ms_ps}
	 \min_{\bb{\tilde{N}}} \lVert \bb{C} - \bb{\tilde{N}} \bb{S} \bb{V}^T \rVert_F
\end{equation}
%%%%%%%%%%%%%%%%%%%
where $\bb{\tilde{N}}$ and $\bb{S}$ are as in \cref{eg:bg_ps_lambertian_ls}
and $\bb{C} = {[\bb{c}_1, \ldots, \bb{c}_d]}^T \in \R^{d \times s}$ and
$\bb{c} = {[c_1, \ldots, c_s]}^T \in \R^{s \times 1}$ are the intensities
per wavelength in the image. $\bb{V} \in \R^{s \times n}$ is the sensor
response matrix and the single entry $v_{ij}$:
%%%%%%%%%%%%%%%%%%%
\begin{equation}\label{eg:bg_ps_sensor_response}
	v_{ij} = \int E_j(\lambda) \alpha(\lambda) S_i(\lambda) \; d\lambda
\end{equation}
%%%%%%%%%%%%%%%%%%%
defines the sensor response for the $i$th sensor under the $j$th light for
a specific wavelength of light denoted $\lambda$. Thus $E_j(\lambda)$ is the
spectral distribution of light $j$, $S_i(\lambda)$ is the sensitivity
of sensor $i$ and $\alpha(\lambda)$ is the characteristic chromaticity for the
given material. Although MS-PS allows for simultaneously illumination of the
face, it requires careful calibration of the sensor response matrix. This
calibration matrix is unique to the illumination set up and face and thus
calibration must be performed separately for each individual
being captured.

As discussed in \cref{subsec:bg_capture}, PS has been used to collect both
facial reflectance and facial surface databases. Due to the explicit
calibration of lighting conditions and the fact that multiple images are
required, we largely regard PS as a dedicated capture methodology. Although
there have been efforts to perform PS on less constrained conditions, as in
the Photoface Database~\cite{zafeiriou2013photoface}, it still requires calibrated
lighting and specialised hardware. However, the PS problem can be relaxed by
considering the separation of lighting and surface components as a matrix
factorisation problem~\cite{basri2007photometric,hayakawa1994photometric}.
This is called \textit{Uncalibrated Photometric Stereo} (U-PS),
due to the lack of requirement for calibrated illumination.

\citet{kemelmacher2012collection} propose a method similar in spirit to
\citet{vetter1997bootstrapping} in order to improve optical flow alignment
between collection of facial images. In \citet{vetter1997bootstrapping} the
authors proposes to build a linear model from all currently aligned images in
order to build a target that is as close to the input image as possible. The
more similar the target and input images are, the better the optical flow
results will be. In \citet{kemelmacher2012collection}, they extend this
idea and further normalise the texture output by constraining the linear
model to a rank 4 subspace. This subspace is consistent with the spherical
harmonics subspace of faces and thus the output reference frame contains
a face closer to the mean but with illumination variation maintained. It is
assumed that all input images are approximately frontal and that there is
sufficient lighting variation in the input in order to allow the illumination
components to dominate the first 4 eigenvectors.
%%%%%%%%%%%%%%%%%%%%%%%%%%%%%%%%%%%%%%%%%%%%%%%%%%%%%%%%%%%%%%%%%%%%%%%%%%%%%%%%
