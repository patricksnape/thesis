%%%%%%%%%%%%%%%%%%%%%%%%%%%%%%%%%%%%%%%%%%%%%%%%%%%%%%%%%%%%%%%%%%%%%%%%%%%%%%%%
\section{Summary and Conclusion}\label{sec:face_flow_conclusion}
%%%%%%%%%%%%%%%%%%%%%%%%%%%%%%%%%%%%%%%%%%%%%%%%%%%%%%%%%%%%%%%%%%%%%%%%%%%%%%%%
In this chapter we investigated the problem of recovering dense correspondences
in 2D across a sequence images. These dense correspondences are essential
for using model based priors for 3D recovery directly from images. We argue that
the 2D correspondences are the most difficult component of this problem, 
particularly if the method of 3D recovery is assumed to involve the fitting
of a 3D statistical model to the 2D points. To this end, we propose a algorithm
for computing dense correspondences across facial sequences that is inspired by
optical flow methods. By assuming the prior of a facial sequence, it becomes
possible to construct a 2D deformation basis that not only constrains the space
of possible 2D deformations but also allows the use of an efficient Lucas-Kanade
style algorithm. Furthermore, we assert a prior on the coefficients
of the deformation basis that assumes that deformations across the sequence
are smooth in the space of the coefficients.

We proposed two methodologies for constructing this deformation basis: one
from the output of carefully restricted optical flow methods and one by
the rasterization of a 3D statistical model. However, our fitting algorithm,
which we coin ``Face Flow'', is independent of these models and we show that
both models work well on challenging sequences. In particular, the model built
from the 3D statistical model enables the recovery of 3D facial surfaces by 
leveraging the correspondence that exists between every vertex in the 3D model
and the dense deformation basis of Face Flow. Face Flow is also shown to be
competitive with state-of-the-art optical flow methods, particularly in
the presence of challenging illumination and occlusions.

Although Face Flow was shown to perform strongly for ``in-the-wild'' sequences,
it still treats the problem of 2D correspondence recovery and 3D shape recovery
separately.  This is similar to the current state-of-the-art of dense non-rigid
structure-from-motion~\cite{garg2013dense}. It would be interesting to
extend this work to incorporate a 3D projection model and remove the need
for the construction of the 2D deformation basis and subsequent 3D rigid pose
recovery. However, formulating an efficient fitting algorithm is more challenging
as the linearisation of the 3D motion model is far more complex than the linear
model used in this chapter.