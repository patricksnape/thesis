Computing optical flow in the presence of non-rigid deformations
is an important and challenging task. It plays a significant role in a wide variety of 
problems such as medical imaging, dense non-rigid 3D reconstruction, 
dense 3D mesh registration, motion segmentation, video re-texturing and super-resolution.
Broadly, optical flow methods describe procedures to relate pixels in
one image to pixels in another image of the same object. They establish a 
displacement field that can be thought of as a sampling, or warping, of the input 
image back onto the reference image. Traditionally, optical flow is applied on a pair 
of consecutive frames of a sequence, treating one of the frames as the template.
However, in terms of revealing the dynamics of a non-rigid scene, it is much more 
useful to estimate the optical flow between every frame of a long sequence and a 
common template. In this scenario, long-term dense 2D tracks across the sequence are established. 
We tackle the problem of multi-frame optical flow by focusing on a 
specific deformable object, the human face.

The most straight-forward way to estimate a multi-frame optical flow is to apply an 
algorithm that solves the traditional two-frame optical flow problem between every 
frame and the template independently.
However, the fact that we have to deal with long sequences poses major difficulties. 
For example, the point displacements between the template and a frame can be substantially 
large and severe occlusions of parts of the template can occur in some frames. Even 
state-of-the-art two-frame optical flow methods that are especially designed to deal 
with large displacements or occlusions, \eg \cite{Brox:2011be,revaud2015epicflow}, are 
prone to fail. This is because they lack any additional cues that could help them 
estimate an appropriate initialisation or to disambiguate severe occlusions.

An alternative solution to the multi-frame optical flow problem, also based on 
two-frame optical flow, is to estimate flow between consecutive frames and then 
combine the various solutions. A simple integration of the solutions to obtain long-term 
2D tracks is prone to drift due to error accumulation \cite{RefWorks:292,Brox:2011be}.
This can be improved by the automatic detection of occlusions, gross errors, and other ambiguities 
\cite{sand2008particle,sundaram2010dense,rubinstein2012towards,ricco2012simultaneous,GSRPCA}, 
but any such solution is still limited by the accuracy of the initial two-frame optical 
flow estimations that are completely local in time and do not exploit any temporal cues.

Several recent methods solve the multi-frame optical flow problem directly, by 
implicitly taking into account the rich temporal information that is present in 
non-rigid scenes \cite{Irani02,Torresani:2001iw,Torresani:2002jn,tomasi2012dense,ricco2013video,Garg:2013hu}. 
For example, the long-term 2D trajectories of points on a surface undergoing non-rigid 
deformation are highly correlated and can be compactly described via a linear combination of a 
low-rank trajectory basis. This basis is typically learnt from the input sequence itself. In 
this way, these methods are more robust to occlusions and yield a temporally coherent 
result. However, they rely only on some generic spatial and temporal regularisation 
priors, applicable to any deformable object and do not utilise any prior knowledge 
about the specific object observed in the scene. This makes them fail in more challenging 
conditions that often occur in real-world scenes, such as severe occlusions or 
significant illumination changes, which cannot be disambiguated by temporal regularisation 
alone.
Furthermore, the memory and runtime efficiency of all existing multi-frame optical 
flow methods is limited by the fact that they have to estimate a very large number 
of parameters, i.e.~a set of parameters for every pixel of the template. Specially designed 
parallelisable algorithms, such as primal-dual optimisation schemes \cite{Wedel:DAGM:2009,Garg:2013hu} 
can be adopted. However, they are only efficient on recent GPU hardware.

In this paper, we overcome the aforementioned limitations by incorporating a face-specific 
deformation model into the multi-frame optical flow estimation. We assume a learnt 
deformation basis, rather than one calculated directly from the sequence itself. We 
focus on human faces which are a very commonly considered object in computer vision, 
and dense face correspondences are required in many research areas and applications. 
That is, the establishment of dense correspondences of deformable faces is the first step 
towards high-performance facial expression recognition \cite{koelstra2010dynamic}, facial 
motion capture \cite{beeler2011high} and 3D face reconstruction \cite{Garg:2013gr}. 
Nevertheless, computing dense face correspondences has received limited attention 
\cite{decarlo2000optical,yacoob1996recognizing}. This is attributed to the difficulty of 
developing a statistical model for dense facial flow due to the in-ability of humans to 
densely annotate sequences and the limited robustness of the optical flow techniques 
\cite{decarlo2000optical}. Hence, the research community has focused on developing 
statistical facial models built on a sparse set of landmarks \cite{xiong2013supervised}, 
which provide limited accuracy to the recognition of subtle expressions \cite{li2013spontaneous}. 
In this paper, we build the first, to the best of our knowledge, statistical models of 
dense facial flow by capitalising on the success of recent optical flow techniques applied to 
densely tracking image sequences \cite{Garg:2013hu}. Due to the use of the statistical 
low-rank model, and in contrast to existing approaches, our method is able to deal with 
particularly challenging conditions such as severe occlusions of the face and strong 
illumination changes. Furthermore, the introduction of a known deformation basis drastically 
reduces the dimensionality of the multi-frame optical flow problem and leads to a 
very efficient algorithm.
%%%%%%%%%%%%%%%%%%%%%%%%%%%%%%%%%%%%%%%%%%%%%%%%%%%%%%%%%%%%%%%%%%%%%%%%%%%%%%%%
\section{Contributions}\label{sec:contributions}
%%%%%%%%%%%%%%%%%%%%%%%%%%%%%%%%%%%%%%%%%%%%%%%%%%%%%%%%%%%%%%%%%%%%%%%%%%%%%%%%
We formulate a novel energy minimisation problem for the robust estimation of 
multi-frame optical flow in an expressive sequence of facial images.
Given that the range of motion expressible by a human face is limited, and that
faces themselves are well known to be highly correlated and compactly described 
by a low-dimensional subspace, we build a dense deformation 
basis for faces.

Furthermore, we exploit the even higher correlation between face deformations in 
a specific input sequence by imposing a low-rank prior on the coefficients of the 
deformation basis. This acts as a data-specific regularisation term leading to 
temporally consistent optical flow.
We also incorporate a sparse landmark prior term to guide the flow estimation in sparse 
point locations that are accurately predicted by a state-of-the-art face 
alignment method \cite{kazemi2014one}.
Finally, we formulate the photometric cost by utilising a state-of-the-art dense 
feature descriptor that offers robustness even with the usage of a simple quadratic 
penaliser.
Our proposed model-based problem formulation, in conjunction with the inverse 
compositional strategy and low-rank matrix optimisation that we adopt, leads to a 
highly efficient algorithm for calculating optical flow across a facial sequence.
For experimental evaluation, we show quantitative experiments on a very challenging 
novel benchmark of face sequences with dense ground truth optical flow based on motion 
capture data. We also provide qualitative results on a real sequence displaying
fast motion and natural occlusions.

%
%Human faces are a special class of deforming objects that
%
%In technological terms, human faces consist one of the most important special classes of non-rigid objects. 
%The accurate estimation of multiframe optical flow of face videos under varying conditions can be extremely 
%useful in several computer vision tasks and real-world applications, such as dense 3D face reconstruction, 
%augmented reality, human-computer interaction ....





%For example, optical flow
%and other dense correspondence methods play an important role in problems
%such as 3D facial surface recovery
%\cite{Garg:2013gr,RefWorks:311,Alexander:2013kd,Garrido:2013di},
%facial recognition \cite{Hsieh:2010jg,Bao:2009gi,Sanchez:2011bz,Liu:2011jv}
%and facial alignment \cite{RefWorks:95,decarlo2000optical,Vetter:1997fi}.
%Broadly, optical flow methods describe procedures to relate pixels in
%one image to pixels in another image of the same object. Typically, this
%is achieved by finding a displacement field that describes how the pixels
%in the reference image have moved in the input image. More importantly,
%this displacement field can be thought of as a sampling, or warping, of
%the input image pixels back onto the reference image.
%
%
%O
%

%
%However, several challenges arise ....
%
%multiframe optical flow
%
%
%Even though in a multi-frame setting we
%
%Classical optical flow algorithms deal with two images only and are based on the assumption that the displacement field is relatively small.
%basic assumption is brightness or colour constancy
%
%
%
%In order to work on a multiframe setting ....
%limitations ...
%
%
%multiframe methods ...
%
%However limitations





% Even given a pair of images lacking external occlusions
% or lighting variation, computing optical flow between the images is ambiguous.
% In practise, objects are often occluded and lighting in a scene
% is rarely absolutely consistent. In the most simple optical flow formulation, the
% range of pixel motion is also assumed to small, which is rarely the case unless
% the two images were acquired sequentially from a smooth sequence. Given a smooth
% sequence, the difference between the first and last frame of a deforming non-rigid
% object can be very large and thus the small neighbourhood assumption is violated.
% However, remarkable progress has been made in the area of optical flow and
% recent benchmarks such as the MPI Sintel \cite{Butler:2012cv} dataset have shown
% that state-of-the-art methods can achieve strong performance on extremely
% challenging scenes.
