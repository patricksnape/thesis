%%%%%%%%%%%%%%%%%%%%%%%%%%%%%%%%%%%%%%%%%%%%%%%%%%%%%%%%%%%%%%%%%%%%%%%%%%%%%%%%
\section{Further Related Work}\label{sec:related_work}
%%%%%%%%%%%%%%%%%%%%%%%%%%%%%%%%%%%%%%%%%%%%%%%%%%%%%%%%%%%%%%%%%%%%%%%%%%%%%%%%
There is a very large body of work on facial alignment that largely revolves
around the concept of identifying a set of sparse target landmarks within an
image. The most relevant algorithm to our proposed method is that of the
Active Appearance Model (AAM) \cite{cootes2001active}, particularly the variation by
Baker and Matthews \cite{matthews2004active} that relates AAMs to the Lucas-Kanade
\cite{lucas1981iterative,baker2004lucas} optical flow literature. However, our method
does not incorporate an appearance model and relies on a single given template
image and is thus closer in nature to the original Lucas-Kanade algorithm. 
Our algorithm also places a low-rank constraint on the shape model coefficients
enforcing a form of temporal consistency, which has not been previously 
considered.

It is also important to note that, other than a couple of examples
\cite{Ramnath:2008jj,Anderson:2013dl}, the AAM literature has focused on the
recovery of sparse landmarks, not a dense motion field as in this work.
Even in the cases of \cite{Ramnath:2008jj,Anderson:2013dl}, the warping of the
input images is achieved via an interpolation method such as
piecewise affine or thin-plate splines. In this work, our warping
method is derived directly from the deformation basis itself and thus recovers
dense correspondences.
As we show in Section~\ref{sec:method}, the linear nature of our warp allows
us to derive a very efficient optimisation strategy,
based on the Inverse Compositional algorithm proposed by Baker and Matthews
\cite{baker2004lucas}. 

Finally, the work of Kemelmacher-Schlizerman \mbox{\etal} \cite{kemelmacher2012collection}
is relevant as it considers learning deformation fields between images of faces.
However, \cite{kemelmacher2012collection} considers an optical flow method
as a key component of the method and does not propose a novel optical formulation,
in contrast to our proposed model-based algorithm.
