% Command for entries in a timeline
\DeclareDocumentCommand{\ytl}{ O{0.2\textwidth} O{0.8\textwidth} m m }{
\parbox[b]{#1}{\hfill{\color{cyan}\bfseries\sffamily #3}~$\cdots\cdots$~}\makebox[0pt][c]{$\bullet$}%
\vrule\quad%
\parbox[c]{#2}{\vspace{7pt}\color{red!40!black!80}\raggedright\sffamily #4.\\[7pt]}\\[-3pt]%
}
\newcommand{\lsubref}[1]{(\protect\subref{#1})}
\newcommand*{\nolink}[1]{\begin{NoHyper}#1\end{NoHyper}}
\newcommand{\printauthor}[1]{\nolink{\citeauthor{#1}}}

% Add a period to the end of an abbreviation unless there's one
% already, then \xspace.
\makeatletter
\DeclareRobustCommand\onedot{\futurelet\@let@token\@onedot}
\def\@onedot{\ifx\@let@token.\else.\null\fi\xspace}

\def\eg{\emph{e.g}\onedot} \def\Eg{\emph{E.g}\onedot}
\def\ie{\emph{i.e}\onedot} \def\Ie{\emph{I.e}\onedot}
\def\cf{\emph{c.f}\onedot} \def\Cf{\emph{C.f}\onedot}
\def\etc{\emph{etc}\onedot} \def\vs{\emph{vs}\onedot}
\def\wrt{w.r.t\onedot} \def\dof{d.o.f\onedot}
\def\etal{\emph{et al}\onedot}
\makeatother

% Example of inserting a timeline - which I will do AFTER I have formalised
% all of the citations.
% \begin{table}
% 	\centering
% 	\begin{minipage}[t]{.8\linewidth}
% 		\color{gray}
% 		\rule{\linewidth}{1pt}

% 		\ytl{1947}{AT and T Bell Labs develop the idea of cellular phones}
% 		\ytl{1968}{Xerox Palo Alto Research Centre envisage the `Dynabook'}
% 		\ytl{1971}{Busicom 'Handy-LE' Calculator}

% 		\bigskip
% 		\rule{\linewidth}{1pt}%
% 	\end{minipage}%
% 	\caption{Timeline of something.}
% \end{table}
