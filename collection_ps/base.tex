%%%%%%%%%%%%%%%%%%%%%%%%%%%%%%%%%%%%%%%%%%%%%%%%%%%%%%%%%%%%%%%%%%%%%%%%%%%%%%%%
\chapter{Image Collections: Class-Specific Photometric Stereo}\label{ch:imag_coll}
%%%%%%%%%%%%%%%%%%%%%%%%%%%%%%%%%%%%%%%%%%%%%%%%%%%%%%%%%%%%%%%%%%%%%%%%%%%%%%%%
\minitoc{}
%%%%%%%%%%%%%%%%%%%%%%%%%%%%%%%%%%%%%%%%%%%%%%%%%%%%%%%%%%%%%%%%%%%%%%%%%%%%%%%%
A common relaxation of the single image shape recovery problem is to reduce
ambiguities by considering more than one image. In the most difficult case,
these images will not have any temporal relationship to one another and thus
will have been captured under a variety of conditions. In the case of
considering illumination constraints for shape recovery a number of images
of a single object in a fixed position under known illumination conditions
may be considered. This family of methods are called Photometric Stereo (PS)
methods and were discussed in detail in \cref{sec:bg_ps}. However, the two
primary constraints, that the illumination conditions are known and that each
image contains pixel-wise correspondence of a single object under the exact
same view, are highly restrictive. The known illumination constraint can be
relaxed by performing
Uncalibrated Photometric Stereo (U-PS)~\cite{hayakawa1994photometric,%
basri2007photometric} which performs remarkably well for
human faces~\cite{KemelmacherShlizerman:2013iv,kemelmacher2011face,%
kemelmacher2012collection}. In contrast, the correspondence problem is an
unrealistic constraint for living objects. Even under highly constrained
scenarios it can be very difficult for a person to remain completely still
whilst multiple images are being captured.
For example, \cref{fig:yaleb_movement_example} shows Subject 6 from the
Yale B~\cite{georghiades2001fromfew} database where 64 images were captured
under different illuminations provided by a geodesic lighting rig in
approximately 2 seconds. Despite the relatively short capture time, the subjects
still often moved during capture which can cause artefacts during 3D surface
recovery. For example, \citet{harrison2012translational} showed that rectifying
translational movements from the Yale B images provided a less biased
reconstruction. However, this kind of translational alignment requires that the
images be approximately aligned to begin with and cannot handle common issues
such as occlusion. It can also not handle the large amount of variation that
may be present between individuals.

In this chapter, we look to borrow from ideas seen within the Photometric Stereo
literature in order to recover shape from objects under unconstrained settings
using \textit{only a set of images}. Typically, these types of unconstrained
photo collections are called ``in-the-wild''. We seek to construct our models in
an automatic manner, without manual feature point placement or careful selection
of the input images.

In particular, we seek to recover the shape of the object by exploiting the
similarity within the object class. In the case of faces, there are millions of
available images that can be utilised to build in-the-wild models. However,
recovering shape from these images is incredibly challenging, as they have been
captured in completely unconstrained conditions. No knowledge of the lighting
conditions, the facial location or the camera geometric properties are provided
with the images. To address these problems, we propose to recover a class
specific spherical harmonic (SH) basis that exploits the low-rank structure of
faces~\cite{georghiades2001fromfew,Basri:2003ie}. Spherical harmonics are ideal for
this purpose as they can be approximated by a low-dimensional linear subspace
\cite{Basri:2003ie,ramamoorthi2001relationship}.
By using the first order SH, $87.5\%$ of
the low-frequency component of the lighting is approximated. The first order SH
can then be used to recover 3D shape as their discrete approximation directly
incorporates the normals of the object.

The most relevant techniques to this chapter involve recovering shape from a
collection of images under varying illumination. Typically, this involves
solving some form of uncalibrated photometric stereo problem
\cite{basri2007photometric,papadhimitri2014closed,papadhimitri2014closed}. However, traditional uncalibrated
photometric stereo techniques still assume that the images provided have been
captured by a photometric stereo system under explicit directed lighting.
The relaxation of the uncalibrated photometric stereo problem to a class of
objects further increases the ambiguity inherent within the problem.
Specifically, it is now necessary to separate the SH lighting from the identity
of the individuals. This problem has been approached for both shape recovery and
facial recognition purposes
\cite{lee2005bilinear,lee2005estimation,minsik2014realtime,minsik2013robust,zhou2007appearance}. 
\citet{lee2005bilinear,lee2005estimation} recover facial shape by separating illumination
from identity in a manner that is similar to 3DMMs. 
\citet{lee2005bilinear,lee2005estimation} separate
\citet{minsik2014realtime,minsik2013robust} the appearance and identity via a low rank
tensor decomposition that provides a very efficient reconstruction methodology.
However, both Lee \etal~and Minsik \etal~still rely on previously built dense 3D
models to perform their decomposition.

Recently, \citet{KemelmacherShlizerman:2013iv} proposed a method for
building morphable models from images of faces downloaded from the Internet.
This work shares similarities with ours in that it attempts to build a subspace
that explicitly separates shape and appearance. However, in~\cite{KemelmacherShlizerman:2013iv}
they do not investigate a robust decomposition, but instead rely on a time
consuming optical flow~\cite{kemelmacher2012collection} based registration process to remove
outliers from the images. Although this methodology allows for expression
transfer, it does not allow the recovered shapes to be used within existing
facial alignment techniques such as Active Appearance Models (AAMs). In
contrast, our use of efficient facial alignment techniques to acquire
correspondence substantially reduces our training time. It also allows our
recovered basis to be coupled with the alignment techniques for simultaneous
facial landmark localisation and dense surface recovery. However, the coarse
geometric alignment we employ is more sensitive to corruptions such as
occlusions and extreme facial pose. For this reason, we employ a low rank
constraint
\cite{candes2011robust,peng2012rasl,sagonas2014raps,cheng2013rank,wu2010robust,lu2013uncalibrated}
to help remove these high frequency errors whilst maintaining the low frequency
lighting variations. Although we share a similar optimisation framework to other
robust principal component analysis problems such as
\cite{candes2011robust,peng2012rasl,wu2010robust,lu2013uncalibrated}, we are the first to propose
a low-rank decomposition that recovers a subspace of spherical harmonics.
%%%%%%%%%%%%%%%%%%%%%%%%%%%%%%%%%%%%%%%%
\begin{figure}
    \hspace*{\fill}
    \includegraphics[width=0.22\textwidth]{collection_ps/images/yaleb/yaleB06_P00A+000E+20} \hfill
    \includegraphics[width=0.22\textwidth]{collection_ps/images/yaleb/yaleB06_P00A+000E-20} \hfill
    \includegraphics[width=0.22\textwidth]{collection_ps/images/yaleb/yaleB06_P00A+005E-10} \hfill
    \includegraphics[width=0.22\textwidth]{collection_ps/images/yaleb/yaleB06_P00A-020E+10}
    \hspace*{\fill}
\label{fig:yaleb_movement_example}
    \caption{An example of unintentional movement during capture of a live
             human subject, from the Yale-B database~\cite{georghiades2001fromfew}.
             This movement violates the pixel-wise correspondence assumption
             made by traditional Calibrated and Uncalibrated Photometric
             Stereo.}
\end{figure}
%%%%%%%%%%%%%%%%%%%%%%%%%%%%%%%%%%%%%%%%
%%%%%%%%%%%%%%%%%%%%%%%%%%%%%%%%%%%%%%%%%%%%%%%%%%%%%%%%%%%%%%%%%%%%%%%%%%%%%%%%
%%%%%%%%%%%%%%%%%%%%%%%%%%%%%%%%%%%%%%%%%%%%%%%%%%%%%%%%%%%%%%%%%%%%%%%%%%%%%%%%
\section{Recovering Shape from Image Collections}\label{sec:imag_coll_problem}
%%%%%%%%%%%%%%%%%%%%%%%%%%%%%%%%%%%%%%%%%%%%%%%%%%%%%%%%%%%%%%%%%%%%%%%%%%%%%%%%
In this section we describe how a spherical harmonic (SH) basis can be recovered
using uncalibrated photometric stereo (PS) techniques. We then describe how this
problem generalises to a multi-person dataset and how a representation of shape
can be recovered per image. Finally, we discuss the importance of achieving
correspondence between the images in an efficient and scalable manner.
%%%%%%%%%%%%%%%%%%%%%%%%%%%%%%%%%%%%%%%%%%%%%%%%%%%%%%%%%%%%%%%%%%%%%%%%%%%%%%%%
\subsection{Spherical Harmonic Bases}\label{subsec:imag_coll_spherical_harmonic}
%%%%%%%%%%%%%%%%%%%%%%%%%%%%%%%%%%%%%%%%%%%%%%%%%%%%%%%%%%%%%%%%%%%%%%%%%%%%%%%%
The lambertian reflectance model states that matte materials reflect light
uniformly in all directions. This simple image formation model assumes that the
intensity of light reflecting from a surface is a function of the shape of the
surface and a linear combination of point light sources. More formally, given an
image $I(x, y)$, the intensity at a given pixel $(x, y)$ of a convex lambertian
surface illuminated by a single light, can be expressed as
%%%%%%%%%%%%%%%%%%%
\begin{equation}\label{eq:lambert_law}
    I(x, y) = \rho(x, y) \bb{l}^T \bb{n}(x, y),
\end{equation}
%%%%%%%%%%%%%%%%%%%
where $\rho(x, y)$ is the albedo at the pixel and represents surface
reflectivity, $\bb{l}$ is the vector denoting the single point light source
illuminating the object and $\bb{n}(x, y)$ is the surface normal at the
pixel.

If we now consider a collection of directional light sources placed at infinity,
the lighting intensity at a given pixel can be expressed as a non-negative
function of the unit sphere using a sum of spherical harmonics. Formally,
%%%%%%%%%%%%%%%%%%%
\begin{equation}\label{eq:spherical_harmonics}
    I(x, y) = \sum^{\infty}_{n=0} \sum^{n}_{m=-n} \alpha_n \; \ell_{nm} \; \rho(x,y) \; Y_{nm}(\bb{n}(x,y)),
\end{equation}
%%%%%%%%%%%%%%%%%%%
where $\alpha_n = \pi, 2\pi/3, \pi/4, \ldots$, $\ell_{nm}$ are the coefficients
of the harmonic expansion of the lighting and $Y_{nm}(\bb{n}(x,y))$ are the
surface SH functions evaluated at the surface normal, $\bb{n}(x,y)$. As
$n \rightarrow \infty$, the coefficients tend to zero, and thus the SH can be
accurately represented by the lower order harmonics. \citet{frolova2004accuracy}, it
was shown that the first order SH function is guaranteed to represent at least
$87.5\%$ of the reflectance and experimentally verified to recover up to $95\%$
in the case of faces. The first order SH expansion is also directly related to
the objects surface normals:
%%%%%%%%%%%%%%%%%%%
\begin{equation}\label{eq:first_order_spherical_harmonics}
    \begin{aligned}
        \bb{Y}(\bb{n}(x, y))  = \rho(x, y) {[ 1, \bb{n}_{x}(x, y), \bb{n}_{y}(x, y), \bb{n}_{z}(x, y) ]}^T,
   \end{aligned}
\end{equation}
%%%%%%%%%%%%%%%%%%%
where $\bb{n}_{i}(x, y)$ denotes the $i$th component of the normal vector. 
This is a particularly useful result as recovering the first order SH means 
directly recovering a representation of shape for an object.
%%%%%%%%%%%%%%%%%%%%%%%%%%%%%%%%%%%%%%%%%%%%%%%%%%%%%%%%%%%%%%%%%%%%%%%%%%%%%%%%
\subsection{Uncalibrated Photometric Stereo}\label{subsec:imag_coll_uncalibrated_ps}
%%%%%%%%%%%%%%%%%%%%%%%%%%%%%%%%%%%%%%%%%%%%%%%%%%%%%%%%%%%%%%%%%%%%%%%%%%%%%%%%
Classical photometric stereo (PS) seeks to recover the normals of a convex
object given a number of images under known different lighting with known
directions. Traditionally, the following decomposition is performed
%%%%%%%%%%%%%%%%%%%
\begin{equation}\label{eq:general_ps}
    \bb{X} = \bb{N} \tilde{\bb{L}},
\end{equation}
%%%%%%%%%%%%%%%%%%%
where $\bb{X} \in \bb{R}^{d \times n}$ is the matrix of observations,
and each of the $n$ columns represents a vectorised image of the object with $d$
total pixels, $\bb{N} \in \bb{R}^{d \times 3}$ contains the normal at
every pixel and $\tilde{\bb{L}} \in \bb{R}^{3 \times n}$ is the matrix
of lighting vectors per image. Assuming accurate light vectors and no shadowing
artifacts, this problem is trivially solved as a linear least squares problem.
Photometric stereo has been shown to provide accurate facial reconstructions
despite faces not representing true lambertian objects. For example, there are
many publicly available facial PS datasets such as the 
Photoface Database~\cite{zafeiriou2013photoface} and the 
Yale B~\cite{georghiades2001fromfew} dataset.

If the lighting vectors are inaccurate or unknown, then PS is said to be
uncalibrated. \citet{basri2007photometric} showed that $\bb{X}$ can
be decomposed via a rank constrained singular value decomposition (SVD) to
recover the SH bases and the lighting coefficients in the uncalibrated setting.
First order SH are recovered by a rank 4 SVD and are accurate up to a $4 \times 4$ 
generalised Lorentz transformation. By enforcing constraints such as the
integrability constraint~\cite{frankot1988method}, the first order SH, and thus the
normals, can be recovered up to a generalised bas relief ambiguity (GBR).
Formally, uncalibrated PS looks to recover
%%%%%%%%%%%%%%%%%%%
\begin{equation}\label{eq:uncalibrated_ps}
    \bb{X} = \bb{B} \bb{L},
\end{equation}
%%%%%%%%%%%%%%%%%%%
where $\bb{X}$ is as before, $\bb{B} \in \bb{R}^{d \times 4}$ 
contains the first order SH basis images and
$\bb{L} \in \bb{R}^{4 \times n}$ is the matrix of lighting coefficients. 
As previously mentioned, the solution to this problem is found by performing an
SVD,
$\bb{X} =\nobreak \bb{U} \bb{\Sigma} \bb{V}^T$, $\bb{B} = \bb{U} \sqrt{\bb{\Sigma}}$
and $\bb{L} = \sqrt{\bb{\Sigma}} \bb{V}^T$. 
Uncalibrated PS is useful as it is not always possible to recover accurate 
lighting estimations for every image.
%%%%%%%%%%%%%%%%%%%%%%%%%%%%%%%%%%%%%%%%%%%%%%%%%%%%%%%%%%%%%%%%%%%%%%%%%%%%%%%%
\subsection{Class Specific Uncalibrated Photometric Stereo}\label{subsec:imag_coll_class_uncalibrated_ps}
%%%%%%%%%%%%%%%%%%%%%%%%%%%%%%%%%%%%%%%%%%%%%%%%%%%%%%%%%%%%%%%%%%%%%%%%%%%%%%%%
A generalisation of the uncalibrated PS problem for a specific class involves
recovering a joint basis of deformation and illumination. In the case of SH for
faces, this means attempting to separate the identity and deformation of the
individual from the spherical harmonic representation of their face. 
This problem is a classic example of a bilinear
decomposition problem and has been previously studied for use in 3D surface
recovery~\cite{zhou2007appearance,minsik2014realtime,minsik2011fast,%
lee2005bilinear,KemelmacherShlizerman:2013iv}. In
the case of SH, we seek to recover a low dimensional linear subspace that can
recover normals for multiple individuals. This subspace implies that a face can
be accurately reconstructed using a linear combination of basis shapes. This
assumption is commonly employed in algorithms such as the 3DMM and AAMs.
Assuming that we want to recover $k$ such components for our shape subspace, and
that we are using the first order SH, we will recover a $d \times 4k$ basis
matrix that allows us to recover 3D facial shape for multiple individuals.
Formally,
%%%%%%%%%%%%%%%%%%%%%%%%%%
\begin{equation}\label{eq:class_specific_ps}
        \bb{X} = \bb{B} (\bb{L} \ast \bb{C}),
\end{equation}
%%%%%%%%%%%%%%%%%%%%%%%%%%
where $\bb{B} \in \bb{R}^{d \times 4k}$ is the linear basis,
$\bb{L} \in \bb{R}^{4 \times n}$ is the matrix of first order SH lighting
coefficients, $\bb{C} \in \bb{R}^{k \times n}$ is the matrix of shape
coefficients and $(\cdot \ast \cdot)$ denotes the Khatri-Rao
product~\cite{khatri1968solutions}. In fact, this is the exact decomposition problem
solved by \citet{KemelmacherShlizerman:2013iv} where they denote the combined coefficients
matrix as $\bb{P} = \bb{L} \ast \bb{C}$. This was partially
recognised by \citet{zhou2007appearance}, however they recover the lighting and shape
coefficient separately by iteratively solving for each in an alternating
fashion. \citet{zhou2007appearance} also do not provide any examples of the quality of the
shape estimate that they recover.

\citet{minsik2014realtime,minsik2011fast} also attempt this decomposition by
posing the problem in the form of a tensor. The decomposition can then be solved
by applying a multilinear SVD.\@ However, multilinear SVD requires a tensor
representation and thus these techniques require prior data to recover results.
A tensor representation is useful, however, for illustrating how to recover the
$d \times 4$ first order SH for an individual, given their coefficients vector
$\bb{c}_i \in \bb{R}^{k \times 1}$. We reshape the basis matrix
$\bb{B}$ as a tensor which we denote $\bb{S} \in \bb{R}^{d \times k
\times 4}$. The tensor product along the second mode, $\bb{S} \times_2 \bb{c}_i$, 
recovers the person specific shape of the $i$th column of
$\bb{X}$. To recover $\bb{B}$ from $\bb{S}$, we perform
matricisation of $\bb{S}$ along the first mode, denoted $\bb{S}_{(1)}$,
to yield $\bb{S}_{(1)} = \bb{B} \in \bb{R}^{d \times 4k}$.

The problem given in~\eqref{eq:class_specific_ps} can now be minimised using
classic matrix decomposition optimisation frameworks, which we examine in
detail in the next section.
%%%%%%%%%%%%%%%%%%%%%%%%%%%%%%%%%%%%%%%%%%%%%%%%%%%%%%%%%%%%%%%%%%%%%%%%%%%%%%%%
\subsection{Robust Construction Of Spherical Harmonic Bases}\label{subsec:imag_coll_robust_sh_basis}
%%%%%%%%%%%%%%%%%%%%%%%%%%%%%%%%%%%%%%%%%%%%%%%%%%%%%%%%%%%%%%%%%%%%%%%%%%%%%%%%
Inspired by recent advances in robust low-rank subspace
recovery~\cite{candes2011robust}, we seek to modify \cref{eq:class_specific_ps} to
include new constraints that impose robustness to sparse outliers.
As mentioned previously, faces
can be accurately reconstructed by a linear combination of faces taken from a
low-dimensional basis. Therefore, we propose to decompose the image matrix into
a low-rank part ($\bb{A}$) capturing the low frequency shape information and a
sparse part ($\bb{E}$) accounting for gross but sparse noise such as partial
occlusions and pixel corruptions. To promote low-rank and sparsity the nuclear
norm (denote by $\lVert \bb{\cdot} \rVert_{*}$) and the $\ell_1$-norm (denote by
$\lVert \bb{\cdot} \rVert_{1}$) are employed, respectively. Formally we propose
to solve the following non-convex optimisation problem:
%%%%%%%%%%%%%%%%%%%
\begin{equation}
\begin{aligned}
    &\argmin_{\bb{A},\bb{E},\bb{B},\bb{L},\bb{C}} & &\lVert \bb{A} \rVert_{*} + \lambda \lVert \bb{E} \rVert_{1} + \frac{\mu}{2} \lVert \bb{A} - \bb{B} (\bb{L} \ast \bb{C}) \rVert^{2}_{F} \\
    & \text{subject to} & & \bb{X} = \bb{A} + \bb{E}, \; \bb{B}^T \bb{B} = \bb{I}.
\end{aligned}
\label{eq:robust_sh_problem}
\end{equation}
%%%%%%%%%%%%%%%%%%%
Although the above problem is non-convex, an accurate solution can be obtained
by employing the Alternating Directions Method (ADM)~\cite{bertsekas2014constrained}. That
is, to minimise the following augmented Lagrangian function:
%%%%%%%%%%%%%%%%%%%
\begin{equation}
    \begin{aligned}
        \mathcal{L}(\bb{A},\bb{E},\bb{B},\bb{C},\bb{L},\bb{Y}) &= \lVert \bb{A} \rVert_{*} + \lambda \lVert \bb{E} \rVert_{1} + \frac{\mu}{2} \lVert \bb{A} - \bb{B} (\bb{L} \ast \bb{C}) \rVert^{2}_{F} \quad + \\
        & \quad \Tr{(\bb{Y}^T(\bb{X} - \bb{A} - \bb{E}))} +  \frac{\mu}{2} \lVert \bb{X} - \bb{A} - \bb{E} \rVert_{F}^{2},
    \end{aligned}\label{eq:augmented_lagrangian}
\end{equation}
%%%%%%%%%%%%%%%%%%%
with respect to $\bb{B}^T \bb{B} = \bb{I}$. Let $t$ denote the iteration index. 
Given $\bb{A}_{[t]}$, $\bb{E}_{[t]}$, $\bb{B}_{[t]}$, $\bb{C}_{[t]}$, $\bb{L}_{[t]}$, $\bb{Y}_{[t]}$
and $\mu_{[t]}$, the iteration of ADM for \cref{eq:robust_sh_problem} reads:
%%%%%%%%%%%%%%%%%%%
\begin{align}
    \bb{A}_{[t+1]} &= \lVert \bb{A}_{[t]} \rVert_{*} + \frac{\mu_{[t]}}{2} \biggl( \lVert \bb{A}_{[t]} - \bb{B}_{[t]} (\bb{L}_{[t]} \ast \bb{C}_{[t]}) \rVert^{2}_{F} + \lVert \bb{X} - \bb{A}_{[t]} - \bb{E}_{[t]} + \frac{\bb{Y}_{[t]}}{\mu_{[t]}} \rVert_{F}^{2} \biggr), \label{eq:text_A_update} \\
    \bb{E}_{[t+1]} &= \argmin_{\bb{E}_{[t]}} \lambda \lVert \bb{E}_{[t]} \rVert_{1} + \frac{\mu_{[t]}}{2} \lVert \bb{X} - \bb{A}_{[t+1]} - \bb{E}_{[t]} + \frac{\bb{Y}_{[t]}}{\mu_{[t]}} \rVert_{F}^{2}, \label{eq:text_E_update} \\
    \bb{B}_{[t+1]} &= \argmin_{\bb{B}_{[t]}^T \bb{B}_{[t]} = \bb{I}} \frac{\mu_{[t]}}{2} \lVert \bb{A}_{[t+1]} - \bb{B}_{[t]} (\bb{L}_{[t]} \ast \bb{C}_{[t]}) \rVert_{F}^{2}, \label{eq:text_B_update} \\
    \left[ \bb{L}_{[t+1]}, \bb{C}_{[t+1]} \right] &= \argmin_{\bb{L}_{[t]}, \bb{C}_{[t]}} \frac{\mu_{[t]}}{2} \lVert \bb{A}_{[t+1]} - \bb{B}_{[t+1]} (\bb{L}_{[t]} \ast \bb{C}_{[t]}) \rVert_{F}^{2}. \label{eq:text_LC_update}
\end{align}
%%%%%%%%%%%%%%%%%%%
Subproblem~\eqref{eq:text_A_update} admits a closed-form solution, given by the
singular value thresholding (SVT)~\cite{cai2010singular} operator as:
%%%%%%%%%%%%%%%%%%%
\begin{equation}\label{eq:text_A_SVT}
    \bb{A}_{[t+1]} = D_{\mu_{[t]}^{-1}} \left[ \bb{M}_{[t]}  - \bb{A}_{[t]} + \bb{X} - \bb{E}_{[t]} + \frac{\bb{Y}_{[t]}}{\mu_{[t]}} \right],
\end{equation}
%%%%%%%%%%%%%%%%%%%
where $\bb{M}_{[t]} = \bb{B}_{[t]} (\bb{L}_{[t]} \ast \bb{C}_{[t]})$ is introduced 
for brevity of the equation and the SVT is defined as
$D_{\tau}(\bb{Q}) = \bb{U} \bb{S}_{\tau} \bb{V}^T$ for any matrix $\bb{Q}$ with SVD:\@
$\bb{Q} = \bb{U} \bb{S} \bb{V}^T$. 
Subproblem~\eqref{eq:text_E_update} has a unique solution that is obtained via 
the elementwise shrinkage operator~\cite{candes2011robust}. The shrinkage operator 
is defined as $\mathcal{S}_{\tau}[q]=\mbox{sgn}(q) \: \max(|q|-\tau, 0)$. 
Therefore, the solution of~\eqref{eq:text_E_update} is
%%%%%%%%%%%%%%%%%%%
\begin{equation}\label{eq:text_E_shrinkage}
    \bb{E}_{[t+1]} = \mathcal{S}_{\lambda \mu_{[t]}^{-1}} \left[ \bb{X} - \bb{A}_{[t+1]} + \frac{\bb{Y}_{[t]}}{\mu_{[t]}} \right].
\end{equation}
%%%%%%%%%%%%%%%%%%%
{ % Courtesy of the wonderful 4G!
%%%%%%%%%%%%%%%%%%%
\def\t#1{{\bb{#1}_{[t]}}}
\def\tp#1{{\bb{#1}_{[t+1]}}}
%%%%%%%%%%%%%%%%%%%
\noindent Subproblem~\eqref{eq:text_B_update} is a reduced rank Procrustes
Rotation problem~\cite{zou2006sparse}. Its solution is given by
$\bb{B}_{[t]} = \bb{U} \bb{V}^{\top}$ with
%%%%%%%%%%%%%%%%%%%
\begin{equation}\label{eq:text_B_update2}
    \tp A {\left(\t L \ast \t C\right)}^{\top} = \bb{U} \bb{\Sigma} \bb{V^{\top}},
\end{equation} 
%%%%%%%%%%%%%%%%%%%
being the SVD of $\tp A {\left(\t L \ast \t C\right)}^{\top}$. However, due the
unitary invariance of the Frobenius norm, Equation~\ref{eq:text_LC_update}
becomes
%%%%%%%%%%%%%%%%%%%
\begin{equation}\label{eq:text_LC_update2}
    \argmin_{\bb{L}_{[t+1]}, \bb{C}_{[t+1]}} \lVert \bb{B}_{[t+1]}^T\bb{A}_{[t+1]} - \bb{L}_{[t]} \ast \bb{C}_{[t]} \rVert_{F}^{2}.
\end{equation}
%%%%%%%%%%%%%%%%%%%
Subproblem~(\ref{eq:text_LC_update},~\ref{eq:text_LC_update2}) is a least 
squares factorisation of a Khatri-Rao product~\cite{roemer2010tensor}, which is 
solved as follows:
Let $\bb{Q} = \tp B^{\top} \tp A, \bb{L} = \t L, \text{and} \; \bb{C} = \t C$. 
Furthermore, let $\bb{q}_i, \bb{l}_i, \text{and} \; \bb{c}_i$ be the $i_{th}$ 
columns of 
matrices $\bb{Q}, \bb{L}, \text{and} \; \bb{C}$, respectively. 
Clearly $\bb{q}_i = \bb{l}_i \oplus \bb{c}_i$, where $\oplus$ denotes the Kronecker 
product. For each column of $\bb{Q}$: Reshape $\bb{q}_i$ into a 
matrix $\tilde{\bb{Q}}_i \in \bb{R}^{4 \times k}$ 
such that 
$\text{vec}\left(\tilde{\bb{Q}}_i\right) = \bb{q}_i$. 
Obviously, $\tilde{\bb{Q}}_i = \bb{c}_i \cdot {\bb{l}_i}^{\top}$ is a rank-one matrix.
Compute the SVD of
$\tilde{\bb{Q}}_i$ as $\tilde{\bb{Q}}_i = \bb{U}_i \bb{\Sigma}_i {\bb{V}_i}^{\top}$. 
The best rank-one approximation of $\tilde{\bb{Q}}_i$ is obtained by truncating the SVD as:
$\bb{l}_i = \bb{u}_i \sqrt{\sigma_1}$ and $\bb{c}_i = \sqrt{\sigma_1} \bb{v}_i$,
where $\bb{u}_i$ and $\bb{v}_i$ are the first column vectors of $\bb{U}_i$
and $\bb{V}_i$, respectively, and $\sigma_1$ is the largest singular value.
The ADM for solving~\eqref{eq:robust_sh_problem} is outlined in 
\cref{alg:imag_coll_adm_solution}.
}

It is important to note that there are inherent ambiguities in this
decomposition, both from the SVD to recover $\bb{B}$ and in the Khatri-Rao
factorisation to recover $\bb{L}$ and $\bb{C}$. In particular, we are most
concerned about how they may affect the recovered normals before we integrate
them to recover depth. In order to resolve these ambiguities, we take the
simplest possible approach, we recover the ambiguity matrix from a template set
of normals provided by a known mean face.
%%%%%%%%%%%%%%%%%%%%%%%%%%%%%%%%%%%%%%%%%%%%%%%%%%%%%%%%%%%%%%%%%%%%%%%%%%%%%%%%
%%%%%%%%%%%%%%%%%%%%%%%%%%%%%%%%%%%%%%%%%%%%%%%%%%%%%%%%%%%%%%%%%%%%%%%%%%%%%%%%
\begin{algorithm}[t]
    \caption{Solving~\eqref{eq:robust_sh_problem} by the ADM method.}
\label{alg:adm_solution}
    {\scriptsize\textbf{Input:} Data Matrix $\bb{X} \in \mathbb{R}^{d \times n}$ and parameter $\lambda$.} \\
    {\scriptsize\textbf{Output:} Matrices $\bb{A}$, $\bb{E}$, $\bb{B}$, $\bb{C}$, $\bb{L}$.}  \\
{
    \scriptsize
    % Define commands
    \def\At{{\bb{A}_{[t]}}}
    \def\Et{{\bb{E}_{[t]}}}
    \def\Ct{{\bb{C}_{[t]}}}
    \def\Lt{{\bb{L}_{[t]}}}
    \def\Bt{{\bb{B}_{[t]}}}
    \def\Yt{{\bb{Y}_{[t]}}}
    \def\mut{{\mu_{[t]}}}
    \def\X{{\bb{X}}}
    
    \begin{algorithmic}[1]
        \STATE{} Initialise: $\bb{A}_{[0]} = 0$, $\bb{E}_{[0]} = 0$, $\bb{B}_{[0]} = 0$, $\bb{C}_{[0]} = 0$, $\bb{L}_{[0]} = 0$, $\bb{Y}_{[0]} = 0$, $\mu_{[0]} = 10^{-6}$, $\rho = 1.1$, $\epsilon = 10^{-8}$
        \WHILE{not converged do} %(Outer loop)
            \STATE{} Fix $\Et$, $\Bt$, $\Ct$, $\Lt$ and update $\bb{A}_{[t + 1]}$ by
                \begin{equation}\label{eq:A_update}
                    \bb{A}_{[t+1]} = D_{\mu_{[t]}^{-1}} \left[  \Bt ( \Lt \ast \Ct ) - \At + \X - \Et + \frac{\Yt}{\mut} \right]
                \end{equation}
            \STATE{} Fix $\bb{A}_{[t+1]}$, $\Lt$, $\Ct$, $\Lt$ and update $\bb{A}_{[t + 1]}$ by
                \begin{equation}\label{eq:E_update}
                    \bb{E}_{[t+1]} = \mathcal{S}_{\lambda \mut^{-1}} \left[ \X - \bb{A}_{[t+1]} + \frac{\Yt}{\mut} \right]
                \end{equation}
            \STATE{} Update $\bb{B}_{[t+1]}$ by first performing the SVD on:
                \begin{equation}\label{eq:ALC_svd}
                    \bb{A}_{[t+1]} {(\Lt \ast \Ct)}^T = \bb{U} \bb{\Sigma} \bb{V}, \; \bb{B}_{[t+1]} = \bb{U} \bb{V}^T
                \end{equation}
            \STATE{} Update $[\bb{L}_{[t+1]}, \bb{C}_{[t+1]}]$ via a Least Squares Khatri-Rao factorization, as described in \cref{subsec:robust_sh_basis}
            \STATE{} Update Lagrange multipliers by
                \begin{equation}\label{eq:Y_update}
                    \bb{Y}_{[t+1]} = \Yt + \mut \left( \X - \bb{A}_{[t+1]} - \bb{E}_{[t+1]} \right)
                \end{equation}
            \STATE{} Update $\mu_{[t+1]}$ by $\mu_{[t+1]} = \min(\rho \mut, 10^6)$
            \STATE{} Check convergence condition
                \begin{equation}\label{eq:convergence_condition}
                    \begin{aligned}
                        &\lVert \X - \bb{A}_{[t+1]} - \bb{E}_{[t+1]} \rVert_{\infty} < \epsilon, \\
                        &\lVert \bb{A}_{[t+1]} - \bb{B}_{[t+1]} - (\bb{L}_{[t+1]} \ast \bb{C}_{[t+1]}) \rVert_{\infty} < \epsilon
                    \end{aligned}
                \end{equation}
            \STATE{} $t \leftarrow t + 1$
        \ENDWHILE{}
    \end{algorithmic}
}
\end{algorithm}
%%%%%%%%%%%%%%%%%%%%%%%%%%%%%%%%%%%%%%%%%%%%%%%%%%%%%%%%%%%%%%%%%%%%%%%%%%%%%%%%
%%%%%%%%%%%%%%%%%%%%%%%%%%%%%%%%%%%%%%%%%%%%%%%%%%%%%%%%%%%%%%%%%%%%%%%%%%%%%%%%
%%%%%%%%%%%%%%%%%%%%%%%%%%%%%%%%%%%%%%%%%%%%%%%%%%%%%%%%%%%%%%%%%%%%%%%%%%%%%%%%
\subsection{Efficient Pixelwise Correspondence}\label{subsec:imag_coll_correspondence}
%%%%%%%%%%%%%%%%%%%%%%%%%%%%%%%%%%%%%%%%%%%%%%%%%%%%%%%%%%%%%%%%%%%%%%%%%%%%%%%%
In contrast to the related work of \citet{KemelmacherShlizerman:2013iv},
we achieved pixelwise correspondence between our images by using existing,
efficient sparse facial alignment algorithms. This has two distinct advantages.
Firstly, recent facial alignment algorithms such as those by
\citet{ren2014face} and \citet{kazemi2014one} can produce a very accurate set
of sparse facial features in the order of a single millisecond. In contrast, the
optical flow method cited in \citet{KemelmacherShlizerman:2013iv} takes multiple seconds even for
a small image. This means that our training time is drastically reduced in
comparison to \citet{KemelmacherShlizerman:2013iv}. Ideally, our technique would be able to scale
to the magnitude of thousands of images, whereas the alignment of
\citet{KemelmacherShlizerman:2013iv} would quickly become infeasible as the number of images
increases. In fact, the optical flow step is run multiple times as the
collection flow algorithm is used~\cite{kemelmacher2012collection} which involves an
iterative algorithm of rank 4 decompositions and repeated optical flow.
Secondly, the use of a direct alignment to a single reference frame enables the
the usage of our basis in existing appearance based facial alignment algorithms
such as AAMs. This means that our basis can be used to reconstruct dense 3D
shape of faces directly from an existing AAM fitting provided the reference
space of the AAM and our subspace is the same.
%%%%%%%%%%%%%%%%%%%%%%%%%%%%%%%%%%%%%%%
\begin{figure*}
    \centering
    \includegraphics[width=4cm]{collection_ps/images/example_warped_original}
    \includegraphics[width=4cm]{collection_ps/images/example_warped}
    \includegraphics[width=4cm]{collection_ps/images/example_warped_low_rank}
    \includegraphics[width=3.8cm,height=3.8cm,trim = 0cm -1cm 0cm 0cm]{collection_ps/images/example_warped_shape}
    \caption{{\bf Example of the low rank effect on warped pose}. From left to
             right: initial input image, input image after warping,
             warped image after the low rank constraint, 
             recovered depth from warped image.}
\label{fig:imag_coll_low_rank_warping}
\end{figure*}
%%%%%%%%%%%%%%%%%%%%%%%%%%%%%%%%%%%%%%%

However, it is important to note that there are two potential drawbacks to our
alignment technique. The alignment is based on a Piecewise Affine warping and is
thus much coarser than the optical flow technique used in \citet{KemelmacherShlizerman:2013iv}.
This is particularly amplified when larger poses are present in the input
images. However, this is partly why the low rank component of our algorithm is
so important. As \cref{fig:imag_coll_low_rank_warping} shows, the robust
decomposition of the basis helps correct these large global errors so that the
shape subspace can be successfully recovered. Secondly, our technique does not
contain a number of sub-clusters that can be used to warp expression onto our
model. However, by using a large number of images that contain expression we
directly include expression within our subspace. In \citet{KemelmacherShlizerman:2013iv}, the
recovered subspace will necessarily be devoid of expression as the global
reference shape is neutral. This means that the subspace recovered by
\citet{KemelmacherShlizerman:2013iv} will not be able to recover expressive 3D shape using
efficient facial alignment algorithms.

%%%%%%%%%%%%%%%%%%%%%%%%%%%%%%%%%%%%%%%%%%%%%%%%%%%%%%%%%%%%%%%%%%%%%%%%%%%%%%%%
\section{Experiments}\label{sec:imag_coll_experiments}
%%%%%%%%%%%%%%%%%%%%%%%%%%%%%%%%%%%%%%%%%%%%%%%%%%%%%%%%%%%%%%%%%%%%%%%%%%%%%%%%
In this section we provide a number of experiments that demonstrate the
advantage of our coarse alignment and the single template recovered by our
coarse alignment scheme. We also provide experiments that emphasise the increased
robustness of our reconstructions over the least-squares solution of
\citet{KemelmacherShlizerman:2013iv}. We show a new application to this type
of model that involves improving the fitting results of an 
AAM~\cite{cootes2001active} using our
constructed SH basis. 

Choosing the number of components, $k$, to recover is an
important problem that was not properly addressed by \citet{KemelmacherShlizerman:2013iv}. 
In these experiments we attempt to recover as many
components as possible in order to strike a balance between cleanly
reconstructed normals and deformations. However, there is a trade-off when choosing
the value of $k$. In particular, if the value of $k$ is too large, then the
decomposition is unable to separate the deformation and illumination and the subspace of
illumination no longer represents valid normals. This is one of the primary advantages
of our robust decomposition, as it allows the value of $k$ to be larger given
the reduced rank of the images. However, a potential disadvantage of our
proposed method is the sensitivity of the algorithm to the parameter $\lambda$,
which must be tuned for every dataset. It is also important to stress that our
main goal is to recover the low frequency shape information to provide plausible
3D facial surfaces under challenging conditions. However, in
\cref{subsec:experiments_smith}, we show that our recovered subspace can
be used in existing high frequency recovery algorithms such as SFS.\@

The area of 3D facial surface recovery is lacking any form of formal
quantitative benchmark. The quantitative benchmark presented in
\cite{KemelmacherShlizerman:2013iv} is performed on depth data recovered from PS.\@
This is not ground truth depth data, as error is introduced during
integration, and a more accurate evaluation would be the angular error of the
recovered normals. However, in the presence of cast shadows, even the normals of
PS are biased. For this reason, the lack of a standard and fair
quantitative evaluation, we focus on qualitative results in this chapter.

Specifically we performed the following experiments:
%%%%%%%%%%%%%%%%%%%%%%%%%%%%%%%%%%%%%%%
\begin{itemize}
    \item We built our subspace using the HELEN~\cite{le2012interactive} 
          dataset. We directly compare against the least-squares decomposition
          proposed in~\cite{KemelmacherShlizerman:2013iv} and show particularly
          challenging images from the dataset. This experiment highlights the
          difficulty in constructing subspaces from large a set of
          ``in-the-wild'' images.
    \item Similar to \citet{KemelmacherShlizerman:2013iv}, we provide
          qualitative examples on shape recovery of the Yale B dataset.
    \item We show that the robust subspace learnt in (1) can be used within the
          SFS framework of \cref{sec:singl_img_gsfs}. By recovering the normals
          from every image of HELEN~\cite{le2012interactive}, we can perform a
          secondary PCA on the normals in order to directly embed them within
          the GSFS algorithm. In this experiment, we compare against a clean
          dataset of normals acquired from the 
          ICT-3DRFE~\cite{stratou2012exploring} database.
    \item We show how our subspace can be combined with an existing facial
    alignment algorithm, namely project-out AAMs~\cite{matthews2004active}. Our
    subspace can be used both as the appearance basis for the AAM and also as a
    methodology of recovering dense 3D shape.
\end{itemize}

In the following section we describe the construction of the bases and explain
what pre-processing was performed on each dataset.
%%%%%%%%%%%%%%%%%%%%%%%%%%%%%%%%%%%%%%%
\begin{table}
    \centering
    \begin{tabular}{lccccc}
        \toprule
        \textbf{Data}          & \textbf{Warp (s)} & \textbf{Decomposition (s)} & \textbf{Inverse Warp (s)} & \textbf{Total (s)} \\ \midrule
        HELEN (2330 images)    & 8                 & 730                        & 25                        & 763                \\
        Tom Hanks (274 images) & 1                 & 21                         & 4                         & 26                 \\  \bottomrule
    \end{tabular}
    \caption{\textbf{Training Times.} Mean training times in seconds over 10
             runs rounded to the nearest second. ``Warp'' denotes warping to the
             LFPW reference frame of $(150 \times 150)$ pixels, ``Inverse Warp''
             denotes warping back to the original images and ``Decomposition''
             denotes the total training time of our method described in
             \cref{subsec:imag_coll_robust_sh_basis}. Original images were larger than 
             the reference, hence the increase from ``Warp'' to ``Inverse Warp''.
             Timings were recorded on an Intel Xeon E5--1650 3.20GHz with
             32GB of RAM.}
\label{tbl:imag_coll_timings}
\end{table}
%%%%%%%%%%%%%%%%%%%%%%%%%%%%%%%%%%%%%%%
%%%%%%%%%%%%%%%%%%%%%%%%%%%%%%%%%%%%%%%%%%%%%%%%%%%%%%%%%%%%%%%%%%%%%%%%%%%%%%%%
\subsection{Constructing The Robust Bases}\label{subsec:imag_coll_construction}
%%%%%%%%%%%%%%%%%%%%%%%%%%%%%%%%%%%%%%%%%%%%%%%%%%%%%%%%%%%%%%%%%%%%%%%%%%%%%%%%
The process of building the robust SH basis was the same for all datasets
involved. Facial annotations consisting of 68 points were recovered through
various methods for each dataset. In the case of the 
HELEN~\cite{le2012interactive} database, the manual
annotations provided by the IBUG group were used~\cite{sagonas2013300,sagonas2013semi},
in the case of the Yale B, Photoface and ICT-3DRFE databases, manual annotations
were used and the ``in-the-wild'' images and video of Tom Hanks were automatically
annotated by the one millisecond facial alignment method of~\cite{kazemi2014one}
provided by the Dlib project~\cite{king2009dlib}.

These annotations were then warped via a Piecewise Affine transformation to a
mean reference shape that was built from all the faces, training and testing, of
the LFPW~\cite{belhumeur2013localizing} facial annotations provided by 
IBUG~\cite{sagonas2013300,sagonas2013semi}.
This provided the dense
correspondence required for performing matrix decompositions. To construct our
SH bases, we performed the algorithm as described in
\cref{subsec:imag_coll_robust_sh_basis} on the warped images. In order to provide
the example reconstructions, the reconstructed images were warped back into
their original shapes and then integrated using the method of 
\citet{frankot1988method}.

\cref{tbl:imag_coll_timings} gives examples of the training time taken for the 
``in-the-wild'' Tom Hanks images and the HELEN dataset. It is important to note that
part of the reason the training time is much lower for the Tom Hanks images is
that they have an inherently lower rank than the HELEN images as they are all of
the same individual. This greatly affects the convergence time and thus the
timings do not scale linearly.
%%%%%%%%%%%%%%%%%%%%%%%%%%%%%%%%%%%%%%%%
\newcommand{\yaleb}[1]
{
\includegraphics[width=3.5cm,height=4cm]{collection_ps/images/yaleb_results/yale_b_#1}                      & \hspace{1.5cm} 
\includegraphics[width=3.5cm,height=4cm]{collection_ps/images/yaleb_results/yale_b_#1_photometric}          & \hspace{1.5cm}
\includegraphics[width=3.5cm,height=4cm]{collection_ps/images/yaleb_results/yale_b_#1_photometric_low_rank}
}
\setlength{\tabcolsep}{1pt}
\begin{figure}
    \centering
    \begin{tabular}{ccc}
        Input & \hspace{1.5cm} Photometric Stereo & \hspace{1.5cm} Proposed  \vspace*{0.2cm} \\ 
        \yaleb{B01}                            \\
        \yaleb{B03}                            \\
        \yaleb{B06}                                                  
    \end{tabular}
    \caption{{Examples comparing against traditional Photometric Stereo}. 
             Images from the Yale B dataset~\cite{georghiades2001fromfew}. 
             First column is the input images, second column is traditional 
             PS result and the third column is the proposed algorithm.}
\label{fig:imag_coll_yale_b}
\end{figure}
\setlength{\tabcolsep}{6pt}
%%%%%%%%%%%%%%%%%%%%%%%%%%%%%%%%%%%%%%%%
%%%%%%%%%%%%%%%%%%%%%%%%%%%%%%%%%%%%%%%%%%%%%%%%%%%%%%%%%%%%%%%%
\subsection{Comparison Using HELEN}\label{subsec:imag_coll_experiments_helen}
%%%%%%%%%%%%%%%%%%%%%%%%%%%%%%%%%%%%%%%%%%%%%%%%%%%%%%%%%%%%%%%%
In this set of experiments we wished to convey two results: (1) that we are
capable of quickly constructing our basis on a large number of in-the-wild
images, (2) that the our robust formulation of the problem gives superior
performance to the blind decomposition used by \citet{KemelmacherShlizerman:2013iv}. 
In this experiment, $k = 200$ and the total number of components was thus $4k = 800$.
\cref{fig:imag_coll_helen_compare} shows the results from this experiment. As we can
clearly see, on challenging images the least-squares decomposition is unable to
separate the appearance from the illumination and thus the recovered normals are
unable to recover accurate shape.

\cref{fig:imag_coll_helen_compare_morphable_model} provides three examples
of reconstructions by commercial 3DMM~\cite{volker1999morphable} systems on
some of the more challenging frames. Note that the underlying shape bares little
resemblance to the perceived shape of the face and that the algorithms are
particularly insensitive to the age of the subject. In contrast, class-specific
PS contains no shape prior that limits the variability of the recovered shape.
%%%%%%%%%%%%%%%%%%%%%%%%%%%%%%%%%%%%%%%%%%%%%%%%%%%%%%%%%%%%%%%%
\subsection{Comparison Using Yale B}\label{subsec:imag_coll_experiments_yaleb}
%%%%%%%%%%%%%%%%%%%%%%%%%%%%%%%%%%%%%%%%%%%%%%%%%%%%%%%%%%%%%%%%
In this experiment we compare the result of our proposed algorithm to 
that of traditional Photometric Stereo. In this scenario, the images are assumed
to be aligned and we do not perform any pre-processing of the images. Only the
inner facial region is considered for recovery in order to reduce noise present
in the background of the image. \cref{fig:imag_coll_yale_b} shows examples
of the results. Our proposed method is smoother than that of the least-squares
PS result as the low-rank constraint naturally removes sparse noise and
high-frequency details. For this experiment $k = 1$ as their is only a single
identity and expression present in the input images. Note that our method
shares similarities to that of \citet{wu2010robust} in this scenario as it
amounts to a robust U-PS algorithm. However, \citet{wu2010robust} do not consider
a spherical harmonic image formation model and so our proposed algorithm is
still novel. Finally, we do not directly compare against
\citet{KemelmacherShlizerman:2013iv} in this experiment as, assuming $k = 1$,
the factorisation performed is exactly that of U-PS~\cite{basri2007photometric}.
%%%%%%%%%%%%%%%%%%%%%%%%%%%%%%%%%%%%%%%%%%%%%%%%%%%%%%%%%%%%%%%%
\subsection{Using The Subspace In SFS}\label{subsec:imag_coll_experiments_smith}
%%%%%%%%%%%%%%%%%%%%%%%%%%%%%%%%%%%%%%%%%%%%%%%%%%%%%%%%%%%%%%%%
The SFS technique of \citet{smith2006recovering}, as described in
\cref{sec:singl_img_gsfs}, relies on a PCA basis
constructed from normals of a single class of object. It then seeks to recover
the high frequency normal information directly from the texture. In order to
create the PCA required by GSFS, we recovered spherical harmonics
for every image in the dataset using the proposed algorithm. We then computed
Kernel-PCA as described in \cref{subsubsec:singl_img_ca_kpca} on the normals
recovered from the HELEN images and supplied them to the GSGS framework.
The lighting vector is also an input to the algorithm and we estimate it by
solving a least-squares problem with the known mean normals.

In order to provide a comparison for our reconstruction, we created a clean
normal subspace using the data from the ICT-3DRFE~\cite{stratou2012exploring}
database. This database is primarily use for image relighting purposes, however,
they provide a very accurate set of normals of faces under a wide range of
expressions. The results of this experiment are shown in \cref{fig:imag_coll_smith}.
Although our subspace did not provide reconstructions that are as visually
accurate as the subspace from ICT-3DRFE, they were still able to successfully
recover a plausible representation of the high frequency shading information.
This demonstrates that our recovered SH subspace is yielding plausible surface
normals that are capable of constraining SFS within the space of human faces.
%%%%%%%%%%%%%%%%%%%%%%%%%%%%%%%%%%%%%%%
\begin{figure}[t]
    \centering
    \includegraphics[width=4cm,height=4.8cm]{collection_ps/images/smith/samuel_beckett}                    \hspace{0.3cm}
    \includegraphics[width=4cm,height=4.8cm]{collection_ps/images/smith/beckett_smith_frontal_ict}         \hspace{0.3cm}
    \includegraphics[width=4cm,height=4.8cm]{collection_ps/images/smith/beckett_smith_frontal_low_rank}    \\
    \includegraphics[width=4cm,height=4.8cm]{collection_ps/images/smith/jude_law}                          \hspace{0.3cm}
    \includegraphics[width=4cm,height=4.8cm]{collection_ps/images/smith/law_smith_frontal_ict}             \hspace{0.3cm}
    \includegraphics[width=4cm,height=4.8cm]{collection_ps/images/smith/law_smith_frontal_low_rank}
    \caption{{\bf Our subspace used for SFS}. Normals learnt automatically from 
             the Spherical Harmonic subspace of HELEN~\cite{le2012interactive} vs
             normals from the clean data of ICT-3DRFE~\cite{stratou2012exploring}.
             Middle column: the clean data. 
             Right column: proposed subspace.}
\label{fig:imag_coll_smith}
\end{figure}
%%%%%%%%%%%%%%%%%%%%%%%%%%%%%%%%%%%%%%%
%%%%%%%%%%%%%%%%%%%%%%%%%%%%%%%%%%%%%%%%%%%%%%%%%%%%%%%%%%%%%%%%
\subsection{Automatic Alignment}\label{subsec:experiments_alignment}
%%%%%%%%%%%%%%%%%%%%%%%%%%%%%%%%%%%%%%%%%%%%%%%%%%%%%%%%%%%%%%%%
%%%%%%%%%%%%%%%%%%%%%%%%%%%%%%%%%%%%%%%
\newcommand{\tomhanksalignment}[1]
{
\includegraphics[width=3.5cm]{collection_ps/images/tom_hanks/tom_hanks_improve_#1_initial} &
\includegraphics[width=3.5cm]{collection_ps/images/tom_hanks/tom_hanks_improve_#1_final}   & \hspace{0.2cm}
\includegraphics[width=3.5cm,height=3.3cm]{collection_ps/images/tom_hanks/tom_hanks_improve_#1_depth}
}
%%%%%%%%%%%%%%%%%%%%%%%%%%%%%%%%%%%%%%%
%%%%%%%%%%%%%%%%%%%%%%%%%%%%%%%%%%%%%%%
\setlength{\tabcolsep}{1pt}
\begin{figure}
    \centering
    \begin{subfigure}[b]{0.65\textwidth}
        \centering
        \begin{tabular}{ccc}
            Initial & Final & \hspace{0.2cm} Recovered Depth \\
            \tomhanksalignment{1}                            \\
            \tomhanksalignment{129}
        \end{tabular}
        \caption{}
\label{subfig:imag_coll_improve_tom_hanks_recovered}
    \end{subfigure}
    \hspace{0.1cm} \vrule \hspace{0.1cm}
    \begin{subfigure}[b]{0.25\textwidth}
        \centering
        \begin{tabular}{c}
            Improvement \\
            \includegraphics[width=3.5cm]{collection_ps/images/tom_hanks/tom_hanks_improve_0_initial} \\
            \includegraphics[width=3.5cm]{collection_ps/images/tom_hanks/tom_hanks_improve_10_final}
        \end{tabular}
        \caption{}
\label{subfig:imag_coll_improve_tom_hanks_improve}
    \end{subfigure}
    \caption{{\bf Person specific model fitting for Tom Hanks.} Images of Tom 
             Hanks coarsely aligned by a facial alignment method. Our algorithm 
             improves the facial alignment and simultaneously recovers depth. 
             Images shown are from a YouTube video of Tom Hanks. 
             \cref{subfig:imag_coll_improve_tom_hanks_recovered} shows the
             recovered depth and improved landmarks from the person specific
             fitting. 
             \cref{subfig:imag_coll_improve_tom_hanks_improve} shows the
             improvement to the mean of all frames after the first iteration
             (top row) and after the final (10th) iteration (bottom row).}
\label{fig:imag_coll_improve_tom_hanks}
\end{figure}
\setlength{\tabcolsep}{6pt}
%%%%%%%%%%%%%%%%%%%%%%%%%%%%%%%%%%%%%%%
%%%%%%%%%%%%%%%%%%%%%%%%%%%%%%%%%%%%%
\newcommand{\tomhanks}[1]
{
\includegraphics[width=2.7cm,height=3.2cm]{collection_ps/images/tom_hanks/tom_hanks_#1}                   &
\includegraphics[width=2.7cm,height=3.2cm]{collection_ps/images/tom_hanks/tom_hanks_#1_low_rank}          & 
\includegraphics[width=2.7cm,height=3.2cm]{collection_ps/images/tom_hanks/tom_hanks_#1_low_rank_textured}
}
%%%%%%%%%%%%%%%%%%%%%%%%%%%%%%%%%%%%%
%%%%%%%%%%%%%%%%%%%%%%%%%%%%%%%%%%%%%
\setlength{\tabcolsep}{1pt}
\begin{figure}
    \centering
    \begin{tabular}{cccccc}
        Input & Shape & Textured & Input & Shape & Textured \vspace*{0.2cm} \\ 
        \tomhanks{14}            & \tomhanks{27}                            \\
        \tomhanks{95}            & \tomhanks{52}
    \end{tabular}
    \caption{{\bf Example reconstructions of Tom Hanks}. Images automatically 
             downloaded from Google Images. First column is the input images, 
             second column is the untextured shape from the proposed technique 
             and the third column is the textured shape.}
\label{fig:imag_coll_tom_hanks}
\end{figure}
\setlength{\tabcolsep}{6pt}
%%%%%%%%%%%%%%%%%%%%%%%%%%%%%%%%%%%%%
In this experiment we use a variation of the traditional AAM~\cite{cootes2001active}
algorithm that is equivalent to the project-out algorithm as proposed by
\citet{matthews2004active}. Practically, we
used the Active Template Model (ATM) provided by the Menpo
project~\cite{menpo14} in order to perform a project-out type algorithm to align
images of Tom Hanks. This model is similar to the Lucas-Kanade
\cite{lucas1981iterative} method but uses a point distribution model (PDM) in order to
perform non-rigid alignment between the images. In particular, the template
image is fixed during optimisation of the PDM, and we use our subspace to
provide a texture representing an approximation of the diffuse component of the
image. This is essentially identical to the procedure performed within a
project-out AAM.\@

We used a person specific SH subspace that was built on images of Tom Hanks that
were downloaded automatically from the Internet. In this case, the images were
automatically aligned using the DLib implementation of~\cite{kazemi2014one}. For
this experiment, $k = 30$ and thus the total number of components $4k = 120$.
Example reconstructions of this person specific SH basis are given in
\cref{fig:imag_coll_tom_hanks} where the output shapes bear a strong resemblance
to Tom Hanks despite the mixture of ages present in the images.

We downloaded 200 frames from a Youtube video of Tom
Hanks\footnote{\url{https://www.youtube.com/watch?v=nFvASiMTDz0} from 3:43} and
attempted to automatically align them using our subspace and the ATM.\@ The ATM
was initialised using the fitting of~\cite{kazemi2014one} and was then
iteratively improved. At each global iteration, we recovered a new set of
diffuse textures for each frame and then performed a refitting of every frame.
This caused the images to align over a sequence of iterations. We performed 10
such iterations. \cref{subfig:imag_coll_improve_tom_hanks_recovered} shows two
example frames where the alignment was improved and dense shape was also
recovered. \cref{subfig:imag_coll_improve_tom_hanks_improve} shows the
improvement in alignment over the global iterations. The final iteration is much
sharper than the initial iteration which demonstrates that the correspondence
across all frames has been visibly improved.
%%%%%%%%%%%%%%%%%%%%%%%%%%%%%%%%%%%%%%%%
\newcommand{\comparemm}[1]
{
\includegraphics[width=3cm,height=3.5cm]{collection_ps/images/helen/helen_#1}                 & \hspace{0.2cm}
\includegraphics[width=3cm,height=3.5cm]{collection_ps/images/helen/helen_#1_frontal_vizago}  & \hspace{0.2cm}
\includegraphics[width=3cm,height=3.5cm]{collection_ps/images/helen/helen_#1_side_vizago}     & \hspace{0.2cm}
\includegraphics[width=3cm,height=3.5cm]{collection_ps/images/helen/helen_#1_frontal_facegen} & \hspace{0.2cm}
\includegraphics[width=3cm,height=3.5cm]{collection_ps/images/helen/helen_#1_side_facegen}
}
\setlength{\tabcolsep}{1pt}
\begin{figure*}
    \centering
    \begin{tabular}{ccccc} \vspace*{0.2cm}
        Input & \multicolumn{2}{c}{vizago.ch} & \multicolumn{2}{c}{facegen.com}  \\
        \comparemm{6}                                                            \\
        \comparemm{680}                                                          \\
        \comparemm{821}                  
    \end{tabular}
    \caption{{\bf Examples of difficult reconstructions for existing commercial 
              Morphable Model implementations}. 
             Images from the HELEN~\cite{le2012interactive} dataset. 
             Both implementations do not provide the ability to render 
             textureless surfaces.
             Facegen.com reported invalid landmarks for the image given in the
             first row.}
\label{fig:imag_coll_helen_compare_morphable_model}
\end{figure*}
\setlength{\tabcolsep}{6pt}
%%%%%%%%%%%%%%%%%%%%%%%%%%%%%%%%%%%%%%%%
%%%%%%%%%%%%%%%%%%%%%%%%%%%%%%%%%%%%%%%%
\newcommand{\comparehelen}[2]
{
\includegraphics[width=2.5cm,height=2.5cm]{collection_ps/images/helen/helen_#1}                  & \hspace{0.5cm}
\includegraphics[width=2.3cm,height=2.5cm]{collection_ps/images/helen/helen_#1_frontal_low_rank} & \hspace{0.5cm}
\includegraphics[width=2.3cm,height=2.5cm]{collection_ps/images/helen/helen_#1_frontal_ira}      & \hspace{0.5cm}
\includegraphics[width=3cm,height=2.5cm]{collection_ps/images/helen/helen_#1_#2_low_rank}        & \hspace{0.5cm}
\includegraphics[width=3cm,height=2.5cm]{collection_ps/images/helen/helen_#1_#2_ira}
}
\begin{landscape}
\thispagestyle{footeronly}
\setlength{\tabcolsep}{1pt}
\begin{figure*}
    \centering
    \begin{tabular}{ccccc} \vspace*{0.2cm}
        Input & \hspace{0.5cm} Proposed & \hspace{0.5cm}~\cite{KemelmacherShlizerman:2013iv} & \hspace{0.5cm} Proposed & \hspace{0.5cm}~\cite{KemelmacherShlizerman:2013iv} \\
        \vspace*{-0.1cm}
        \comparehelen{1348}{side} \\ \vspace*{-0.07cm}
        \comparehelen{555}{side}  \\ \vspace*{-0.07cm}
        \comparehelen{680}{chin}  \\ \vspace*{-0.07cm}
        \comparehelen{6}{chin}    \\ \vspace*{-0.07cm}
        \comparehelen{821}{side}  \\ \vspace*{-0.07cm}
        \comparehelen{77}{side}                                      
    \end{tabular}
    \caption{{\bf Comparison with the least-squares decomposition 
             of~\cite{KemelmacherShlizerman:2013iv}}.
             Images from the HELEN~\cite{le2012interactive} dataset.}
\label{fig:imag_coll_helen_compare}
\end{figure*}
\setlength{\tabcolsep}{6pt}
\end{landscape}
%%%%%%%%%%%%%%%%%%%%%%%%%%%%%%%%%%%%%%%%

%%%%%%%%%%%%%%%%%%%%%%%%%%%%%%%%%%%%%%%%%%%%%%%%%%%%%%%%%%%%%%%%%%%%%%%%%%%%%%%%
\stopcontents[chapters]
%%%%%%%%%%%%%%%%%%%%%%%%%%%%%%%%%%%%%%%%%%%%%%%%%%%%%%%%%%%%%%%%%%%%%%%%%%%%%%%%
