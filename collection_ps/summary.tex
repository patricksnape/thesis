%%%%%%%%%%%%%%%%%%%%%%%%%%%%%%%%%%%%%%%%%%%%%%%%%%%%%%%%%%%%%%%%%%%%%%%%%%%%%%%%
\section{Summary and Conclusion}\label{sec:imag_coll_summary}
%%%%%%%%%%%%%%%%%%%%%%%%%%%%%%%%%%%%%%%%%%%%%%%%%%%%%%%%%%%%%%%%%%%%%%%%%%%%%%%%
In this chapter we investigated 3D facial shape recovery on collections of
``in-the-wild'' images. This is a challenging scenario as no knowledge of the
lighting conditions, the facial location or the camera geometric properties is
provided. However, by leveraging the large number of images and the inherent
low-rank properties of the human face we have shown that plausible 3D shape
can be recovered without any 3D prior. We have provided an optimisation
framework for the recovery and separation of both the illumination and
deformation parameters of a spherical harmonic subspace. Most importantly,
this optimisation framework contains low-rank constraints that mitigate the
effect of common sparse image outliers such as occlusions. We also demonstrated
that a global and sparse alignment is sufficient for recovering plausible shape
without the need for expensive per pixel alignment such as optical flow.

We have evaluated our algorithm qualitatively on it's ability to recover
plausible facial shape on the ``in-the-wild'' HELEN~\cite{le2012interactive}
and demonstrate robustness to occlusions. This is particularly evident when
qualitatively compared to the least-squares solution as originally proposed by
\citet{KemelmacherShlizerman:2013iv}. We have also shown that our algorithm,
due to the single global template space, is applicable for 
GSFS (\cref{sec:singl_img_gsfs}) and iterative person specific image alignment
and depth recovery. Unfortunately, there lacks a standard benchmark for
3D surface recover from images using image formation techniques. Thus we provide
relatively few quantitative experiments and rely on the perception of our
recovered shape as is common in other works in the
area~\cite{KemelmacherShlizerman:2013iv,kemelmacher2011face,kemelmacher2012collection}.

Although the recovery of plausible normals directly from images, without any
3D prior, is an attractive problem, it still does not recover shapes that
are visually comparable to techniques that employ an explicit 3D prior. Therefore,
in the next chapter we investigate another form of the image collection problem,
where temporal consistency is assumed. We formulate the problem as one of alignment
and use an explicit 3D model in order to recover 3D shape implicitly as part
of the alignment output.
