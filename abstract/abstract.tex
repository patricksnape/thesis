%%%%%%%%%%%%%%%%%%%%%%%%%%%%%%%%%%%%%%%%%%%%%%%%%%%%%%%%%%%%%%%%%%%%%%%%%%%%%%%%
\begin{abstract}
%%%%%%%%%%%%%%%%%%%%%%%%%%%%%%%%%%%%%%%%%%%%%%%%%%%%%%%%%%%%%%%%%%%%%%%%%%%%%%%%
Human faces are one of the most frequently captured objects in both videos
and photographs due to their fundamental role in communication and social
interactions.
The variability of this facial imagery makes it difficult to automate the understanding
of scenes containing faces under unconstrained conditions.
For faces, recovering accurate
dense 3D facial shape from images and videos enables much richer understanding
of the human face and its interaction with the scene. In this thesis,
we seek to extend the work in the area of dense 3D facial shape recovery
under challenging unconstrained conditions. There are a wealth of ways
to recover 3D shape from images and videos, all of which make specific
assumptions about the relationship between the individual images and the
construction of the scene. Given this broad selection of methods available, we
examine three different scenarios
for dense 3D facial shape recovery: i) recovery from a single image, ii) recovery
from an unconstrained image collection without any explicit 3D shape priors
and iii) recovery from a video sequence. We focus on these three cases and show how
facial priors can be introduced to tackle the dense 3D facial surface
recovery problem. We propose to investigate the use of shading
constraints for dense shape recovery from unconstrained images. Given
the challenging nature of these images, the introduction of priors
greatly improves performance over the generic shape-from-shading literature.
However, the introduction of explicit priors comes with a further problem,
that of correspondence. That is, recovering the relationship between pixels
in the image and the structure of our model. For this reason, we also
investigate the importance of finding dense correspondences between facial
images. We show that it is possible to recover plausible dense 3D facial
surfaces under a variety of different input conditions.
%%%%%%%%%%%%%%%%%%%%%%%%%%%%%%%%%%%%%%%%%%%%%%%%%%%%%%%%%%%%%%%%%%%%%%%%%%%%%%%%
\end{abstract}
%%%%%%%%%%%%%%%%%%%%%%%%%%%%%%%%%%%%%%%%%%%%%%%%%%%%%%%%%%%%%%%%%%%%%%%%%%%%%%%%
