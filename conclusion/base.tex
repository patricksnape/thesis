%%%%%%%%%%%%%%%%%%%%%%%%%%%%%%%%%%%%%%%%%%%%%%%%%%%%%%%%%%%%%%%%%%%%%%%%%%%%%%%%
\chapter{Summary and Conclusion}\label{ch:conclusions}
%%%%%%%%%%%%%%%%%%%%%%%%%%%%%%%%%%%%%%%%%%%%%%%%%%%%%%%%%%%%%%%%%%%%%%%%%%%%%%%%
\minitoc{}
%%%%%%%%%%%%%%%%%%%%%%%%%%%%%%%%%%%%%%%%%%%%%%%%%%%%%%%%%%%%%%%%%%%%%%%%%%%%%%%%
\section{Thesis Summary}
%%%%%%%%%%%%%%%%%%%%%%%%%%%%%%%%%%%%%%%%%%%%%%%%%%%%%%%%%%%%%%%%%%%%%%%%%%%%%%%%
We have presented here our work on the problem of applying Computer Vision
for recovering dense 3D facial surfaces from images. The larger area of
3D recovery from images is arguably the most fundamental goal of Computer
Vision. Thus, 3D facial surface recovery inherits many of the difficulties
and ambiguities that are present in this problem. For this reason, the
breadth of the applicable literature is large and there are many possible
strategies for recovering 3D shape, much of which is determined by the constraints
placed on the input data. Therefore, in this thesis we investigated three 
distinct but complementary scenarios for dense 3D facial surface recovery.
These scenarios are primarily determined by the type of input data provided
for solving the problem. As discussed in \cref{ch:background} it can be loosely
specified that 3D recovery relies on either constraints on the assumption of the
image formation problem or prior knowledge on the correspondences of certain
functions of the scene. One possible image formation assumption is to assume
a particular interaction between the lighting present in the scene and it's
interaction with the objects in the scene and the image capturing equipment.
Using these cues is typically referred to as using shading constraints and
it is these type of constraints that we focus on in this thesis. Complementary
to this is the concept of correspondences within the scene. These correspondences
are important to disambiguate properties of the scene and also to enforce priors
on the formation of the scene. In this thesis, our strongest prior is that
the images we are interested in contain faces. To this end we propose
to use both shading and correspondence techniques for recovering 3D facial
surfaces under the following three scenarios:


\textbf{Single Image: Investigating statistical models of surface normals.}
In the most challenging case, that of a single
input image of a face, we propose to solve the problem by employing shading
constraints. Shape-from-shading and it's related literature is one of the
few methods capable of solving the single image shape recovery problem. However,
Shape-from-shading alone is still incapable of recovering plausible facial shapes
and therefore we proposed to incorporate a statistical model of facial surface
normals. We show that these statistical models are useful not only for dense
3D facial surface recovery, but also as a tool for gaining correspondences
between the model and input data. We investigated two data sources: 2.5D
depth data and volumetric 3D data in the form of medical images. We showed
that a statistical model of surface normals provides a robust subspace that
is unaffected by gross outliers in the input data and thus particularly suited
to the alignment of images that contain occlusions.

\textbf{Image Collections: Recovering dense 3D without an explicit 3D prior.}
Here we relax the single image constraint and propose to use shading information
to jointly recover surface normals from an unconstrained collection of images.
Here we assumed that correspondences were provided for the input image but
an explicit 3D shape prior was not. Our robust decomposition, inspired by
uncalibrated photometric stereo~\cite{basri2007photometric}, was shown to
be effective for ``in-the-wild'' images. These images contain a variety of 
occlusions and lighting conditions that violate the underlying spherical
harmonic image formation assumed. However, by incorporating a robust low-rank
constraint on the decomposition we demonstrated that plausible 3D surfaces
may still be recovered from ``in-the-wild'' images even in the presence
of these outliers.

\textbf{Image sequences: The problem of correspondences.}
In the final chapter we note that although shading constraints provide an
attractive method of 3D surface recovery, their practical reliance on
priors in the form of statistical models asserts a strong constraint: the 
existence of correspondences between the inputs and the model. Even when the
model is constructed in a data driven manner, as in \cref{ch:imag_coll},
these correspondences are still assumed to pre-exist. Therefore, we investigate
how dense 2D correspondences may be recovered under the prior that the images
contain a face. We construct a dense 2D deformation basis and show how this can
be used to construct an efficient algorithm for dense 2D face alignment. 
Furthermore, we focus on image sequences which are more challenging than the
commonly investigated two-frame scenario assumed in the optical flow literature.
To this end, we further extend our model based optical flow with a low-rank
constraint that acts as a weak temporal regulariser and is experimentally shown
to outperform state-of-the-art optical flow methods for challenging synthetic
sequences. These 2D correspondences are also appropriate for 3D surface recovery
under the assumption that the 2D landmarks are in correspondence with an
existing 3D statistical model. Under this assumption we show that plausible
3D shape may be recovered and that this shape appears more consistent than
the shape recovered when using only sparse 2D correspondences.
%%%%%%%%%%%%%%%%%%%%%%%%%%%%%%%%%%%%%%%%%%%%%%%%%%%%%%%%%%%%%%%%%%%%%%%%%%%%%%%%
\section{Future Work}
%%%%%%%%%%%%%%%%%%%%%%%%%%%%%%%%%%%%%%%%%%%%%%%%%%%%%%%%%%%%%%%%%%%%%%%%%%%%%%%%
Given the distinct scenarios investigated in this thesis, we consider
the further work that might be undertaken per chapter separately. However,
one specific area that all facial recovery methods stand to benefit from
is that of ground truth data collection. Without the existence of a dataset
for quantitatively evaluating 3D facial surface recovery \textit{under
challenging ``in-the-wild'' conditions under a variety of poses and expressions}
it remains difficult to compare and contrast the current state-of-the-art.
It is our firm belief that the contribution of a dataset for reasonable
ground truth facial surfaces, and ideally known lighting conditions, would
be of enormous benefit to the research community.


%%%%%%%%%%%%%%%%%%%%%%%%%%%%%%%%%%%%%%%%%%%%%%%%%%%%%%%%%%%%%%%%%%%%%%%%%%%%%%%%
\section{Final Words}
%%%%%%%%%%%%%%%%%%%%%%%%%%%%%%%%%%%%%%%%%%%%%%%%%%%%%%%%%%%%%%%%%%%%%%%%%%%%%%%%
The problem of 3D surface recover remains an extremely challenging one. Even
when assuming that the object being recovered is only from a single class,
such as faces, it is still difficult to recover accurate shape. This is in
part due to the technical ambiguities present in the problem and in part
due to the familiarity of humans with human faces. The recovered surfaces must
be highly accurate to convince the viewer that they were indeed plausible
reconstructions of an image. However, the work presented here has shown that
we are approaching a point where it may be possible to recover useful facial
surfaces in a variety of scenarios. We hope that work continues in these
directions and that in the future facial surface recovery becomes a technology
rather than a research problem.
%%%%%%%%%%%%%%%%%%%%%%%%%%%%%%%%%%%%%%%%%%%%%%%%%%%%%%%%%%%%%%%%%%%%%%%%%%%%%%%%
\stopcontents[chapters]
%%%%%%%%%%%%%%%%%%%%%%%%%%%%%%%%%%%%%%%%%%%%%%%%%%%%%%%%%%%%%%%%%%%%%%%%%%%%%%%%