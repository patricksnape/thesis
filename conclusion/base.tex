%%%%%%%%%%%%%%%%%%%%%%%%%%%%%%%%%%%%%%%%%%%%%%%%%%%%%%%%%%%%%%%%%%%%%%%%%%%%%%%%
\chapter{Summary and Conclusion}\label{ch:conclusions}
%%%%%%%%%%%%%%%%%%%%%%%%%%%%%%%%%%%%%%%%%%%%%%%%%%%%%%%%%%%%%%%%%%%%%%%%%%%%%%%%
\minitoc{}
%%%%%%%%%%%%%%%%%%%%%%%%%%%%%%%%%%%%%%%%%%%%%%%%%%%%%%%%%%%%%%%%%%%%%%%%%%%%%%%%
\section{Thesis Summary}
%%%%%%%%%%%%%%%%%%%%%%%%%%%%%%%%%%%%%%%%%%%%%%%%%%%%%%%%%%%%%%%%%%%%%%%%%%%%%%%%
We have presented here our work on the problem of applying Computer Vision
for recovering dense 3D facial surfaces from images. The larger area of
3D recovery from images is arguably the most fundamental goal of Computer
Vision. Thus, 3D facial surface recovery inherits many of the difficulties
and ambiguities that are present in this problem. For this reason, the
breadth of the applicable literature is large and there are many possible
strategies for recovering 3D shape, much of which is determined by the constraints
placed on the input data. Therefore, in this thesis we investigated three 
distinct but complementary scenarios for dense 3D facial surface recovery.
These scenarios are primarily determined by the type of input data provided
for solving the problem. As discussed in \cref{ch:background} it can be loosely
specified that 3D recovery relies on either constraints on the assumption of the
image formation problem or prior knowledge on the correspondences of certain
functions of the scene. One possible image formation assumption is to assume
a particular interaction between the lighting present in the scene and it's
interaction with the objects in the scene and the image capturing equipment.
Using these cues is typically referred to as using shading constraints and
it is these type of constraints that we focus on in this thesis. Complementary
to this is the concept of correspondences within the scene. These correspondences
are important to disambiguate properties of the scene and also to enforce priors
on the formation of the scene. In this thesis, our strongest prior is that
the images we are interested in contain faces. To this end we propose
to use both shading and correspondence techniques for recovering 3D facial
surfaces under the following three scenarios:

\textbf{Single Image: Investigating statistical models of surface normals.}
In the most challenging case, that of a single
input image of a face, we propose to solve the problem by employing shading
constraints. Shape-from-shading and it's related literature is one of the
few methods capable of solving the single image shape recovery problem. However,
Shape-from-shading alone is still incapable of recovering plausible facial shapes
and therefore we proposed to incorporate a statistical model of facial surface
normals. We show that these statistical models are useful not only for dense
3D facial surface recovery, but also as a tool for gaining correspondences
between the model and input data. We investigated two data sources: 2.5D
depth data and volumetric 3D data in the form of medical images. We showed
that a statistical model of surface normals provides a robust subspace that
is unaffected by gross outliers in the input data and thus particularly suited
to the alignment of images that contain occlusions.

\textbf{Image Collections: Recovering dense 3D without an explicit 3D prior.}
Here we relax the single image constraint and propose to use shading information
to jointly recover surface normals from an unconstrained collection of images.
Here we assumed that correspondences were provided for the input image but
an explicit 3D shape prior was not. Our robust decomposition, inspired by
uncalibrated photometric stereo~\cite{basri2007photometric}, was shown to
be effective for ``in-the-wild'' images. These images contain a variety of 
occlusions and lighting conditions that violate the underlying spherical
harmonic image formation assumed. However, by incorporating a robust low-rank
constraint on the decomposition we demonstrated that plausible 3D surfaces
may still be recovered from ``in-the-wild'' images even in the presence
of these outliers.

\textbf{Image sequences: The problem of correspondences.}
In the final chapter we note that although shading constraints provide an
attractive method of 3D surface recovery, their practical reliance on
priors in the form of statistical models asserts a strong constraint: the 
existence of correspondences between the inputs and the model. Even when the
model is constructed in a data driven manner, as in \cref{ch:imag_coll},
these correspondences are still assumed to pre-exist. Therefore, we investigate
how dense 2D correspondences may be recovered under the prior that the images
contain a face. We construct a dense 2D deformation basis and show how this can
be used to construct an efficient algorithm for dense 2D face alignment. 
Furthermore, we focus on image sequences which are more challenging than the
commonly investigated two-frame scenario assumed in the optical flow literature.
To this end, we further extend our model based optical flow with a low-rank
constraint that acts as a weak temporal regulariser and is experimentally shown
to outperform state-of-the-art optical flow methods for challenging synthetic
sequences. These 2D correspondences are also appropriate for 3D surface recovery
under the assumption that the 2D landmarks are in correspondence with an
existing 3D statistical model. Under this assumption we show that plausible
3D shape may be recovered and that this shape appears more consistent than
the shape recovered when using only sparse 2D correspondences.
%%%%%%%%%%%%%%%%%%%%%%%%%%%%%%%%%%%%%%%%%%%%%%%%%%%%%%%%%%%%%%%%%%%%%%%%%%%%%%%%
\section{Future Work}
%%%%%%%%%%%%%%%%%%%%%%%%%%%%%%%%%%%%%%%%%%%%%%%%%%%%%%%%%%%%%%%%%%%%%%%%%%%%%%%%
Given the distinct scenarios investigated in this thesis, we consider
the further work that might be undertaken per chapter separately. However,
one specific area that all facial recovery methods stand to benefit from
is that of ground truth data collection. Without the existence of a dataset
for quantitatively evaluating 3D facial surface recovery \textit{under
challenging ``in-the-wild'' conditions under a variety of poses and expressions}
it remains difficult to compare and contrast the current state-of-the-art.
It is our firm belief that the contribution of a dataset for reasonable
ground truth facial surfaces, and ideally known lighting conditions, would
be of enormous benefit to the research community. It is this lack of an existing
ground truth ``in-the-wild'' dataset that caused our focus on qualitative
results for ``in-the-wild'' inputs or synthetic data in order to construct 
plausible datasets.
%%%%%%%%%%%%%%%%%%%%%%%%%%%%%%%%%%%%%%%%%%%%%%%%%%%%%%%%%%%%%%%%%%%%%%%%%%%%%%%%
\subsection{Statistical Models of Surface Normals}
%%%%%%%%%%%%%%%%%%%%%%%%%%%%%%%%%%%%%%%%%%%%%%%%%%%%%%%%%%%%%%%%%%%%%%%%%%%%%%%%
\textbf{Further Kernels.} In \cref{ch:singl_imag} we proposed to view the 
construction of statistical models of surface normals as a Kernel-PCA problem.
Although directional statistics, such as surface normals, are non-euclidean,
Kernel-PCA provides a well defined method of performing component analysis
between non-linear spaces under the assumption that an inner product, or kernel,
may be performed. Although we derived two explicit Hilbert spaces for normals,
we did not investigate any other common non-linear kernels such as radial
basis functions or polynomial kernels. It may be particularly interesting
to investigate basis functions for surface normals and their relation to
the spherical harmonic bases used in the explicit Shape-from-Shading
literature.

\textbf{Extending Geometric Shape-from-Shading.} 
The geometric Shape-from-Shading (GSFS) algorithm used in this work, as proposed
by \citet{worthington1999new} and extended to use statistical models by
\citet{smith2010estimating}, was shown to provide a solid foundation for
single image surface recovery. However, in this work we made no effort to extend
this algorithm. For example, in \citet{smith2008facial} the GSFS algorithm
is extended to include robust statistics in order to account for pixels
that violate the Lambertian assumption made by the GSFS algorithm. An interesting
extension to this work would be to attempt to incorporate other constraints on
the output of the model in order to guide the results. For example, there is
always a trade-off between the high frequency detail expressed by the individual
normals as described by the shading constraint and the smooth surface predicted
by the statistical model. It would be interesting to investigate methods
of balancing the contribution of the two terms. Similarly, it is common in
``in-the-wild'' images for occlusions to be present. In this case, it would be
optimal to allow the model to complete these areas and disregard the contribution
from the shading constraints. This is also true of areas that grossly deviate
from the Lambertian assumption. Although this is partly addressed in 
\citet{smith2008facial}, hard occlusions such as hand-over-face gestures
are not investigated.

Given our lack of extension of the GSFS algorithm, we also suffer from the
same drawbacks discussed in \citet{smith2010estimating} and 
\citet{smith2008facial}, namely the constant albedo assumption, dependency
on the viewpoint, requirement of known lighting directions and the 
hard requirement of existing correspondences. Since
we propose a simpler method of building statistical normal models, it remains
further work to attempt to incorporate a varying albedo model into the GSFS
algorithm. 

\textbf{Alignment Using Normals.} In this work we provided evidence that
statistical models of surface normals inherit the robust properties of other
angular measures of similarity, such as the robust cosine metric
proposed by \citet{tzimiropoulos2012subspace}. However, in the case of 2.5D
data the normals were used simply as an image descriptor and used for
alignment in a manner similar to \citet{antonakos2015feature}. Directly
incorporating the 2.5D normals into the cost function is left as future work.
In the case of 3D normals, although an explicit robust cost function was
formulated, the use case of the work was not fully investigated. Currently
there is very little work in using Lucas-Kanade for 3D alignment of medical
images and we believe this is an interesting area of future work. Specifically
this would require incorporating a more expressive motion model than the Affine
model considered here. However, we do note that the inner product cost
function proposed in \cref{subsubsec:lk-inner-product} was extended into
the non-rigid case in the work of \citet{snape2016volumetric}.
%%%%%%%%%%%%%%%%%%%%%%%%%%%%%%%%%%%%%%%%%%%%%%%%%%%%%%%%%%%%%%%%%%%%%%%%%%%%%%%%
\subsection{Robust Decomposition of Image Collections}
%%%%%%%%%%%%%%%%%%%%%%%%%%%%%%%%%%%%%%%%%%%%%%%%%%%%%%%%%%%%%%%%%%%%%%%%%%%%%%%%
\textbf{Ground Truth Data.} Given the specific goal of shape recovery from
unconstrained collections of ``in-the-wild'' facial images, ground truth data
does not exist. As previously mentioned, the contribution of a ground truth
set of ``in-the-wild'' style images would be of great benefit to this chapter.
It may also be possible to generate realistic synthetic data that simulates
``in-the-wild'' images. Although this may be technically possible it remained
out of scope for this work as photo-realistic rendering of human faces
is a research topic in it's own right and there are few viable options available
to non-commercial individuals.

\textbf{Reliance on existing correspondences.} In this chapter we were interested
in investigating the fidelity of our reconstructions given a coarse piecewise
warping function. This was useful as the largest sets of facial images with
known correspondences are the sparsely annotated images provided for 2D
face alignment learning. Thus, the piecewise warping function enables
computing coarse correspondences for these images that are invariant to pose,
illumination and occlusions\footnote{Due to the fact that they are manually
annotated by humans.}. However, attempting to jointly estimate the surface
normals and the correspondences would be of great interest. This is further
highlighted by the use of the low-rank constraint. Given that the low-rank
constraint was shown to be beneficial it implies that there is a low-rank subspace
that describes the facial shape. In theory, this would also apply to the
correspondences. Therefore, it may be interesting to investigate low-rank
alignment methods such as RAPS~\citet{sagonas2014raps} for attempting to
jointly learn a constrained subspace and align the set of images.

\textbf{Lack of 3D prior.} Although the aim of \cref{ch:imag_coll} was to 
present a method that did \textit{not} require an explicit 3D prior, both
\cref{ch:singl_imag} and \cref{ch:face_flow} demonstrate that modern 3D statistical
models are both highly expressive and compact. Therefore, it would be
interesting to incorporate a 3D prior on the recover in order further constrain
the space of plausible facial shapes. In particular, it would be interesting
to borrow from other template based methods 
such as~\cite{kemelmacher2011facereconstruction} and seek to use the decomposition
as a method of recovering person specific deformations from the underlying
3D model. However, this problem is still plagued by that of correspondences
between the image and the model whereby recovering 2D correspondences between
totally unrelated ``in-the-wild'' images remains a challenging problem.

\textbf{Further investigation into the form of the decomposition.}
In \cref{ch:imag_coll} we proposed a robust matrix decomposition method and 
presented it's application to 3D surface recovery. However, matrix decomposition
methods have further uses in areas such as facial recognition. The general
form of our decomposition allows for any data that can be decomposed into a number
of discrete modes to be used. In fact, we hypothesise that allow we only
considered two modes: lighting and shape, it may be possible to perform
the decomposition across a number of Khatri-Rao products. This would then allow
the decomposition of data containing many modes of variation such as expression,
pose and illumination. Furthermore, the form of our decomposition would
allow for the recovery of these modes without the requirement of a complete
data tensor and is commonly required for multi-dimensional (tensor) decomposition
techniques.
%%%%%%%%%%%%%%%%%%%%%%%%%%%%%%%%%%%%%%%%%%%%%%%%%%%%%%%%%%%%%%%%%%%%%%%%%%%%%%%%
\subsection{Dense 2D Correspondences and Model Based 3D Recovery}
%%%%%%%%%%%%%%%%%%%%%%%%%%%%%%%%%%%%%%%%%%%%%%%%%%%%%%%%%%%%%%%%%%%%%%%%%%%%%%%%
\textbf{Ground Truth Data.} As previously discussed, the lack of a ground
truth dataset of 3D facial shape makes it difficult to quantitatively evaluate
facial surface recovery methods. Similarly, there has only recently emerged
a large dataset of sparse 2D landmarks for sequences~\cite{shen2015first} and
this dataset only contains 68 semi-automatically annotated landmarks. In this
work, we were investigating dense landmark recovery, of which there are 
currently no existing ground truth datasets. This is further complicated by
the fact that computing these ground truth correspondences is very difficult.
They cannot be manually annotated and thus another approximate method of
capture must be used such as computational stereo. However, these commonly
employed high-resolution capture methods do not perform well in the 
``in-the-wild'' scenarios considered in this work.

\textbf{Joint recovery of 2D alignment and 3D shape.}
In this work, we focus on the challenging problem of recovering dense 2D
correspondences across sequences. We demonstrate that 3D models can be used
to construct viable 2D deformation models with the additional benefit that the
correspondence between the 2D and 3D points allows for model-based
reconstruction methods to be employed. However, the model-based reconstruction
considered here is still a highly non-linear optimisation problem that may
yield suboptimal results. Furthermore, solving this second problem is costly
as the increase in the number of points causes the solving of the linear
systems to be expensive. In fact, the solution to these linear systems puts a 
limit on the density of the model as if the model is too dense it will become
impossible to solve the problem due to reasonable memory constraints. Therefore,
jointly solving this problem in the latent space of the 3D model would be
preferable as the algorithm complexity is reduced and the relationship between the
2D correspondences and 3D shape becomes fixed. However, introducing a 3D
motion model into the Lucas-Kanade algorithm makes it complicated to linearise.
For this reason, the best method to investigate would be to follow the recent
works in discriminative methods for facial alignment and learn a 
Supervised Descent~\cite{xiong2013supervised} style algorithm for 3D facial
alignment. Preliminary work has been performed in this 
area~\cite{thies2016face,Jourabloo:2015dw,Huber:2015bs,Jeni:2015ft,jo2015single}.

\textbf{Further uses of the 2D deformation basis.} In this work we focused
on the performance of our 2D alignment and it's application to model-based 3D
surface recovery. However, we did not provide specific uses of our alignment.
For example, sparse 2D facial landmarks are commonly employed in areas such as
face recognition and facial affect prediction. Anywhere where 2D facial landmarks
are currently considered as input features for inference may benefit from
the increased density and fidelity of our dense 2D landmarks. In particular,
it may help in areas such as Action Unit~\cite{ekman1977facial} prediction
where the sparsity of the commonly considered 2D landmarks is insufficient
to predict many of the more subtle action units.
%%%%%%%%%%%%%%%%%%%%%%%%%%%%%%%%%%%%%%%%%%%%%%%%%%%%%%%%%%%%%%%%%%%%%%%%%%%%%%%%
\section{Final Words}
%%%%%%%%%%%%%%%%%%%%%%%%%%%%%%%%%%%%%%%%%%%%%%%%%%%%%%%%%%%%%%%%%%%%%%%%%%%%%%%%
The problem of 3D surface recover remains an extremely challenging one. Even
when assuming that the object being recovered is only from a single class,
such as faces, it is still difficult to recover accurate shape. This is in
part due to the technical ambiguities present in the problem and in part
due to the familiarity of humans with human faces. The recovered surfaces must
be highly accurate to convince the viewer that they were indeed plausible
reconstructions of an image. However, the work presented here has shown that
we are approaching a point where it may be possible to recover useful facial
surfaces in a variety of scenarios. We hope that work continues in these
directions and that in the future facial surface recovery becomes a technology
rather than a research problem.
%%%%%%%%%%%%%%%%%%%%%%%%%%%%%%%%%%%%%%%%%%%%%%%%%%%%%%%%%%%%%%%%%%%%%%%%%%%%%%%%
\stopcontents[chapters]
%%%%%%%%%%%%%%%%%%%%%%%%%%%%%%%%%%%%%%%%%%%%%%%%%%%%%%%%%%%%%%%%%%%%%%%%%%%%%%%%