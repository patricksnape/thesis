%%%%%%%%%%%%%%%%%%%%%%%%%%%%%%%%%%%%%%%%%%%%%%%%%%%%%%%%%%%%%%%%%%%%%%%%%%%%%%%%
\chapter{Introduction}\label{chap:intro}
%%%%%%%%%%%%%%%%%%%%%%%%%%%%%%%%%%%%%%%%%%%%%%%%%%%%%%%%%%%%%%%%%%%%%%%%%%%%%%%%
The recovery of the underlying structure of a scene captured in an image is 
arguably one the core problems in Computer Vision. Although many properties
of a scene may be recovered, it is the geometry of the objects within the scene
that is of most use in many problems, as the geometry provides a strong model
from which to perform inference. In particular, the 3D shape of the underlying
objects is arguably the strongest cue for common tasks such as object
recognition and object localisation. However, the general problem or recovering
the 3D shape of an object from image (s) is incredibly ill-conditioned. Even when 
provided with multiple images, or additional information about the scene or
capturing conditions, 3D recovery is rife with ambiguities. As demonstrated
by~\cref{tbl:3d_recovery_methods}, many strategies have been proposed for
solving this problem.

In contrast to the difficulty of the general case, the recovery of 3D facial
has been highly successful. Human faces have a number of qualities that are
highly desirable for shape recovery: they are extremely homogenous in
configuration (all healthy human faces have two eyes, a nose and mouth in the 
same approximate location), convex, exhibit approximately lambertian 
reflectance~\cite{Sirovich:1987te,RefWorks:314,Basri:2003ie,RefWorks:98,Hallinan:1994dz},
are largely captured from a single direction (frontal) and are deformable
but not self occluding. Furthermore, there are is a large amount of publicly
available imagery of faces and faces are of significant interest in a number
of fields including entertainment, medicine and psychology. For example,
the seminal work of Blanz and Vetter~\cite{RefWorks:96} provided an impressive
shape recovery of a well-lit neutral face. However, despite the success of the
3D Morphable Model~\cite{RefWorks:96}, the problem of high quality 3D facial 
surface recovery from challenging images is still far from solved. For example,
the recovery of 3D facial surfaces under arbitrary conditions,
so called ``in-the-wild'' images, remains a challenging and active area of
research~\cite{KemelmacherShlizerman:2013iv,Suwajanakorn:2015gf,Suwajanakorn:2014bl,Snape:2015gl,Roth:2015hq}.
%%%%%%%%%%%%%%%%%%%%%%%%%%%%%%%%%%%%%%%%
\begin{table}[t]
	\centering
	\resizebox{\textwidth}{!}{%
		\begin{tabular}{@{}llccl@{}}
		\toprule
		\multicolumn{1}{c}{Classification}                  & \multicolumn{1}{c}{Method}                                                   & \# Images & Other Input                                                                       & \multicolumn{1}{c}{Output} \\ \midrule
		\multirow{4}{*}[-0.7cm]{Image Formation}            & Shape-from-Shading                                                           & $1$       & \begin{tabular}[c]{@{}c@{}}Reflectance Model, \\ Lighting Directions\end{tabular} & Surface Normals            \\ \cmidrule(l){3-5} 
		                                                    & \begin{tabular}[c]{@{}l@{}} (Un) Calibrated \\ Photometric Stereo\end{tabular} & $\geq 3$  & \begin{tabular}[c]{@{}c@{}}Lighting Directions \\ (If Calibrated)\end{tabular}    & Surface Normals            \\ \cmidrule(l){3-5} 
		                                                    & Shape-from-Contour                                                           & $1$       & Object Contour                                                                    & Coarse Surface Normals     \\ \cmidrule(l){3-5} 
		                                                    & Analysis-by-Synthesis                                                        & $1$       & 3D Model                                                                          & 3D Shape                   \\ \midrule
		\multicolumn{1}{c}{\multirow{2}{*}{Photogrammetry}} & Multi-View Stereo                                                            & $k$       & Camera Extrinsics                                                                 & 3D Shape                   \\ \cmidrule(l){3-5} 
		\multicolumn{1}{c}{}                                & Structure-from-Motion                                                        & $k$       & Corresponding Points                                                              & 3D Shape                   \\ \midrule
		\multicolumn{1}{c}{Other}                           & Shape Transfer                                                               & $1$       & 3D Model                                                                          & 3D or 2.5D Shape           \\ \bottomrule
		\end{tabular}
	}
	\caption{A summary of methods for recovering 3D shape from images. The two 
	         largest subcategories are image formation algorithms and 
	         photogrammetry.}
\label{tbl:3d_recovery_methods}
\end{table}
%%%%%%%%%%%%%%%%%%%%%%%%%%%%%%%%%%%%%%%%
%%%%%%%%%%%%%%%%%%%%%%%%%%%%%%%%%%%%%%%%%%%%%%%%%%%%%%%%%%%%%%%%%%%%%%%%%%%%%%%%
\section{Contributions}\label{sec:intro_contrib}
%%%%%%%%%%%%%%%%%%%%%%%%%%%%%%%%%%%%%%%%%%%%%%%%%%%%%%%%%%%%%%%%%%%%%%%%%%%%%%%%

%%%%%%%%%%%%%%%%%%%%%%%%%%%%%%%%%%%%%%%%%%%%%%%%%%%%%%%%%%%%%%%%%%%%%%%%%%%%%%%%
\section{Publications}\label{sec:intro_pubs}
%%%%%%%%%%%%%%%%%%%%%%%%%%%%%%%%%%%%%%%%%%%%%%%%%%%%%%%%%%%%%%%%%%%%%%%%%%%%%%%%
In this section I provide a list of publication that were authored during the
course of my thesis. These publications are split into two sections, those
that are related to the contents of this thesis 
(\cref{subsec:intro_rel_pubs}) and other publications that 
are not directly relevant (\cref{subsec:intro_other_pubs}).
%%%%%%%%%%%%%%%%%%%%%%%%%%%%%%%%%%%%%%%%%%%%%%%%%%%%%%%%%%%%%%%%%%%%%%%%%%%%%%%%
\subsection{Related Publications}\label{subsec:intro_rel_pubs}
%%%%%%%%%%%%%%%%%%%%%%%%%%%%%%%%%%%%%%%%%%%%%%%%%%%%%%%%%%%%%%%%%%%%%%%%%%%%%%%%
\begin{itemize}
	\item\bibentry{Snape:2014de}
	\item\bibentry{Snape:2015gl}
	\item\bibentry{Snape:2015hj}
\end{itemize}
%%%%%%%%%%%%%%%%%%%%%%%%%%%%%%%%%%%%%%%%%%%%%%%%%%%%%%%%%%%%%%%%%%%%%%%%%%%%%%%%
\subsection{Other Publications}\label{subsec:intro_other_pubs}
%%%%%%%%%%%%%%%%%%%%%%%%%%%%%%%%%%%%%%%%%%%%%%%%%%%%%%%%%%%%%%%%%%%%%%%%%%%%%%%%
\begin{itemize}
	\item\bibentry{menpo14}
	\item\bibentry{Chrysos:2015gt}
\end{itemize}
%%%%%%%%%%%%%%%%%%%%%%%%%%%%%%%%%%%%%%%%%%%%%%%%%%%%%%%%%%%%%%%%%%%%%%%%%%%%%%%%
\section{Outline}\label{sec:introduction_outline}
%%%%%%%%%%%%%%%%%%%%%%%%%%%%%%%%%%%%%%%%%%%%%%%%%%%%%%%%%%%%%%%%%%%%%%%%%%%%%%%%

%%%%%%%%%%%%%%%%%%%%%%%%%%%%%%%%%%%%%%%%%%%%%%%%%%%%%%%%%%%%%%%%%%%%%%%%%%%%%%%%