%%%%%%%%%%%%%%%%%%%%%%%%%%%%%%%%%%%%%%%%%%%%%%%%%%%%%%%%%%%%%%%%%%%%%%%%%%%%%%%%
\section{Summary and Conclusion}\label{sec:singl_img_summary}
%%%%%%%%%%%%%%%%%%%%%%%%%%%%%%%%%%%%%%%%%%%%%%%%%%%%%%%%%%%%%%%%%%%%%%%%%%%%%%%%
In this chapter we investigated the power of creating statistical models
of surface normals. The creation of these statistical models is complicated
by the non-euclidean nature of directional data such as surface normals.
Therefore, we proposed to formulate statistical model building of normals
as a Kernel-PCA problem as proposed two kernels for computing these non-linear
models. We also incoporated existing work on using surface normals in 
statistical models and show the 
Azimuthal Equidistant Projection (AEP)~\cite{smith2006recovering} and
Principal Geodesic Analysis (PCA)~\cite{smith2008facial} can be cast as 
kernels in our Kernel-PCA framework. We extend the geometric 
Shape-from-Shading (SfS) of \citet{smith2006recovering} and show that our 
Kernel-PCA framework naturally augments the geometric SfS algorithm. We also
provide the first ``in-the-wild'' results using this SfS algorithm and show
that model-based SfS algorithms are capable of recovering plausible high-frequency
normals and very challenging conditions.

Furthermore, we noted from the analysis of our two novel kernels that normals
are suitable for use in Lucas-Kanade~\cite{lucas1981iterative} alignment
algorithms. We investigated the alignment of both 2.5D depth data and volumetric
3D data. In particular, we empirically showed that 3D gradients, which are
analagous to normals in 3D, can be used to construct an alignment algorithm
that is robust to gross outliers. We experimentally verified these claims
on simulated outliers on data from the Visible Human project~\cite{RefWorks:81}.

However, despite the robust properties of normals and the promising results
on building statistical models of normals, building these models is still
quite restrictive. The main restriction is the fact that the data used
for constructing the models all comes from carefully constructed datasets
that do not model the ``in-the-wild'' conditions we wish to recover shape
from. Furthermore, SfS still requires the estimation of the single point light 
source illuminating the image, which is a coarse and unrealistic lighting
model for ``in-the-wild'' images. Therefore, in the next chapter we investigate
how shading constraints can be used to build statistical models of shape
directly from the data without any shape priors whatsoever. This is achieved
through the adoption of a more realistic lighting model called Spherical
Harmonic illumination. To this end, we describe how Spherical Harmonic illumination
bases can be constructed robustly directly from ``in-the-wild'' images
without shape priors.