%%%%%%%%%%%%%%%%%%%%%%%%%%%%%%%%%%%%%%%%%%%%%%%%%%%%%%%%%%%%%%%%%%%%%%%%%%%%%%%%
\subsubsection{Spherical SSD LK}\label{subsubsec:lk-spherical}
%%%%%%%%%%%%%%%%%%%%%%%%%%%%%%%%%%%%%%%%%%%%%%%%%%%%%%%%%%%%%%%%%%%%%%%%%%%%%%%%
In contrast to the inner product derivation in the previous section, the
spherical representation requires the optimisation of the summation of two
cosine correlations. In theory, it would be possible to solve for each
correlation separately in a manner similar to that proposed
by \citet{RefWorks:6}. This would be suboptimal as it would require an
alternating optimisation scheme. Therefore, in the interest of solving a single 
objective, we note the following relationship between the two summations:
%%%%%%%%%%%%%%%%%%%%
\begin{equation}
    \begin{aligned}\label{eq:argmax_spherical}
        \argmax_{\p} \quad &\sum_k \cos{\left(\Delta \phi(\p)\right)} + \sum_k \cos{\left(\Delta \theta(\p)\right)}
    \end{aligned}
\end{equation}
%%%%%%%%%%%%%%%%%%%%
is equivalent to the minimisation of
%%%%%%%%%%%%%%%%%%%%
\begin{equation}
    \begin{aligned}\label{eq:argmin_spherical}
        &\argmin_{\p} \left\lVert
            \begin{pmatrix}
                \cos{\phi_I}(\p) \\
                \sin{\phi_I}(\p) \\
                \cos{\theta_I}(\p) \\
                \sin{\theta_I}(\p)
            \end{pmatrix}
            -
            \begin{pmatrix}
                \cos{\phi_T}(\zero) \\
                \sin{\phi_T}(\zero) \\
                \cos{\theta_T}(\zero) \\
                \sin{\theta_T}(\zero)
            \end{pmatrix}
            \right\lVert^2 \triangleq \\
        &\argmin_{\p} \left\lVert
            \begin{pmatrix}
                \bb{\tilde{g}}_{I,x}(\p) \\ 
                \bb{\tilde{g}}_{I,y}(\p) \\ 
                \bb{\tilde{g}}_{I,z}(\p) \\ 
                \sqrt{1 - \bb{\tilde{g}}_{I,z}^2}(\p)
            \end{pmatrix}
            -
            \begin{pmatrix}
                \bb{\tilde{g}}_{T,x}(\zero) \\ 
                \bb{\tilde{g}}_{T,y}(\zero) \\ 
                \bb{\tilde{g}}_{T,z}(\zero) \\ 
                \sqrt{1 - \bb{\tilde{g}}_{T,z}^2}(\zero)
            \end{pmatrix}
            \right\lVert^2
    \end{aligned}
\end{equation}
%%%%%%%%%%%%%%%%%%%%
where $\sin{\theta_{\ast}}(\cdot) = \sqrt{1 - {\bb{\tilde{g}}_{\ast,z}^2}(\cdot)}$
where $\ast$ denotes either the template or the input image. For notational
simplicity, let $\bb{\tilde{sz}}(\cdot) = \sqrt{1 - {\bb{\tilde{g}}_{z}^2}(\cdot)}$.

We define the forward additive objective function as
%%%%%%%%%%%%%%%%%%%%
\begin{equation}\label{eq:spher-ssd}
    \argmin_{\p} \norm{\bb{\hat{g}}_I(\p) - \bb{\hat{g}}_T(\zero)}^2
\end{equation}
%%%%%%%%%%%%%%%%%%%%
where $\bb{\hat{g}}_I(\p) = {\left[  \bb{\tilde{g}}_{I,x}(\p),
\bb{\tilde{g}}_{I,y}(\p), \bb{\tilde{g}}_{I,z}(\p),
\bb{\tilde{g}}_{I,sz}(\p) \right]}^{\top}$ and $\bb{\hat{g}}_T(\zero) =
{\left[ \bb{\tilde{g}}_{T,x}(\zero), \bb{\tilde{g}}_{T,y}(\zero),
\bb{\tilde{g}}_{T,z}(\zero), \bb{\tilde{g}}_{T,sz}(\zero)
\right]}^{\top}$, the concatenated vectors of each normalised component.

Similar to the derivation in \cref{subsec:lk-inner-product}, the Jacobian
must be taken over each component and thus linearising around
$\bb{\hat{g}}_I(\p)$ yields
%%%%%%%%%%%%%%%%%%%%
\begin{equation}\label{eq:spher-ssd-linearised}
    \argmin_{\p} \norm{\bb{\hat{g}}_{I}(\p) + \hat{\bb{J}}_{\bb{g}} \Delta \p - \bb{\hat{g}}_{T}(\zero)}^2
\end{equation}
%%%%%%%%%%%%%%%%%%%%
where
$\hat{\bb{J}}_{\bb{g}} = {\left[ \hat{\bb{J}}_{\bb{g},x}, \hat{\bb{J}}_{\bb{g},y}, \hat{\bb{J}}_{\bb{g},z}, \hat{\bb{J}}_{\bb{g},sz} \right]}^{\top}$. 
Unlike in \cref{subsec:lk-inner-product}, the calculation of each 
Jacobian is not identical due to the different normalisation procedure taken 
for each component. Given that we can split the partial derivative
$\bb{J}_{\bb{g},x} = \bb{J}_x \odot \frac{\partial \bb{g}_{I,x}(\p)}{\partial \p}$, 
we define the component specific Jacobian,
$\bb{J}_i \nobreak \forall \nobreak i \in \nobreak \{ x,y,z,sz \}$ as:
%%%%%%%%%%%%%%%%%%%%
\begin{equation}\label{eq:spher-ssd-chain-rule}
    \begin{aligned}
        \bb{J}_x    &= \frac{{\bb{g}_{I,y}(\p)}^2}{{\left({\bb{g}_{I,x}(\p)}^2 + {\bb{g}_{I,y}(\p)}^2 + {\bb{g}_{I,z}(\p)}^2 \right)}^{\sfrac{3}{2}}} \\
        \bb{J}_y    &= \frac{{\bb{g}_{I,x}(\p)}^2}{{\left({\bb{g}_{I,x}(\p)}^2 + {\bb{g}_{I,y}(\p)}^2 + {\bb{g}_{I,z}(\p)}^2 \right)}^{\sfrac{3}{2}}} \\
        \bb{J}_z    &= \frac{{\bb{g}_{I,x}(\p)}^2 + {\bb{g}_{I,y}(\p)}^2}{{\left({\bb{g}_{I,x}(\p)}^2 + {\bb{g}_{I,y}(\p)}^2 + {\bb{g}_{I,z}(\p)}^2 \right)}^{\sfrac{3}{2}}} \\
        \bb{J}_{sz} &= - \frac{\bb{g}_{I,z}(\p) {\left( \frac{{\bb{g}_{I,x}(\p)}^2 + {\bb{g}_{I,y}(\p)}^2}{{\bb{g}_{I,x}(\p)}^2 + {\bb{g}_{I,y}(\p)}^2 + {\bb{g}_{I,z}(\p)}^2} \right)}^{\sfrac{3}{2}}}{{\bb{g}_{I,x}(\p)}^2 + {\bb{g}_{I,y}(\p)}^2}
    \end{aligned}
\end{equation}
%%%%%%%%%%%%%%%%%%%%
Now, given the definitions of the correct Jacobians per component, we can solve 
(\ref{eq:spher-ssd-linearised}) as:
%%%%%%%%%%%%%%%%%%%%
\begin{equation}\label{eq:spher-ssd-gauss-newton}
    \Delta \p = \bb{H}^{-1} \hat{\bb{J}}_{\bb{g}}^{\top} \left[ \bb{\hat{g}}_{T}(\zero) - \bb{\hat{g}}_{I}(\p) \right]
\end{equation}
%%%%%%%%%%%%%%%%%%%%
Given that the update in (\ref{eq:spher-ssd-gauss-newton}) is identical to that
of (\ref{eq:l2-lk-gauss-newton-fa}), it would be trivial to formulate an inverse
compositional form of this residual by following the steps described in
\cref{subsec:lk-ic}.
%%%%%%%%%%%%%%%%%%%%%%%%%%%%%%%%%%%%%%%%%%%%%%%%%%%%%%%%%%%%%%%%%%%%%%%%%%%%%%%%