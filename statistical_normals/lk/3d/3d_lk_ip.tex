%%%%%%%%%%%%%%%%%%%%%%%%%%%%%%%%%%%%%%%%%%%%%%%%%%%%%%%%%%%%%%%%%%%%%%%%%%%%%%%%
\subsubsection{Inner Product ECC LK}\label{subsubsec:lk-inner-product}
%%%%%%%%%%%%%%%%%%%%%%%%%%%%%%%%%%%%%%%%%%%%%%%%%%%%%%%%%%%%%%%%%%%%%%%%%%%%%%%%
Given the inner product similarity measure as described in
\cref{subsubsec:cosine_inner_product}, we seek to embed it within the LK
framework in order to present a robust parametric alignment algorithm.
Therefore, we begin by restating our cost function:
%%%%%%%%%%%%%%%%%%%%
\begin{equation}\label{eq:cosine_alignment}
   \argmax_{\p} \quad \sum_k \cos{\left(\sigma(\p)\right)}
\end{equation}
%%%%%%%%%%%%%%%%%%%%
with an abuse of notation for the parameters which are hidden within the cosine
function. Expanding (\ref{eq:cosine_alignment}) reveals the parameters and makes
the relationship between our inner product similarity and the ECC framework
clear:
%%%%%%%%%%%%%%%%%%%%
\begin{equation}\label{eq:ip-ecc-lk-max}
   \argmax_{\p} \quad {\bb{\tilde{g}}_{I}(\p)}^{\top} \bb{\tilde{g}}_T(\zero)
\end{equation}
%%%%%%%%%%%%%%%%%%%%
which yields our forward additive algorithm.

However, unlike in ECC where the vectors represent concatenated normalised
intensities, we are considering normalised gradients. Since gradients have three
separate components we must consider the derivatives when linearising
$\bb{g}_I$. Since $\bb{g}_I$ is composed of multiple components, there
will be extra derivatives to calculate via the chain rule. Formally, linearising
(\ref{eq:ip-ecc-lk-max}) with respect to $\bb{g}_I$ yields
%%%%%%%%%%%%%%%%%%%%
\begin{equation}\label{eq:ip-ecc-lk-linearised}
    \argmax_{\p} \quad \bb{\tilde{g}}_T \frac{\bb{\tilde{g}}_{I}(\p) + \bb{J}_{\bb{g}} \Delta \p}{\norm{{\bb{\tilde{g}}_{I}(\p) + \bb{J}_{\bb{g}} \Delta \p}}}
\end{equation}
%%%%%%%%%%%%%%%%%%%%
where $\bb{J}_{\bb{g}}$ is the matrix formed by correctly computing
the derivative of $\bb{g}_{I}$ with respect to each component of
$\bb{g}_I$. For example, given that $\bb{g}_{I,x}(\p)$ is a vector
formed of the $x$-components of the gradients and $\bb{\tilde{g}}_{I,x}(\p)$
is a vector formed of $\bb{\tilde{g}}_{I,x}(\p)[k] =
\frac{\bb{g}_{I,x}(\p)[k]}{\norm{\bb{g}_{I}(\p)[k]}}$, the true
derivative of $\bb{\tilde{g}}_{I,x}(\p)$ is
%%%%%%%%%%%%%%%%%%%%
\begin{equation}\label{eq:ip-ecc-chain-rule}
    \begin{aligned}
        \frac{\partial \bb{\tilde{g}}_{I,x}(\p)}{\partial \p}          &= \frac{\partial \bb{\tilde{g}}_{I,x}(\p)}{\partial \bb{g}_{I,x}(\p)} \frac{\partial \bb{g}_{I,x}(\p)}{\partial \p} \\
        \frac{\partial \bb{\tilde{g}}_{I,x}(\p)}{\partial \bb{g}_{I,x}(\p)} &= \frac{{\bb{g}_{I,y}(\p)}^2 + {\bb{g}_{I,z}(\p)}^2}{{\left({\bb{g}_{I,x}(\p)}^2 + {\bb{g}_{I,y}(\p)}^2 + {\bb{g}_{I,z}(\p)}^2 \right)}^{\sfrac{3}{2}}}
    \end{aligned}
\end{equation}
%%%%%%%%%%%%%%%%%%%%
where $\frac{\partial \bb{g}_{I,x}(\p)}{\partial \p}$ is equivalent to
$\frac{\partial I(\p)}{\partial \p}$ in the original ECC equations. The $y$ and
$z$ derivatives are given in a similar fashion as
%%%%%%%%%%%%%%%%%%%%
\begin{equation}\label{eq:ip-ecc-chain-rule-y-z}
    \begin{aligned}
        \frac{\partial \bb{\tilde{g}}_{I,y}(\p)}{\partial \bb{g}_{I,y}(\p)} &= \frac{{\bb{g}_{I,x}(\p)}^2 + {\bb{g}_{I,z}(\p)}^2}{{\left({\bb{g}_{I,x}(\p)}^2 + {\bb{g}_{I,y}(\p)}^2 + {\bb{g}_{I,z}(\p)}^2 \right)}^{\sfrac{3}{2}}} \\
        \frac{\partial \bb{\tilde{g}}_{I,z}(\p)}{\partial \bb{g}_{I,z}(\p)} &= \frac{{\bb{g}_{I,x}(\p)}^2 + {\bb{g}_{I,y}(\p)}^2}{{\left({\bb{g}_{I,x}(\p)}^2 + {\bb{g}_{I,y}(\p)}^2 + {\bb{g}_{I,z}(\p)}^2 \right)}^{\sfrac{3}{2}}}
    \end{aligned}
\end{equation}
%%%%%%%%%%%%%%%%%%%%
However, $\nabla \bb{g}_{I,x}$, formally the image gradient, represents the
gradient over only the $x$-component, and is equivalent to the second order
derivative of the gradients with respect to $x$.

Since $\frac{\partial \bb{g}_{I,x}(\p)}{\partial \p}$ is a matrix and
$\frac{\partial \bb{\tilde{g}}_{I,x}(\p)}{\partial \bb{g}_{I,x}(\p)}$ is
a vector, we multiply the two using a Hadamard product, denoted
by the $\odot$ operator. However, $\frac{\partial
\bb{\tilde{g}}_{I,x}(\p)}{\partial \bb{g}_{I,x}(\p)}$ must first form a
matrix, $\bb{J}_x$, of size $D \times p$ by repeating the vector $p$
times to form columns within the matrix. Finally the total $x$-component
Jacobian is given by
$\bb{J}_{\bb{g},x} = \bb{J}_x \odot \frac{\partial \bb{g}_{I,x}(\p)}{\partial \p}$.

Given that $\bb{J}_{\bb{g},i} \forall i \in \{ x,y,z \}$ have been
calculated, the total derivative term is given by
 $\bb{J}_{\bb{g}} = {\left[ \bb{J}_{\bb{g},x}, \bb{J}_{\bb{g},y}, \bb{J}_{\bb{g},z} \right]}^{\top}$. 
Solving for $\Delta \p$ is now identical to the ECC formulation:
%%%%%%%%%%%%%%%%%%%%
\begin{equation}\label{eq:ip-ecc-lk-gauss-newton}
    \Delta \p = \bb{H}^{-1} \bb{J_g} \left[ \frac{\norm{\bb{g}_I(\p)}^2 - {\bb{g}_I(\p)}^{\top} \bb{Q} \bb{g}_I(\p)}{\bb{\tilde{g}}_T^{\top} \bb{g}_I(\p) - \bb{\tilde{g}}_T^{\top} \bb{Q} \bb{g}_I(\p)} \bb{\tilde{g}}_T - \bb{g}_I(\p) \right]
\end{equation}
%%%%%%%%%%%%%%%%%%%%
Since the update step is identical to the one given in 
(\ref{eq:ecc-lk-gauss-newton-ic}) it is simple to reformulate the inner product 
ECC in an inverse compositional form by following a derivation identical to 
\cref{subsec:lk-ic}. In the experimental section, we consider the inverse
compositional form of the algorithm.
%%%%%%%%%%%%%%%%%%%%%%%%%%%%%%%%%%%%%%%%%%%%%%%%%%%%%%%%%%%%%%%%%%%%%%%%%%%%%%%%