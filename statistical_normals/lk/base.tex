%%%%%%%%%%%%%%%%%%%%%%%%%%%%%%%%%%%%%%%%%%%%%%%%%%%%%%%%%%%%%%%%%%%%%%%%%%%%%%%%
\section{Lucas-Kanade Alignment for 2.5D and 3D Images}\label{sec:singl_imag_lk}
%%%%%%%%%%%%%%%%%%%%%%%%%%%%%%%%%%%%%%%%%%%%%%%%%%%%%%%%%%%%%%%%%%%%%%%%%%%%%%%%
In this section we seek to perform alignment on 2.5D and 3D images using
normals as our feature representation. We build upon the
Lucas-Kanade~\cite{lucas1981iterative} literature as these algorithms provide
a strong foundation for construction of parametric image alignment methods. We note that
the cases of 2.5D and 3D data will be considered separately and the derivation
of the Lucas-Kanade algorithm in each case is slightly different. In the
case of 2.5D data, such as that of FRGC~\cite{phillips2005overview}, we follow
the methodology of \citet{antonakos2015feature} and use normals as an image
``feature'' or ``descriptor'' and linearise it as though it were the image pixels.
We also utilise the proposed KPCA framework for normals in order to use normals
to perform parametric image alignment with an appearance model as described
in the Active Appearance Model (AAM)~\cite{cootes2001active} literature. We
utilise the AAM formulation described by \citet{matthews2004active} which
demonstrates that AAMs can be seen as an extension of the classical Lucas-Kanade
algorithm to include a statistical model of appearance. These statistical models
are most commonly constructed using PCA and therefore our proposed KPCA
framework is crucial to the construction of statistical appearance models for
parametric alignment using surface normals. To this end, we describe how
normals can be used for 2D alignment using 2.5D data in the case of the
classical Affine Lucas-Kanade alignment and it's extension as AAMs using a
KPCA appearance model.

For 3D data, we take a more structured approach and present two fitting algorithms
that explicitly linearise the 3D gradients, or normals, in a manner inspired
by \citet{tzimiropoulos2011robust}. This was not possible in the case of 2.5D
data as it is not possible to directly linearise the normals in 2D.

The remainder of this section will detail the traditional Affine Lucas-Kanade
algorithm which we use for both 2.5D and 3D alignment. Therefore, we begin by
discussing the generic construction of the Lucas-Kanade algorithm which is
agnostic to the coordinate system.
%%%%%%%%%%%%%%%%%%%%%%%%%%%%%%%%%%%%%%%%%%%%%%%%%%%%%%%%%%%%%%%%%%%%%%%%%%%%%%%%
\paragraph{Lucas-Kanade Notation}\label{subsubsec:lk-notation}
%%%%%%%%%%%%%%%%%%%%%%%%%%%%%%%%%%%%%%%%%%%%%%%%%%%%%%%%%%%%%%%%%%%%%%%%%%%%%%%%
When referring to the operations performed by the LK algorithm we will use the
following notations. Images are denoted by unbolded capital letters such as
$I$ and $T$. Indexing the image $I$ by a discretised coordinate is
denoted $I(x, y) = c$ where $c$ would be a grayscale value in this case. We
further define a linear index $k$ into the vectorised image such that $I(k)$
implies the $k = x * height + y$ pixel coordinate.
This allows for simpler indexing of the vectorised image.
Warp functions that describe how to sample a sub-pixel coordinate
vector from an image are denoted by
$\mathcal{W}(\bb{x}_i;\p) = \left[\mathcal{W}_x(\bb{x}_i;\p), \mathcal{W}_y(\bb{x}_i;\p),\mathcal{W}_z(\bb{x}_i;\p) \right]$ for a 3D image and $\mathcal{W}(\bb{x}_i;\p) = \left[\mathcal{W}_x(\bb{x}_i;\p), \mathcal{W}_y(\bb{x}_i;\p) \right]$ for a 2D image. This notation
expresses the warping of the $i$th coordinate vector,
$\bb{x}_i = {[x_i, y_i, z_i]}^{\top}$ for a 3D coordinate and
$\bb{x}_i = {[x_i, y_i]}^{\top}$ for a 2D coordinate, by a set of parameters
$\p = {\left[p_1, \ldots, p_n\right]}^{\top}$, where $n$ is the number of warp
parameters. We define the vectorised set of singe coordinate vectors as
$\bb{x} = \left[ x_1, y_1, z_1, \ldots, x_D, y_D, z_D \right]$ for 3D and
$\bb{x} = \left[ x_1, y_1 \ldots, x_D, y_D \right]$ for 2D that
represents the concatenated vector of coordinates, of length $D$, which allows
the definition of a single warp for an entire 3D image as
$\mathcal{W}(\bb{x};\p) = \left[\right. \mathcal{W}_x(\bb{x}_1;\p), \mathcal{W}_y(\bb{x}_1;\p), \mathcal{W}_z(\bb{x}_1;\p),\ldots,\mathcal{W}_x(\bb{x}_D;\p), \mathcal{W}_y(\bb{x}_D;\p), \mathcal{W}_z(\bb{x}_D;\p) \left.\right]$ and
$\mathcal{W}(\bb{x};\p) = \left[\right. \mathcal{W}_x(\bb{x}_1;\p), \mathcal{W}_y(\bb{x}_1;\p),\ldots,\mathcal{W}_x(\bb{x}_D;\p), \mathcal{W}_y(\bb{x}_D;\p) \left.\right]$ for a 2D image.
We assume that the identity warp is found when $\p = \zero$, which implies that
$\mathcal{W}(\bb{x};\zero) = \bb{x}$. We abuse notation and define the
warping of an image $I$ by parameter vector $\p$ as $I(\p) = I(\mathcal{W}(\bb{x};\p))$,
where $I(\p)$ is a single column vector of concatenated pixels. For example,
$I(\zero) = {[c_1, \ldots, c_D]}^T \in \R^{D \times 1}$ denotes the
sampling of the image using the
identity warp and therefore each pixel in a grayscale image, $c_i$, are returned
as a concatenated vector. Following the definition of
\citet{antonakos2015feature}, we extend this sampling to a multi-channel image
and so the sampling of an $C$ channel image is defined by
$I(\zero) = {[c^1_1, \ldots, c^C_1, \ldots, c^1_D, \ldots, c^C_D]}^T \in \R^{(CD) \times 1}$.
The warping function is more intuitively defined as the sampling of a grid
of pixels within the input image. This sampling is commonly performed using
an interpolation method such as bilinear interpolation. This grid is then
transformed using a parametric transformation such as an Affine or Similarity
transform. For the following sections on 2D Lucas-Kanade alignment, let us
assume that the warp parameters $\p$ are the $6$ parameters of a 2D Affine warp.
We recommend the reader see \citet{baker2004lucas} for more information on the
Lucas-Kanade algorithm as presented in this section.
%%%%%%%%%%%%%%%%%%%%%%%%%%%%%%%%%%%%%%%%%%%%%%%%%%%%%%%%%%%%%%%%%%%%%%%%%%%%%%%%
\paragraph{Forward Additive LK Fitting}\label{subsubsec:lk-fa}
%%%%%%%%%%%%%%%%%%%%%%%%%%%%%%%%%%%%%%%%%%%%%%%%%%%%%%%%%%%%%%%%%%%%%%%%%%%%%%%%
%%%%%%%%%%%%%%%%%%%%%%%%%%%%%%%%%%%%%%%%%%%%%%%%%%%%%%%%%%%%%%%%%%%%%%%%%%%%%%%%
\begin{algorithm}[t]
\caption{Forward Additive 2D Lucas-Kanade Algorithm.}
\label{alg:lk_2d_fa}
    {\textbf{Input:} Input Image $I$, Template Image $T$} \\
    {\textbf{Output:} Warp parameter vector $\p$}  \\
    \begin{algorithmic}[1]
        \State{}Initialise: $\p = \zero$\\
        \While{$\norm{\Delta \p} > \epsilon$} %(Outer loop)
            \State{}Warp $I$ with current estimate of the parameters $I(\p)$
            \State{}Compute the error image $I(\p) - T(\zero)$
            \State{}Warp the gradient of the input image $\nabla I(\p)$
            \State{}Compute the warp Jacobian at the current parameter estimate $\frac{\partial \mathcal{W}}{\partial \p}$
            \State{}Compute the steepest descent images $\frac{\partial I(\p)}{\partial \p} = \nabla I(\p) \frac{\partial \mathcal{W}}{\partial \p}$
            \State{}Compute the Hessian $\left[ {\frac{\partial I(\p)}{\partial \p}}^{\top} \frac{\partial I(\p)}{\partial \p} \right]$
            \State{}Compute the $\Delta \p$ using \cref{eq:l2-lk-gauss-newton-fa}
            \State{}Compute the additive update $\p \leftarrow \p + \Delta \p$
        \EndWhile{}
    \end{algorithmic}
\end{algorithm}
%%%%%%%%%%%%%%%%%%%%%%%%%%%%%%%%%%%%%%%%%%%%%%%%%%%%%%%%%%%%%%%%%%%%%%%%%%%%%%%%
The original forward additive $\ltwo$ LK
algorithm~\cite{baker2004lucas,lucas1981iterative} seeks to minimise the sum of
squared differences (SSD) between a given template image and an input image by
minimising the sum of the squared pixel differences:
%%%%%%%%%%%%%%%%%%%%
\begin{equation}\label{eq:l2-lk-fa}
    \argmin_{\p} \quad \norm{I(\p) - T(\zero)}^2
\end{equation}
%%%%%%%%%%%%%%%%%%%%
where $T(\zero)$ is the unwarped reference template image. Due to the non-linear
nature of (\ref{eq:l2-lk-fa}) with respect to $\p$, (\ref{eq:l2-lk-fa}) is
linearised by taking the first order Taylor series expansion. By iteratively
solving for some small $\Delta \p$ update to $\p$, the objective function
becomes
%%%%%%%%%%%%%%%%%%%%
\begin{equation}\label{eq:l2-lk-linearised-fa}
    \argmin_{\p} \quad \norm{I(\p) + \nabla I(\p) \frac{\partial \mathcal{W}}{\partial \p} \Delta \p - T(\zero)}^2
\end{equation}
%%%%%%%%%%%%%%%%%%%%
where $\nabla I(\p)$ is the gradient over each dimension of $I(\p)$ warped into the
frame of $T$ by the current warp estimate $\mathcal{W}(\bb{x};\p)$.
$\frac{\partial \mathcal{W}}{\partial \p}$ is the Jacobian of the warp and
represents the first order partial derivatives of the warp with respect to each
parameter. $\nabla I(\p) \frac{\partial \mathcal{W}}{\partial \p}$ is commonly
referred to as the steepest descent images. We will express the steepest descent
images as $\frac{\partial I(\p)}{\partial \p}$.
\cref{eq:l2-lk-linearised-fa} is now solvable by assuming the
Gauss-Newton approximation to the Hessian,
$\bb{H} = \left[ {\frac{\partial I(\p)}{\partial \p}}^{\top} \frac{\partial I(\p)}{\partial \p} \right]$:
%%%%%%%%%%%%%%%%%%%%
\begin{equation}\label{eq:l2-lk-gauss-newton-fa}
    \Delta \p = \bb{H}^{-1} \frac{\partial I(\p)}{\partial \p}^{\top} \left[ T(\zero) - I(\p) \right]
\end{equation}
%%%%%%%%%%%%%%%%%%%%
\cref{eq:l2-lk-gauss-newton-fa} can then be solved by iteratively
updating $\p \leftarrow \p + \Delta \p$ until convergence.
See \cref{alg:lk_2d_fa} for an explanation of the iterative steps required
for solving the problem.
%%%%%%%%%%%%%%%%%%%%%%%%%%%%%%%%%%%%%%%%%%%%%%%%%%%%%%%%%%%%%%%%%%%%%%%%%%%%%%%%
\paragraph{ECC LK Fitting}\label{subsubsec:lk-ecc}
%%%%%%%%%%%%%%%%%%%%%%%%%%%%%%%%%%%%%%%%%%%%%%%%%%%%%%%%%%%%%%%%%%%%%%%%%%%%%%%%
The enhanced correlation coefficient (ECC) measure, proposed by
\citet{evangelidis2008parametric}, seeks to be invariant to illumination differences
between the input and template image. This is done by suppressing the magnitude
of each pixel through normalisation. In~\cite{evangelidis2008parametric}, they provide the
following cost function
%%%%%%%%%%%%%%%%%%%%
\begin{equation}\label{eq:ecc-lk-max}
   \argmax_{\p} \quad \frac{{I(\p)}^{\top} T(\zero)}{\norm{I(\p)} \norm{T(\zero)}}
\end{equation}
%%%%%%%%%%%%%%%%%%%%
Assuming a delta update as before and linearising in a similar manner to
(\ref{eq:l2-lk-linearised-fa}) results in
%%%%%%%%%%%%%%%%%%%%
\begin{equation}\label{eq:ecc-lk-linearised}
    \argmax_{\p} \quad \hat{T} \frac{I(\p) + \J \Delta \p}{\norm{{I(\p) + \J \Delta \p}}}
\end{equation}
%%%%%%%%%%%%%%%%%%%%
where $\hat{T} = \frac{T(\zero)}{\norm{T(\zero)}}$ and
$\J = \frac{\partial I(\p)}{\partial \p}$ for brevity in the following equations.
\citet{evangelidis2008parametric} give a very comprehensive proof of
the upper bound of \cref{eq:ecc-lk-linearised}, which yields the
following solution for $\Delta \p$
%%%%%%%%%%%%%%%%%%%%
%%%%%%%%%%%%%%%%%%%%
\begin{equation}\label{eq:ecc-lk-gauss-newton-fa}
    \Delta \p = \bb{H}^{-1} \J^{\top} \left[ \frac{\norm{I(\p)}^2 - {I(\p)}^{\top} \bb{Q} I(\p)}{\hat{T}^{\top} I(\p) - \hat{T}^{\top} \bb{Q} I(\p)} \hat{T} - I(\p) \right]
\end{equation}
%%%%%%%%%%%%%%%%%%%%
where the Hessian as defined as before, $\bb{H} = \J^{\top} \J$ and $\bb{Q}$ is
an orthogonal projection operator on the Jacobian $\J$,
defined as $\bb{Q} = \J {(\J^\top \J)}^{-1} \J^\top$. Finally, the
parameter update is performed additively via $\p \leftarrow \p + \Delta \p$.

In fact, the $\Delta p$ update given in~\cite{evangelidis2008parametric}
is more complex than
(\ref{eq:ecc-lk-gauss-newton-fa}), as it seeks to find an upper bound on the
correlation between the two images. However, in the case where
(\ref{eq:ecc-lk-gauss-newton-fa}) does not apply, it is unlikely that the
algorithm is able to converge. For this reason, we only consider the update
equation presented in (\ref{eq:ecc-lk-gauss-newton-fa}).
See \cref{alg:lk_2d_ecc_fa} for an explanation of the iterative steps required
for solving the problem.
%%%%%%%%%%%%%%%%%%%%%%%%%%%%%%%%%%%%%%%%%%%%%%%%%%%%%%%%%%%%%%%%%%%%%%%%%%%%%%%%
\begin{algorithm}[t]
\caption{Forward Additive 2D ECC Lucas-Kanade Algorithm.}
\label{alg:lk_2d_ecc_fa}
    {\textbf{Input:} Input Image $I$, Template Image $T$} \\
    {\textbf{Output:} Warp parameter vector $\p$}  \\
    \begin{algorithmic}[1]
        \State{}Initialise: $\p = \zero$, $\hat{T} = \frac{T(\zero)}{\norm{T(\zero)}}$\\
        \While{$\norm{\Delta \p} > \epsilon$} %(Outer loop)
            \State{}Warp $I$ with current estimate of the parameters $I(\p)$
            \State{}Compute the error image with the normalized template $I(\p) - \hat{T}$
            \State{}Warp the gradient of the input image $\nabla I(\p)$
            \State{}Compute the warp Jacobian at the current parameter estimate $\frac{\partial \mathcal{W}}{\partial \p}$
            \State{}Compute the steepest descent images Jacobian $\J = \nabla I(\p) \frac{\partial \mathcal{W}}{\partial \p}$
            \State{}Compute the Hessian $\left[ \J^{\top} \J \right]$
            \State{}Compute the Orthogonal projection operator $\bb{Q} = \J {(\J^\top
\J)}^{-1} \J^\top$
            \State{}Compute the $\Delta \p$ using \cref{eq:ecc-lk-gauss-newton-fa}
            \State{}Compute the additive update $\p \leftarrow \p + \Delta \p$
        \EndWhile{}
    \end{algorithmic}
\end{algorithm}
%%%%%%%%%%%%%%%%%%%%%%%%%%%%%%%%%%%%%%%%%%%%%%%%%%%%%%%%%%%%%%%%%%%%%%%%%%%%%%%%
%%%%%%%%%%%%%%%%%%%%%%%%%%%%%%%%%%%%%%%%%%%%%%%%%%%%%%%%%%%%%%%%%%%%%%%%%%%%%%%%
\paragraph{Inverse Compositional LK}\label{subsubsec:lk-ic}
%%%%%%%%%%%%%%%%%%%%%%%%%%%%%%%%%%%%%%%%%%%%%%%%%%%%%%%%%%%%%%%%%%%%%%%%%%%%%%%%
%%%%%%%%%%%%%%%%%%%%%%%%%%%%%%%%%%%%%%%%%%%%%%%%%%%%%%%%%%%%%%%%%%%%%%%%%%%%%%%%
\begin{algorithm}[t]
\caption{Inverse Compositional 2D Lucas-Kanade Algorithm.}
\label{alg:lk_2d_ic}
    {\textbf{Input:} Input Image $I$, Template Image $T$} \\
    {\textbf{Output:} Warp parameter vector $\p$}  \\
    \begin{algorithmic}[1]
        \State{}Initialise: $\p = \zero$\\
        \State{Precompute:}
        \State{}\quad Gradient of the unwarped template $\nabla T(\zero)$
        \State{}\quad Warp Jacobian at identity $\frac{\partial \mathcal{W}(\bb{x};\zero)}{\partial \p}$
        \State{}\quad Steepest descent images $\frac{\partial T(\zero)}{\partial \p} = \nabla T(\zero) \frac{\partial \mathcal{W}(\bb{x};\zero)}{\partial \p}$
        \State{}\quad Hessian $\left[ {\frac{\partial T(\zero)}{\partial \p}}^{\top} \frac{\partial T(\zero)}{\partial \p} \right]$ \\
        \While{$\norm{\Delta \p} > \epsilon$}
            \State{}Warp $I$ with current estimate of the parameters $I(\p)$
            \State{}Compute the error image $I(\p) - T(\zero)$
            \State{}Compute the $\Delta \p$ using \cref{eq:l2-lk-gauss-newton-ic}
            \State{}Compute the compositional update $\mathcal{W}(\bb{x};\p) \leftarrow \mathcal{W}(\bb{x};\p) \circ {\mathcal{W}(\bb{x};\Delta \p)}^{-1}$
        \EndWhile{}
    \end{algorithmic}
\end{algorithm}
%%%%%%%%%%%%%%%%%%%%%%%%%%%%%%%%%%%%%%%%%%%%%%%%%%%%%%%%%%%%%%%%%%%%%%%%%%%%%%%%
The inverse compositional algorithm, proposed by \citet{baker2004lucas},
performs a compositional update of the warp and
linearises over the template rather than the input image. Linearisation of the
template image causes the gradient in the steepest descent images term to become
fixed. The compositional update of the warp assumes linearisation of the term
$\frac{\partial \mathcal{W}(\bb{x};\zero)}{\partial \p}$, which is also
fixed. Therefore, the entire Jacobian term, and by extension the Hessian matrix,
are also fixed. Similar to the $\ltwo$ SSD algorithm described in
in the forward additive case, we pose the objective function as:
%%%%%%%%%%%%%%%%%%%%
\begin{equation}\label{eq:l2-lk-ic}
    \argmin_{\p} \norm{T(\Delta \p) - I(\p)}^2
\end{equation}
%%%%%%%%%%%%%%%%%%%%
where we notice that the roles of the template and input image have been
swapped. Assuming an inverse compositional update to the warp,
$\mathcal{W}(\bb{x};\p) \leftarrow \mathcal{W}(\bb{x};\p) \circ {\mathcal{W}(\bb{x};\Delta \p)}^{-1}$
and linearisation around the template, (\ref{eq:l2-lk-ic}) can be expanded as:
%%%%%%%%%%%%%%%%%%%%
\begin{equation}\label{eq:l2-lk-linearised-ic}
    \argmin_{\p} \quad \norm{I(\p) - \frac{\partial T(\zero)}{\partial \p} \Delta \p - T(\zero)}^2
\end{equation}
%%%%%%%%%%%%%%%%%%%%
Solving for $\Delta \p$ is identical to (\ref{eq:l2-lk-gauss-newton-fa}), except
the Jacobian and Hessian have been pre-computed
%%%%%%%%%%%%%%%%%%%%
\begin{equation}\label{eq:l2-lk-gauss-newton-ic}
    \Delta \p = \bb{H}^{-1} \frac{\partial T(\zero)}{\partial \p}^{\top} \left[ I(\p) - T(\zero) \right]
\end{equation}
%%%%%%%%%%%%%%%%%%%%
See \cref{alg:lk_2d_ic} for an explanation of the iterative steps required
for solving the problem.

The ECC can also be described as an inverse compositional algorithm, by
performing the same update to the warp and simply swapping the roles of the
template and reference image. In short, solving ECC in the inverse compositional
case becomes
%%%%%%%%%%%%%%%%%%%%
\begin{equation}\label{eq:ecc-lk-gauss-newton-ic}
    \Delta \p = \bb{H}^{-1} \frac{\partial \hat{T}}{\partial \p}^{\top} \left[ \frac{\norm{\hat{T}}^2 - \hat{T}^{\top} \bb{Q} \hat{T}}{{I(\p)}^{\top} \hat{T} - {I(\p)}^{\top} \bb{Q} \hat{T}} I(\p) - \hat{T} \right]
\end{equation}
%%%%%%%%%%%%%%%%%%%%
where $\bb{Q}$ is as before, except $\J = \frac{\partial \hat{T}}{\partial \p}$.
Any term involving $\hat{T}$ is fixed and pre-computable, so the reduction of
calculations per-iteration is substantial.

It is worth noting that not every family of warps is suitable for the inverse
compositional approach. The warp must belong to a family that forms a group, and
the identity warp must exist in the set of possible warps. For more complex
warps, such as piecewise affine and thin plate spline warping, approximations to
the inverse compositional updates have been
proposed~\cite{matthews2004active,papandreou2008adaptive}.
%%%%%%%%%%%%%%%%%%%%%%%%%%%%%%%%%%%%%%%%%%%%%%%%%%%%%%%%%%%%%%%%%%%%%%%%%%%%%%%%
%%%%%%%%%%%%%%%%%%%%%%%%%%%%%%%%%%%%%%%%%%%%%%%%%%%%%%%%%%%%%%%%%%%%%%%%%%%%%%%%
\subsection{2D Lucas-Kanade Alignment of 2.5D Images}\label{subsec:singl_img_lk_2d}
%%%%%%%%%%%%%%%%%%%%%%%%%%%%%%%%%%%%%%%%%%%%%%%%%%%%%%%%%%%%%%%%%%%%%%%%%%%%%%%%
In this section we propose to use normals as a robust descriptor for the
alignment of 2.5D depth data. As previously discussed, 2.5D data provides a depth
or height value per-pixel in the image domain and thus is discretised into
a 2D image. Therefore, construction of a Lucas-Kanade algorithm for the alignment
of depth data follows directly form the Lucas-Kanade algorithm descriptions
outlined in the previous section. The only difference is that, instead of
computing the SSD error on the colour or grayscale pixel data we instead compute
the SSD error on the depth values.

In order to augment the classical Affine Lucas-Kanade algorithm to use normals
we follow the methodology proposed by \citet{antonakos2015feature}. That is,
we treat the normals as a feature that is extracted from the depth values
and linearise the feature as though it were a multi-channel input. In this case,
the vector valued SSD error described in the previous section simply
becomes the concatenation of all of the channels of the multi-channel
``feature image''. Specifically, in the case of normals, we would assume
that the input image $I$ becomes $f(I)$ where
$f(I) = {[n^x_1, n^y_1, n^z_1, \ldots, n^x_D, n^y_D, n^z_D]}^T$ is defined as
the feature extraction function that computes the surface normals from the
image. Given that no statistical model of appearance is employed in Affine
Lucas-Kanade, using the normals directly as feature is equivalent to
the inner product kernel described in \cref{subsubsec:sing_img_ip_kernel}.
More explicitly, we define two new feature description methods which are
restated from our kernels described in \cref{subsubsec:singl_img_ca_kpca}.
We provide the following definitions which are defined to operate on a single
depth pixel for simplicity:
%%%%%%%%%%%%%%%%%%%%%%%%%%%%
\begin{equation}
    \begin{aligned}\label{eq:normal_feature_functions}
        f_{\ipname}(i)    &= {[n_x, n_y, n_z]}^T \\
        f_{\sphername}(i) &= {[\cos \phi \sin \phi \cos \theta \sin \theta]}^T
    \end{aligned}
\end{equation}
%%%%%%%%%%%%%%%%%%%%%%%%%%%%
where
%%%%%%%%%%%%%%%%%%%%%%%%%%%%
\begin{equation}
    \begin{aligned}\label{eq:normalised-spherical}
        \cos \phi   &=& \tilde{n_x} \;\;\;\; \sin \phi   &=& \tilde{n_y} \\
        \cos \theta &=& \tilde{n_z} \;\;\;\; \sin \theta &=& \sqrt{1 - {\tilde{n_z}}^2}
    \end{aligned}
\end{equation}
%%%%%%%%%%%%%%%%%%%%%%%%%%%%
and $\tilde{n_x} = \frac{n_x}{\sqrt{n_x^2 + n_y^2}}$,
$\tilde{n_y} = \frac{n_y}{\sqrt{n_x^2 + n_y^2}}$,
$\tilde{n_z} = \frac{n_z}{\sqrt{n_x^2 + n_y^2 + n_z^2}}$.
This normalisation of each component is done to suppress any magnitude
contribution from orientation.

\textbf{Applicability of the AEP and PGA kernel to parametric image alignment.}
Although it is simple to define our proposed kernels as descriptors suitable
for parametric image alignment, this is not the case for the
AEP~\cite{smith2006recovering} and PGA~\cite{smith2008facial} projection
operations. Firstly, we note that both the AEP and PGA operators are identical
in their mapping function with the primary difference being in the computation
of the per-pixel mean estimate required for computing the tangent plane. This
is in contrast to our proposed kernels which are independent per-pixel. This
reliance on a mean estimate for computing the differences is precisely what
makes these operations unsuitable for alignment. In the case of template based
Lucas-Kanade it is unclear where the estimate of the mean should computed as
only a single image is provided as a template and therefore a mean cannot
be computed. Furthermore, even if a mean is constructed from a training set
consisting of a single individual, it is highly likely that the deviation
from this mean face for a given input image will be very close to this mean. This
means that the distance from the neutral template image and the likely neutral
mean will be extremely small. In fact, in the worst case where the neutral does
not deviate from the mean neutral face at all the template image in ``feature''
space would be completely zero. This is because there is zero distance in
the tangent plane between the mean estimate and the template. Furthermore,
since the tangent projection already measures a distance, this means that
the similarity between the template and the current estimate is not well
described by the feature. For these reasons, we do not consider the AEP or PGA
kernels in the following parametric alignment investigation.

As discussed, we also propose to use the AAM extension of the classical
Affine Lucas-Kanade algorithm as proposed by \citet{matthews2004active}.
Therefore, the remainder of the section describes how to construct an AAM
using a multi-channel image.
%%%%%%%%%%%%%%%%%%%%%%%%%%%%%%%
\subsubsection{Lucas-Kanade AAMs}\label{subsubsec:2d-lk-aams}
%%%%%%%%%%%%%%%%%%%%%%%%%%%%%%%
An AAM is defined by a shape, appearance and a motion model. The shape model is
typically learnt by annotating $N$ fiducial points, $\bb{s} = [x_1, y_1,
\ldots, x_N, y_N]^T$ on each image in a set of training images. PCA is then
applied to these points and the shape, $\bb{s}$, can be expressed as a
base shape $\bb{s}_0$ plus a linear combination of $P$ shape vectors,
$\bb{s}_i$:
%%%%%%%%%%%%%%%%%%%%%%%%%%%%
\begin{equation}\label{eq:aam-shape-model}
    \bb{s} = \bb{s}_0 + \sum^P_{i=1} p_i \bb{s}_i
\end{equation}
%%%%%%%%%%%%%%%%%%%%%%%%%%%%
where $p_i$ are the shape coefficients. The appearance model is learnt by first
warping each training image to the reference frame defined by $\bb{s}_0$
to yield a set of shape-free textures. Each image is warped using an appropriate
non-rigid warping function such as piecewise affine \cite{cootes2001active} or thin
plate splines \cite{papandreou2008adaptive}. PCA is applied to the shape-free textures to
yield a set of $M$ appearance vectors. The appearance vectors are defined for
each pixel inside $\bb{s}_0$ when $\p = \zero$. Therefore, the
appearance, $A_\lambda(\zero)$, can be expressed as a base appearance,
$A_0(\zero)$, plus a linear combination of $M$ appearance vectors:
%%%%%%%%%%%%%%%%%%%%%%%%%%%%
\begin{equation}\label{eq:aam-appearance-model}
    A_{\blambda}(\zero) = A_0(\zero) + \bb{A} \blambda
\end{equation}
%%%%%%%%%%%%%%%%%%%%%%%%%%%%
where $\bb{A} = [A_1(\zero), \ldots, A_M(\zero)]$, the matrix of
concatenated appearance vectors, $\blambda = [\lambda_0, \ldots, \lambda_M]^T$,
the vector of appearance parameters, and thus $\bb{A} \blambda =
\sum^M_{i=1} \lambda_i A_i(\zero)$. Given a test image, $I$, fitting an AAM
entails estimating the parameters $\p = [p_0, \ldots, p_P]^T$ and $\blambda$.
Formally, the AAM objective function is
%%%%%%%%%%%%%%%%%%%%%%%%%%%%
\begin{equation}\label{eq:aam-objective}
    \argmin_{\p,\blambda} \norm{I(\p) - A_{\blambda}(\zero)}^2
\end{equation}
%%%%%%%%%%%%%%%%%%%%%%%%%%%%
A number of approaches have been proposed to minimise this objective function
\cite{gross2005generic,matthews2004active,papandreou2008adaptive}, the most popular of
which is the project-out inverse compositional algorithm (PIC)
\cite{cootes2001active,amberg2009compositional} due to its efficiency. Although efficient, PIC
is unable to perform well under unseen variation and therefore we have chosen to
use the alternating simultaneous approach described in \cite{matthews2004active}.
%%%%%%%%%%%%%%%%%%%%%%%%%%%%%%%
\subsubsection{Simultaneous IC Algorithm}\label{subsec:aam-simultaneous}
%%%%%%%%%%%%%%%%%%%%%%%%%%%%%%%
The simultaneous algorithm \cite{gross2005generic} finds both the $\Delta \p$ and
$\Delta \blambda$ updates simultaneously. This involves iteratively solving for
$\Delta \p$ and $\Delta \blambda$ by linearising \cref{eq:aam-objective} such
that
%%%%%%%%%%%%%%%%%%%%%%%%%%%%
\begin{equation}\label{eq:aam-simultaneous-linearise}
    \argmin_{\Delta \p, \Delta \blambda} \norm{I(\p) - A_{\blambda}(\zero) - \frac{\partial A_{\blambda}(\zero)}{\partial \p} \Delta \p - \bb{A} \blambda}^2
\end{equation}
%%%%%%%%%%%%%%%%%%%%%%%%%%%%
Let $\Delta \bb{q} = [{\Delta \p}^T, {\Delta \blambda}^T]^T$ be the
concatenated vector of parameters. By performing a compositional update to the
warp parameters and an additive update to the appearance parameters, $\Delta
\bb{q}$ can be found simultaneously via
%%%%%%%%%%%%%%%%%%%%%%%%%%%%
\begin{equation}\label{eq:aam-simultaneous-deltaq-update}
    \Delta \bb{q} = \bb{H}_{\bb{q}}^{-1} \bb{J}_{\bb{q}}^{T} \left[ I(\p) - A_{\blambda}(\zero) \right]
\end{equation}
%%%%%%%%%%%%%%%%%%%%%%%%%%%%
where $\bb{H}_{\bb{q}} = \bb{J}_{\bb{q}}^{T}
\bb{J}_{\bb{q}}$ and $\bb{J}_{\bb{q}} = \left[
{\frac{\partial A_{\blambda}(\zero)}{\partial \p}}^T, \bb{A} \right]$.
Due to the additive update of the appearance parameters, $\blambda \leftarrow
\blambda + \Delta \blambda$, and thus the dependence of
$\bb{J}_{\bb{q}}$ on $\blambda$, the Jacobian and Hessian
matrices must be recomputed at each step. Although this is much less efficient
than the PIC algorithm, it has been shown to given excellent fitting performance
in practise.
%%%%%%%%%%%%%%%%%%%%%%%%%%%%%%%
\subsubsection{Alternating IC Algorithm}\label{subsec:aam-alternating}
%%%%%%%%%%%%%%%%%%%%%%%%%%%%%%%
The variation of the simultaneous inverse compositional algorithm proposed in
\cite{matthews2004active} solves for the shape and appearance updates in an
alternating manner, as
%%%%%%%%%%%%%%%%%%%%%%%%%%%%
\begin{equation}
    \begin{aligned}\label{eq:aam-alternating}
        \Delta \tilde{\p} &=       \argmin_{\Delta \p}       \norm{I(\p) - A_{\blambda}(\zero) - \frac{\partial A_{\blambda}(\zero)}{\partial \p} \Delta \p}^2_{\bb{I} - \bb{A} \bb{A}^T} \\
        \Delta \tilde{\blambda} &= \argmin_{\Delta \blambda} \norm{I(\p) - A_{\blambda}(\zero) - \frac{\partial A_{\blambda}(\zero)}{\partial \p} \Delta \tilde{\p} - \bb{A} \Delta \blambda}^2
    \end{aligned}
\end{equation}
%%%%%%%%%%%%%%%%%%%%%%%%%%%%
where $\bb{I} - \bb{A} \bb{A}^T$ represents the
projecting out of the appearance basis $\bb{A}$ as described for the PIC
algorithm in \cite{matthews2004active}. The update of $\Delta \p$ is given by
%%%%%%%%%%%%%%%%%%%%%%%%%%%%
\begin{equation}\label{eq:aam-alternating-deltap-update}
        \Delta \p = {\tilde{\bb{H}}}^{-1} {\tilde{\bb{J}}}^{T} \left[ I(\p) - A_0(\zero) \right]
\end{equation}
%%%%%%%%%%%%%%%%%%%%%%%%%%%%
where ${\tilde{\bb{H}}} = {\tilde{\bb{J}}}^{T}
\tilde{\bb{J}}$ and $\tilde{\bb{J}} = (\bb{I} -
\bb{A} \bb{A}^T) \left[ {\frac{\partial
A_{\blambda}(\zero)}{\partial \p}}^T, \bb{A} \right]^T$. Given the
current estimate for the optimum, $\Delta \tilde{\p} = \Delta \p$, we can solve
the second optimisation equation for $\Delta \blambda$, as
%%%%%%%%%%%%%%%%%%%%%%%%%%%%
\begin{equation}\label{eq:aam-alternating-deltalambda-update}
        \Delta \blambda = \bb{A}^T \left[ I(\p) - A_{\blambda}(\zero) - \frac{\partial A_{\blambda}(\zero)}{\partial \p} \Delta \tilde{\p} \right]
\end{equation}
%%%%%%%%%%%%%%%%%%%%%%%%%%%%
As for the simultaneous algorithm, the warp parameters are update with a
compositional update and the appearance parameters are updated additively.
%%%%%%%%%%%%%%%%%%%%%%%%%%%%%%%
\subsubsection{Normal Kernel Algorithm}\label{subsec:aam-normal-kernel}
%%%%%%%%%%%%%%%%%%%%%%%%%%%%%%%
The key difference between image alignment using LK and an AAM is the use of the
statistical prior to handle unseen variation. Creating this prior for depth maps
can be handled identically to creating a prior for image textures. However,
depth maps, much like textures, are heavily affected by outliers. Specifically,
outliers are defined as anything that the appearance model cannot reconstruct
because (i) it was not seen in the training set, (ii) it does not belong in the
space of faces (e.g. occlusions), (iii) it was excluded from the appearance
bases as noise when reducing the number of principal components. However, as
described in \cref{subsubsec:singl_img_ca_kpca}, we have proposed a set of
kernels within a KPCA framework that enable component analysis on normals.
Therefore, by using these robust kernels as the appearance priors in AAMs, a
robust deformable fitting can be performed.

Given a set of depth maps, we calculate the normals and warp them to extract
a set of shape-free normals. By applying one of the projection operators,
$\Phi(\bb{x})$, from \cref{subsubsec:singl_img_ca_kpca}, we can
redefine the appearance model as
%%%%%%%%%%%%%%%%%%%%%%%%%%%%
\begin{equation}\label{eq:aam-kernel-appearance-model}
    A_{\blambda}^{\Phi}(\zero) = \bb{A}^{\Phi} \blambda
\end{equation}
%%%%%%%%%%%%%%%%%%%%%%%%%%%%
where $\bb{A}^{\Phi} = [A_0^{\Phi}(\zero), \ldots, A_M^{\Phi}(\zero)]$
is the matrix of concatenated appearance vectors gained from applying KPCA to
the shape-free normals. Notice that the first eigenvector of our appearance
model represents the mean face. This is because, as described in
\cref{subsubsec:singl_img_ca_kpca}, we do not perform a mean subtraction
when computing the component analysis. This has the effect of slightly
simplifying the AAM derivations such that the base appearance, $A_0(\zero)$,
does not explicitly feature. For example, the updates in the alternating
algorithm become:
%%%%%%%%%%%%%%%%%%%%%%%%%%%%
\begin{equation}
    \begin{aligned}\label{eq:aam-kernel-alternating-update}
        \Delta \p       &= {\bb{H}^{\Phi}}^{-1} {\bb{J}^{\Phi}}^{T} I^{\Phi}(\p) \\
        \Delta \blambda &= {\bb{A}^{\Phi}}^T \left[ I^{\Phi}(\p) - A^{\Phi}_{\blambda}(\zero) - \frac{\partial A^{\Phi}_{\blambda}(\zero)}{\partial \p} \Delta \tilde{\p} \right]
    \end{aligned}
\end{equation}
%%%%%%%%%%%%%%%%%%%%%%%%%%%%
where
%%%%%%%%%%%%%%%%%%%%%%%%%%%%
\begin{equation*}
    \begin{aligned}
        \bb{J}^{\Phi} &= (\bb{I} - \bb{A}^{\Phi} {\bb{A}^{\Phi}}^T) \left[ {\frac{\partial A^{\Phi}_{\blambda}(\zero)}{\partial \p}}^T, \bb{A}^{\Phi} \right]^T \\
        \bb{H}^{\Phi} &= {\bb{J}^{\Phi}}^{T} \bb{J}^{\Phi}
    \end{aligned}
\end{equation*}
%%%%%%%%%%%%%%%%%%%%%%%%%%%%
which only differs from the formulation given in \cref{subsec:aam-alternating} in assuming the use of kernel projected spaces and that the mean
appearance is implicitly part of the appearance bases. The shape model is
unchanged from the original AAM formulation.
%%%%%%%%%%%%%%%%%%%%%%%%%%%%%%%%%%%%%%%%%%%%%%%%%%%%%%%%%%%%%%%%%%%%%%%%%%%%%%%%
%%%%%%%%%%%%%%%%%%%%%%%%%%%%%%%%%%%%%%%%%%%%%%%%%%%%%%%%%%%%%%%%%%%%%%%%%%%%%%%%
\subsubsection{Experiments}\label{subsubsec:singl_img_2d_lk_experiments}
%%%%%%%%%%%%%%%%%%%%%%%%%%%%%%%%%%%%%%%%%%%%%%%%%%%%%%%%%%%%%%%%%%%%%%%%%%%%%%%%
%%%%%%%%%%%%%%%%%%%%%%%%%%%%%%%%%%%%%%%%%%%%%%%%%%%%%%%%%%%%%%%%%%%%%%%%%%%%%%%%

%%%%%%%%%%%%%%%%%%%%%%%%%%%%%%%%%%%%%%%%%%%%%%%%%%%%%%%%%%%%%%%%%%%%%%%%%%%%%%%%

%%%%%%%%%%%%%%%%%%%%%%%%%%%%%%%%%%%%%%%%%%%%%%%%%%%%%%%%%%%%%%%%%%%%%%%%%%%%%%%%
\subsection{Robust 3D Lucas-Kanade Alignment}\label{subsec:singl_img_lk_3d}
%%%%%%%%%%%%%%%%%%%%%%%%%%%%%%%%%%%%%%%%%%%%%%%%%%%%%%%%%%%%%%%%%%%%%%%%%%%%%%%%
%%%%%%%%%%%%%%%%%%%%%%%%%%%%%%%%%%%%%%%%%%%%%%%%%%%%%%%%%%%%%%%%%%%%%%%%%%%%%%%%
